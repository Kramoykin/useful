\section*{Заключение}  \label{sec:concl} 
\addcontentsline{toc}{section}{Заключение}

С целью калибровки электромагнитного калориметра \textit{ECAL} эксперимента \textit{NA64}, а также получения зависимости предельного энергетического разрешения калориметра от энергии пучка было осуществлено численное моделирование с помощью пакета \textit{GEANT4}. Помимо этого проводилось изучение энергетических утечек.

В результате моделирования энергетического разрешения выяснилось, что зависимость разрешения от энергии инициирующей частицы содержит постоянный член, неучтённый в моделях работы [16]. Это можно объяснить неустранимыми энергетическими утечками в материале калориметра или ошибкой округления при суммировании низкоэнергетической компоненты спектра. С увеличением размеров калориметра, энергия поглощенная в калориметре увеличивается, однако недостаток энергии около \mbox{200 МэВ} не удается снизить таким образом. Предположительно, его формируют слабо взаимодействующие частицы, рождающиеся в фотоядерных реакциях.

Рассмативаемый алгоритм извлечения калибровочных коэффициентов предполагает параметризацию электромагнитного ливня, развивающегося в детекторе. Для этого было осуществлено моделирование, после чего полученный профиль ливня аппроксимировался факторизованной функцией, полученной в приближении гомогенного калориметра. После применения калибровки на экспериментальных данных, был получен результат, свидетельствующий о значительной количественной ошибке в калибровке, однако в дальнейшем планируется учесть рассмотренные в работе эффекты связанные с анизотропией излучения в веществе калориметра, конверсией типов частиц и светосбором с целью повышения точности реконструкции.
