\newpage
\section{Извлечение калибровочных коэффициентов}  \label{chap8}

На основе набора данных калибровочной статистики, можно получить значения энерговыделения в калориметре в результате бомбардировки электронами каждой из 36 ячеек. После чего, осуществляя отбор по отдельным ячейкам, получили распределения энергии, поглощенной в каждой ячейке. Такое распределение для одной из центральных ячеек основной части калориметра показано на рис. \ref{fig:areaNoCat}.

%%% Raw energy distribution for 1-3-3
\begin{figure}[H]
    \centering
    \includegraphics[width=0.75\textwidth]{areaNoCat133Log.png}
    \caption{Распределение энерговыделения в одной из центральных ячеек осовной части калориметра}
    \label{fig:areaNoCat}
\end{figure}

В распределении выделяются три области:
\begin{enumerate}[wide]
\item Область низких значений, первый пик. Ей соответствуют события, в которых основное энерговыделение произошло в других ячейках, а сигнал в рассматриваемой ячейке обусловлен поглощением энергии достигших ее вторичных частиц.
\item Центральная область. Соответствует сигналу ячейки, через которую проходит ось распрстранения электромагнитного ливня при нормальном попадании пучка электронов в детектор. Исключительно значения из этой области должны учитываться при вычислении калибровочных коэффициентов.
\item Область высоких значений. Вклад в эту область вносят два или более попадания частицы в детектор, разнесенные во времени на достаточно малый промежуток, из-за чего они обрабатываются как одно событие. Такие сигналы следует дискриминировать как зашумляющие.
\end{enumerate}

Для того, чтобы обрабатывать события, соответствующие только центральному пику основного сигнала, был реализован алгоритм, поиска и отбора пиков, результат использования которого показан на рис. \ref{fig:areaCat}.
%%% Central peak from energy distribution for 1-3-3
\begin{figure}[H]
    \centering
    \includegraphics[width=0.75\textwidth]{areaCat133.png}
    \caption{Распределение энерговыделения в одной из центральных ячеек осовной части калориметра после дискриминации вторичных пиков}
    \label{fig:areaCat}
\end{figure}
После выделения центрального пика распределения, для каждой ячейки был получен интервал значений энерговыделения, подходящий для расчета калибровочных коэффициентов. 

Калибровочный коэффициент для каждой ячейки вычислялся как отношение долей энерговыделения, расчет которых описан в предыдущем разделе, к значению энерговыделения из распределения. Таким образом были получены распределения калибровочных коэффициентов (рис. \ref{fig:coeffDist}). Распределения коэффициентов аппроксимировались нормальным распределением, положение максимума которого принималось за усредненное значение коэффициента.

%%% Coefficients distribution for 1-3-3, fitted by gaussian
\begin{figure}[H]
    \centering
    \includegraphics[width=0.75\textwidth]{coef133.jpg}
    \caption{Распределение калибровочных коэффициентов для одной из центральных ячеек осовной части калориметра}
    \label{fig:coeffDist}
\end{figure}

Рассчитанные значение калибровочных коэффициентов для каждой ячейки предливневой и главной части электромагнитного калориметра ECAL представлены в таблице \ref{tab:calib-coeffs}.

%%% Calib-coefficient and RMS table
\begin{table}[H]
\footnotesize
\centering
\caption{Калибровочные коэффициенты электромагнитного калориметра ECAL}
\label{tab:calib-coeffs}
\begin{tabular}{|c|c|c|c|c|c|c|c|}
\hline
\multicolumn{2}{|c|}{\begin{tabular}[c]{@{}c@{}}Индексы \\ ячейки\end{tabular}} & \begin{tabular}[c]{@{}c@{}}Калибровочный\\ коэффициент\end{tabular} & \begin{tabular}[c]{@{}c@{}}Среднеквадратичное\\ отклонение\end{tabular} & \multicolumn{2}{c|}{\begin{tabular}[c]{@{}c@{}}Индексы\\  ячейки\end{tabular}} & \begin{tabular}[c]{@{}c@{}}Калибровочный\\ коэффициент\end{tabular} & \begin{tabular}[c]{@{}c@{}}Среднеквадратичное\\ отклонение\end{tabular} \\ \hline
\textit{x}                                      & \textit{y}                                      & \multicolumn{2}{c|}{Предливневая часть}                                                                                                       & \textit{x}                                      & \textit{y}                                     & \multicolumn{2}{c|}{Основная часть}                                                                                                           \\ \hline
0                                      & 0                                      & 0.00036                                                             & 0.00030                                                                 & 0                                      & 0                                     & 0.00412                                                             & 0.00048                                                                 \\ \hline
0                                      & 1                                      & 0.00035                                                             & 0.00028                                                                 & 0                                      & 1                                     & 0.00402                                                             & 0.00051                                                                 \\ \hline
0                                      & 2                                      & 0.00036                                                             & 0.00021                                                                 & 0                                      & 2                                     & 0.00394                                                             & 0.00046                                                                 \\ \hline
0                                      & 3                                      & 0.00030                                                             & 0.00018                                                                 & 0                                      & 3                                     & 0.00451                                                             & 0.00046                                                                 \\ \hline
0                                      & 4                                      & 0.00026                                                             & 0.00017                                                                 & 0                                      & 4                                     & 0.00435                                                             & 0.00046                                                                 \\ \hline
0                                      & 5                                      & 0.00029                                                             & 0.00019                                                                 & 0                                      & 5                                     & 0.00392                                                             & 0.00041                                                                 \\ \hline
1                                      & 0                                      & 0.00036                                                             & 0.00029                                                                 & 1                                      & 0                                     & 0.00379                                                             & 0.00046                                                                 \\ \hline
1                                      & 1                                      & 0.00032                                                             & 0.00027                                                                 & 1                                      & 1                                     & 0.00377                                                             & 0.00041                                                                 \\ \hline
1                                      & 2                                      & 0.00028                                                             & 0.00018                                                                 & 1                                      & 2                                     & 0.00459                                                             & 0.00043                                                                 \\ \hline
1                                      & 3                                      & 0.00032                                                             & 0.00020                                                                 & 1                                      & 3                                     & 0.00439                                                             & 0.00048                                                                 \\ \hline
1                                      & 4                                      & 0.00029                                                             & 0.00018                                                                 & 1                                      & 4                                     & 0.00401                                                             & 0.00038                                                                 \\ \hline
1                                      & 5                                      & 0.00025                                                             & 0.00017                                                                 & 1                                      & 5                                     & 0.00438                                                             & 0.00045                                                                 \\ \hline
2                                      & 0                                      & 0.01000                                                             & 0.00001                                                                 & 2                                      & 0                                     & 0.00389                                                             & 0.00050                                                                 \\ \hline
2                                      & 1                                      & 0.00037                                                             & 0.00030                                                                 & 2                                      & 1                                     & 0.00378                                                             & 0.00030                                                                 \\ \hline
2                                      & 2                                      & 0.00032                                                             & 0.00028                                                                 & 2                                      & 2                                     & 0.00405                                                             & 0.00038                                                                 \\ \hline
2                                      & 3                                      & 0.00031                                                             & 0.00019                                                                 & 2                                      & 3                                     & 0.00385                                                             & 0.00036                                                                 \\ \hline
2                                      & 4                                      & 0.00032                                                             & 0.00019                                                                 & 2                                      & 4                                     & 0.00384                                                             & 0.00038                                                                 \\ \hline
2                                      & 5                                      & 0.00029                                                             & 0.00025                                                                 & 2                                      & 5                                     & 0.00415                                                             & 0.00040                                                                 \\ \hline
3                                      & 0                                      & 0.00033                                                             & 0.00028                                                                 & 3                                      & 0                                     & 0.00462                                                             & 0.00051                                                                 \\ \hline
3                                      & 1                                      & 0.00030                                                             & 0.00025                                                                 & 3                                      & 1                                     & 0.00420                                                             & 0.00048                                                                 \\ \hline
3                                      & 2                                      & 0.00030                                                             & 0.00019                                                                 & 3                                      & 2                                     & 0.00402                                                             & 0.00044                                                                 \\ \hline
3                                      & 3                                      & 0.00029                                                             & 0.00019                                                                 & 3                                      & 3                                     & 0.00406                                                             & 0.00041                                                                 \\ \hline
3                                      & 4                                      & 0.00028                                                             & 0.00018                                                                 & 3                                      & 4                                     & 0.00418                                                             & 0.00038                                                                 \\ \hline
3                                      & 5                                      & 0.00039                                                             & 0.00030                                                                 & 3                                      & 5                                     & 0.00456                                                             & 0.00043                                                                 \\ \hline
4                                      & 0                                      & 0.01000                                                             & 0.00000                                                                 & 4                                      & 0                                     & 0.00399                                                             & 0.00049                                                                 \\ \hline
4                                      & 1                                      & 0.00043                                                             & 0.00032                                                                 & 4                                      & 1                                     & 0.00367                                                             & 0.00025                                                                 \\ \hline
4                                      & 2                                      & 0.00035                                                             & 0.00028                                                                 & 4                                      & 2                                     & 0.00398                                                             & 0.00048                                                                 \\ \hline
4                                      & 3                                      & 0.00028                                                             & 0.00023                                                                 & 4                                      & 3                                     & 0.00379                                                             & 0.00044                                                                 \\ \hline
4                                      & 4                                      & 0.00029                                                             & 0.00025                                                                 & 4                                      & 4                                     & 0.00421                                                             & 0.00039                                                                 \\ \hline
4                                      & 5                                      & 0.00027                                                             & 0.00017                                                                 & 4                                      & 5                                     & 0.00422                                                             & 0.00049                                                                 \\ \hline
5                                      & 0                                      & 0.00031                                                             & 0.00028                                                                 & 5                                      & 0                                     & 0.00364                                                             & 0.00027                                                                 \\ \hline
5                                      & 1                                      & 0.00032                                                             & 0.00027                                                                 & 5                                      & 1                                     & 0.00444                                                             & 0.00054                                                                 \\ \hline
5                                      & 2                                      & 0.00026                                                             & 0.00017                                                                 & 5                                      & 2                                     & 0.00423                                                             & 0.00047                                                                 \\ \hline
5                                      & 3                                      & 0.00030                                                             & 0.00025                                                                 & 5                                      & 3                                     & 0.00420                                                             & 0.00045                                                                 \\ \hline
5                                      & 4                                      & 0.00056                                                             & 0.00038                                                                 & 5                                      & 4                                     & 0.00415                                                             & 0.00046                                                                 \\ \hline
5                                      & 5                                      & 0.00032                                                             & 0.00026                                                                 & 5                                      & 5                                     & 0.00465                                                             & 0.00041                                                                 \\ \hline
\end{tabular}%
\end{table}

После применения полученных калибровочных коэффициентов к экспериментальным данным 2017 года, было получено распределение реконструированного энерговыделения в электромагнитном калориметре (рис. \ref{fig:overall}).

%%% ECAL overall energy distribution
\begin{figure}[H]
    \centering
    \includegraphics[width=0.75\textwidth]{overall.jpg}
    \caption{Распределение реконструированного энерговыделения}
    \label{fig:overall}
\end{figure}

Калибровочным моноэнергетическим событиям, не зашумленным попаданиями нескольких частиц во временной интервал записи АЦП, соответствует первый пик распределения, приходящийся на \mbox{40 ГэВ}. Второй и третий пик соответствуют \textit{pile-up} событиям от двух и трех частиц соответственно. Очевидно, что результат, при котором энерговыделение большинства событий от попадания электронного пучка с энергией \mbox{100 ГэВ} в электромагнитный калориметр оценивается в \mbox{40 ГэВ}, свидетельствует о недооценке.

Вероятные причины ошибки:
%%% Miscalibrations explanation
\begin{enumerate}[wide]
\item Неточность \textit{GEANT4} модели электромагнитного калориметра. 
\item Грубое приближение при аппроксимации профиля ливня формулой, полученной для гомогенного калориметра, приводящее к зависимости радиального профиля от глубины ливня.
\item Нелинейность преобразования энергии в веществе сцинтиллятора по отношению к энергии поглощённой в свинце.
\end{enumerate}

Последнюю причину следует разобрать более подробно. Сигнал калориметра образует преимущественно энергия, поглощенная в сцинтилляторе. Это значит, что для калибровки следовало бы использовать только энерговыделение в слоях полиметилметакрилата. При дальнейшем усовершенствовании процедуры калибровки предполагается исключать из рассмотрения энерговыделение в свинце путём введения дополнительного множителя в формулу, аппроксимирующую профиль ливня. Однако понимание этой методической ошибки приводит к некоторым интересным заключениям.

Отбирая события с наименьшим энерговыделением в адронном калориметре \textit{HCAL}, расположенном в телескопе детекторов позади калориметра \textit{ECAL} можно добиться того, что совокупная энергия, поглощенная в электромагнитном калориметре, будет соответствовать событиям с наибольшей герметичностью. Распределения энерговыделения с возрастающим ограничением по энергии, поглощенной в адронном калориметре, представлены на рис. \ref{fig:cutHCAL0}, \ref{fig:cutHCAL1}, \ref{fig:cutHCAL2} и \ref{fig:cutHCAL3}.

%%% No HCAl cut
\begin{figure}[H]
    \centering
    \includegraphics[width=0.75\textwidth]{HCALcut0.png}
    \caption{Распределение реконструированного энерговыделения (ограничение энергии, выделившейся в адронном калориметре 20000 вспом. ед.)}
    \label{fig:cutHCAL0}
\end{figure}

В распределении (рис. \ref{fig:cutHCAL0}) различимы области, соотвествующие аппаратурным шумам, одиночным событиям, двойным и тройным \textit{pile-up} событиям.

%%% The first cut - kill noize
\begin{figure}[H]
    \centering
    \includegraphics[width=0.75\textwidth]{HCALcut1.png}
    \caption{Распределение реконструированного энерговыделения (ограничение энергии, выделившейся в адронном калориметре 2000 вспом. ед.)}
    \label{fig:cutHCAL1}
\end{figure}

В распределении (рис. \ref{fig:cutHCAL1}) пропадает участок аппаратурного шума.

%%% The first cut - kill pileup (almost)
\begin{figure}[H]
    \centering
    \includegraphics[width=0.75\textwidth]{HCALcut2.png}
    \caption{Распределение реконструированного энерговыделения (ограничение энергии, выделившейся в адронном калориметре 400 вспом. ед.)}
    \label{fig:cutHCAL2}
\end{figure}

В распределении (рис. \ref{fig:cutHCAL2}) пропадает область, соответствующая тройным \textit{pile-up} событиям.

%%% The first cut - kill noize
\begin{figure}[H]
    \centering
    \includegraphics[width=0.75\textwidth]{HCALcut3.png}
    \caption{Распределение реконструированного энерговыделения (ограничение энергии, выделившейся в адронном калориметре 300 вспом. ед.)}
    \label{fig:cutHCAL3}
\end{figure}

В распределении (рис. \ref{fig:cutHCAL3}) практически не остается \textit{pile-up}-событий, центральный пик, приходящийся на \mbox{55 ГэВ}, -- основной сигнал. Можно отметить, что положение пика и его форма не изменяются после дискриминации зашумляющих событий. Такому поведению соответствует случай, в котором алгоритм калибровки выстроен верно, но значения коэффициентов определены с методологической ошибкой, обусловленной недооценкой влияния пространственной анизотропии преобразования энергии выделившейся в веществе калориметра в световую.