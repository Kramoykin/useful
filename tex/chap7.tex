\newpage
\section{Аппроксимация профиля ливня}  \label{chap7}  

В результате моделирования электромагнитного ливня были получены значения энерговыделения для каждого сегмента калориметра. После чего осуществлялась аппроксимация профиля ливня функциями (\ref{eq:longProf}) и (\ref{eq:ansatz}), используя процедуры библиотеки \textit{GSL} [17]. Аппроксимированные радиальные профили представлены на рис. \ref{fig:rad1}, \ref{fig:rad40} , \ref{fig:rad85}, продольный -- на рис. \ref{fig:longProf}.

 %%% Radial profile for the 1st cell
\begin{figure}[H]
    \centering
    \includegraphics[width=0.5\textwidth]{rad1.jpg}
    \caption{Радиальный профиль ливня в 1-й ячейке вдоль оси распространения}
    \label{fig:rad1}
\end{figure}

Радиальный профиль в самом начале ливня представляет собой резко убывающую кривую, напоминающую график экспоненты. Это объясняется тем, что в начале ливня число рожденных частиц невелико и образуются они близко к оси, вдоль которой происходит развитие ливня.

 %%% Radial profile for the 40 cell
\begin{figure}[H]
    \centering
    \includegraphics[width=0.5\textwidth]{rad40.jpg}
    \caption{ Радиальный профиль ливня в 40-й ячейке вдоль оси распространения (спадание у нуля является численным артефактом фитирующей функции)}
    \label{fig:rad40}
\end{figure}

Распределение энергии в области интенсивного энерговыделения становится шире и появляется максимум энерговыделения, соответствующий концу первой ячейки в направлении от центра детектора к периферии.

 %%% Radial profile for the 85 cell
\begin{figure}[H]
    \centering
    \includegraphics[width=0.5\textwidth]{rad85.jpg}
    \caption{Радиальный профиль ливня в 85-й ячейке вдоль оси распространения}
    \label{fig:rad85}
\end{figure}

Радиальное распределение энергии в области затухания ливня становится очень гладким со слабо выраженным максимумом во второй ячейке.

 %%% Longitudal profile
\begin{figure}[H]
    \centering
    \includegraphics[width=0.75\textwidth]{long.jpg}
    \caption{Продольный профиль ливня}
    \label{fig:longProf}
\end{figure}

 Трехмерный график энерговыделения в калориметре и аппроксимация энерговыеления факторизованной функцией показаны на рис. \ref{fig:shower3D}.

 %%% Electromagnetic shower profile with fit
\begin{figure}[H]
    \centering
    \includegraphics[width=0.75\textwidth]{shower3D.png}
    \caption{ Аппроксимация электромагнитного ливня факторизованной функцией }
    \label{fig:shower3D}
\end{figure}

Осуществили интегрирование функции, аппроксимирующей профиль ливня, в областях, ограничивающих  ячейку, на которую приходится ось развития ливня, ячейку, имеющую с центральной смежную грань, и ячейку, расположенную относительно центральной по диагонали, для предливневой и основной части калориметра ECAL. Значения, полученные в результате интегрирования, отнесенные к энергии пучка, представлены в таблице \ref{tab:enFrac}.

%%% Table with energy deposit fractions 
\begin{table}[H]
\centering
\caption{Энерговыделение в ячейках калориметра ECAL}
\label{tab:enFrac}
\begin{tabular}{|c|c|c|}
\hline
            & \multicolumn{2}{c|}{$E/E_0$, \%}        \\ \hline
Ячейка      & Предливневая часть & Основная часть \\ \hline
Центральная & 1,55               & 37,79          \\ \hline
Смежная     & 0,16               & 3,93           \\ \hline
Диагональная     & 0,09               & 2,27           \\ \hline
\end{tabular}
\end{table}