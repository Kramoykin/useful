\newpage
 \section{Моделирование энергетического разрешения калориметра}  \label{chap5}  
 
 Моделирование электронного ливня в калориметре \textit{ECAL} производилось с использованием программного пакета \textit{GEANT4}. Ранее в научной группе ТПУ/\textit{NA64} была создана цифровая модель гетерогенного электромагнитного калориметра (рис. \ref{fig:ecal}), состоящего из чередующихся пластин из свинца, толщиной \mbox{1,5 мм}, и сцинтиллятора из полиметилметакрилата, толщиной \mbox{1,55 мм}. Геометрия калориметра и сцентарии моделирования настраивались через соотвествующие конфигурационные файлы. 
 
 %%% Geant4 ECAL model
\begin{figure}[H]
    \centering
    \includegraphics[width=0.75\textwidth]{ECAL-model.jpg}
    \caption{Модель калориметра \textit{ECAL}, \textit{GEANT4}}
    \label{fig:ecal}
\end{figure}

Для вычисления энергетического разрешения калориметра необходимо знать среднеквадратичное отклонение совокупного энерговыделения (высвеченная энергия) от среднего значения (квази-)моноэнергетического пучка, которое выражается как корень из суммы квадратов энерговыделения в ячейках:

%%% STD as sum of std
\begin{equation}\label{eq:sigmaCal}
\sigma = \sqrt{\sum_{i=1}^{36} \sigma_i^2} .
\end{equation}

В свою очередь выражение для среднеквадратичного отклонения энерговыделения в ячейке: 

%%% STD classic formula
\begin{equation}\label{eq:classicSTD}
\sigma_i = \sqrt{\frac{1}{n} \sum_{k=1}^n (x_k - \bar{x})^2},
\end{equation}
где \textit{n} -- число событий; $x_k$ -- энерговыделение в \textit{k}-м событии; $\bar{x}$ -- среднее энерговыделение в ячейке.

Выражение (\ref{eq:classicSTD}) путём простых преобразований сводится к 

%%% STD formula in use
\begin{equation}\label{eq:mySTD}
\sigma_i = \frac{1}{n} 
\left\{ 
\sum_{k=1}^n x_i^2 - \frac{1}{n} 
\left(
 \sum_{k=1}^n x_k \right) ^2 
\right\} .
\end{equation}

Проводилось моделирование попадания электрона в детектор для 500 событий в диапазоне энергий электрона от 10 до \mbox{100 ГэВ}. После чего по формуле (\ref{eq:mySTD}) вычислялось значение энергетического разрешения для заданной энергии. Полученная зависимость энергетического разрешения калориметра ECAL представлена на рис. \ref{fig:enResFit} звездочками.

%%% Energy resolution on the energy dependence
\begin{figure}[H]
    \centering
    \includegraphics[width=0.75\textwidth]{14.jpg}
    \caption{Зависимость энергетического разрешения электромагнитного калориметра \textit{ECAL} от энергии электрона, инициирующего ливень}
    \label{fig:enResFit}
\end{figure}

Зависимость энергетического разрешения сэмплирующего калориметра от энергии пучка можно описать так [16]:

%%% Energy resolution from delPeso
\begin{equation}\label{eq:delPeso}
\frac{\sigma_s}{E_s} = \frac{\sigma_0}{\sqrt{E}}
\left(
\frac{t}{X_t}
\right)
^{\alpha}
\left(
\frac{s}{X_s}
\right)
^{-\beta},
\end{equation}
где $X_t$ и $X_s$ -- радиационные длины поглотителя и сцинтиллятора соответственно, $\mathrm{\sigma_s}$ -- среднеквадратичное отклонение высвеченной энергии, $E_s$ -- высвеченная энергия, $\mathrm{\alpha}$, $\mathrm{\beta}$ и $\mathrm{\sigma_0}$ -- параметры аппроксимирующей функции. 
 
Однако, зависимость, изображенную на рис. \ref{fig:enResFit} удается аппроксимировать формулой (\ref{eq:delPeso}) с достаточной точностью только введя дополнительное постоянное слагаемое. Ключевой особенностью такого слагаемого является то что флуктуация не зависит от энергии ливня. Такая аппроксимация изображена на рис. \ref{fig:enResFit} красной кривой. Необходимость добавления постоянного слагаемого можно объяснить влиянием таких факторов как энергетические утечки, неустранимые флуктуации и ошибка вычисления высвеченной энергии, возникающая при суммировании множества маленьких значений.

