\documentclass[a4paper,14pt]{article}

\usepackage{cmap}           						  % Поиск в PDF
\usepackage{mathtext}                            % Использование в формулах русского языка
\usepackage[T2A]{fontenc}                     % Кодировка
\usepackage[utf8]{inputenc}                   % Кодировка исходного текста
\usepackage[english,russian]{babel}     % Локализация и переносы
\usepackage{indentfirst}                          % Отступы для первого абзаца

%%% Работа с рисунками
\usepackage{graphics}
\usepackage{caption}
\DeclareCaptionLabelSeparator{emdash}{\space\textemdash\space}
\captionsetup{labelsep=emdash}
\captionsetup{font=it}
\graphicspath{ {./pic/after} }

\usepackage{lipsum}
\usepackage{titlesec}
\titlespacing*{\section}{\parindent}{1ex}{1em}
\titlespacing*{\subsection}{\parindent}{1ex}{1em}

\setlength{\parindent}{4pc}

%%% Дополнительная работа с математикой
\usepackage{amsmath,amsfonts,amssymb,amsthm,mathtools} %AMS
\usepackage{icomma} % Умная запятая: $0,2 -- запятая; $0, 2 -- перечисление

%%%Шрифты
\usepackage{euscript}			% Евклид
\usepackage{mathrsfs}			% Красивый матшрифт

%%% Свои команды
\DeclareMathOperator{\sgn}{\mathop{sgn}}

%%% Перенос знаков в формулах (по Львовскому)
\newcommand*{\hm}[1]{#1\nobreak\discretionary{}
{\hbox{$\mathsurround=0pt #1$}}{}}

\author{Иван Крамойкин}
\title{Калибровка электромагнитного калориметра эксперимента NA64}

%%% Запрет деления слов (перенос слова на новую строку)
\tolerance=1
\emergencystretch=\maxdimen
\hyphenpenalty=10000
\hbadness=10000



\begin{document}

%%% Меняем "Рис._" на "Рисунок_"
\def\figurename{Рисунок}

\maketitle

\newpage

\newpage
\section*{Введение}  \label{sec:intro}
\addcontentsline{toc}{section}{Введение}

В 30-х годах XX века наблюдения за движением материи во Вселенной стали указывать на наличие во Вселенной значительной массы, недоступной для прямой регистрации и проявляющей себя только в гравитационном взаимодействии [1]. Эта скрытая масса была названа тёмной материей. Однако, за прошедшее время темную материю не удалось зарегестрировать непосредственно, и до сих пор она остается одной из важнейших загадок современных ксомологии и астрофизики.

Есть основания полагать [2], что поиски темной материи можно проводить не только методами астрофизики, но и на ускорителях частиц..  Предполагается, что при столкновении высокоэнергетичных частиц между собой или с веществом мишени могут образовываться частицы тёмной материи. Такие эксперименты обычно подразумевают использование калориметров для регистрации полной энергии в различных реакциях (герметичная постановка эксперимента со сбросом пучка -- \textit{beam dump}) [3].

Изначально калориметры производились как довольно грубые, но дешевые инструменты для специализированного применения. Например, для детектирования нейтринных взаимодействий. В современных коллайдерных экспериментах они являются одними из ключевых элементов в телескопах детекторов. Они подходят для множества задач, начиная от отбора событий и заканчивая точными измерениями четырехвекторов отдельных частиц и сгустков частиц и получением информации об энерговыделении в результате различных событий.
	
Вклад калориметрической информации в анализ данных заключается, в основном, в идентификации частиц и измерении энергии частиц, порождающих электромагнитные ливни. Ожидается, что значение адронной калориметрии будет расти с дальнейшим ростом энергий сталкивающихся частиц.
            
\textit{NA64} – это эксперимент с фиксированной мишенью, на протонном суперсинхротроне (\textit{SPS}), расположенном в Европейском центре ядерных исследований (\textit{CERN}), и направленный на поиск экзотических легких бозонов. Ключевым детектором в постановке эксперимента является гетерогенный электромагнитный калориметр \textit{ECAL}. Для получения достоверных результатов необходима тщательная калибровка калориметра. 

Одним из наиболее результативных методов калибровки является калибровка, основанная на компьютерном моделировании. Целью настоящей работы является построение модели электромагнитного ливня для реконструкции энерговыделения в калориметре \textit{ECAL}.

Задачи:

 \begin{itemize}[leftmargin=1.6\parindent, wide]
 	\item[---] определение структуры работ в рамках научного исследования;
 		\item[---] обзор литературы с целью определения наиболее подходящей параметризации электромагнитного ливня для известных конструкции калориметра и диапазона энергий;
 			\item[---] получение профиля электромагнитного ливня на основе МК-модели, достижение согласия результатов моделирования с предсказаниями параметризации;
 				\item[---] определение энергетического разрешения электромагнитного калориметра.
 \end{itemize}


\newpage
\section{Физические процессы в калориметрах}  \label{chap1}
 
 Существует несколько процессов, играющих роль в развитии электромагнитного ливня. Электроны и позитроны теряют энергию на ионизацию и излучение. Первый процесс преобладает при низких энергиях, второй – при высоких. Критическая энергия, при которой роль обоих процессов одинакова в первом приближении обратно пропорциональна зарядовому числу вещества поглотителя [4]:
 
%%% Kritical energy formula (from Wigmans)
\begin{equation}\label{eq:kriten}
\epsilon_c = \frac{610\ \mathrm{MeV}}{Z+1,24}
\end{equation}
 
Фотоны взаимодействуют с веществом преимущественно посредством фотоэффекта, Комптоновского рассеяния и эффекта образования электро-позитронных пар. Фотоэффект преобладает при низких энергиях, а образование пар при высоких. Соответствующие сечения зависят от {\textit Z}. Например, сечение фотоэффекта пропорционально $Z^{5}$ и $E^{-3}$, в то время как сечение образования пар постепенно возрастает с ростом {\textit Z} и {\textit E}, асимптотически приближаясь к определенному значению при значениях энергии порядка \mbox{1 ГэВ}. Угловое распределение более или менее изотропно для фото- и комптоновских электронов, но имеет строго преобладающее направление для электронов и позитронов, рожденных в результате образования пар.

Начиная с энергий от \mbox{1 ГэВ} и выше, электроны и фотоны образуют электромагнитный ливень в веществе, в которое они проникают. Электроны теряют свою энергию преимущественно на излучение, фотоны с наибольшей энергией, рожденные в этом процессе, конвертируются в электрон-позитронные пары, которые также излучают фотоны и т. д.  Число частиц, образующихся в развивающемся ливне достигает максимума на определенной глубине внутри поглотителя, после чего постепенно убывает. Глубина, на которой ливень достигает своего максимума, логарифмически возрастает с увеличением энергии инициирующей ливень частицы.

%%% Deposited energy graphic (from Wigmans)
\begin{figure}[H]
    \centering
    \includegraphics[width=0.75\textwidth]{1.jpg}
    \caption{Энерговыделение как функция глубины для электромагнитных ливней, образованных 1, 10, \mbox{100 ГэВ} и \mbox{1 ТэВ} электроном в медном поглотителе (а). Радиальное распределение энергии, оставленной \mbox{10 ГэВ} электроном в меди на различных глубинах (б). Результат симуляции в программе \textit{EGS4} [4]}
    \label{fig:enDep}
\end{figure}

Поперечное развитие ливня обусловлено множественным рассеянием электронов и позитронов, а так же изотропным и несоосным с ливнем характером рождения фотонов и позитронов.

Первый процесс преобладает на ранних стадиях развития ливня, а второй после того, как ливень преодолевает максимум. Влияние обоих процессов хорошо демонстрируется рис.  \ref{fig:enDep}(б), на котором изображена радиальная плотность энергии ливня, порожденного электроном в меди, на трех различных глубинах внутри калориметра.

Развитие ливня может быть описано более или менее независимо от материала поглотителя в терминах радиационной длины $X_0$ (в продольном направлении) и Мольеровского радиуса $\mathrm{\rho}_M$ (в поперечном направлении). Радиационная длина различается для фотонов и электронов достаточно существенно. Ливни, инициированные высокоэнергетичным  электроном или же фотоном, имеют существенные отличия. Попадая  в материал, высокоэнергетичные электроны начинают излучать немедленно. На их пути, через первые несколько миллиметров материала, они могут испускать тысячи фотонов тормозного излучения. С другой стороны, высокоэнергетичный фотон может как провзаимодействовать с веществом на такой же длине, так и не провзаимодействовать.  В последнем случае он не потеряет энергии совсем, а в случае взаимодействия, может потерять даже больше, чем электрон в таком же количестве материала. Такая разница иллюстрируется на рис. \ref{fig:gamAndEnFraq}. В одинаковом количестве материала электроны теряют большую долю своей энергии, чем фотоны, но разброс энергетических потерь у фотонов больше.

%%% gamma and electron en fraction (from Wigmans)
\begin{figure}[H]
    \centering
    \includegraphics[width=0.75\textwidth]{2.jpg}
    \caption{Распределение доли энергии, оставленной при прохождении пяти радиационных длин электроном и фотоном с энергиями \mbox{10 ГэВ} [4]}
    \label{fig:gamAndEnFraq}
\end{figure}

В первом приближении профиль электромагнитного ливня в основном определяется значениями $X_0$ и $\mathrm{\rho}_M$,  но  такое  приближение неидеально,  что демонстрируется на рис. \ref{fig:elEnDepMat},  где  показано  энерговыделение  на  единице радиационной длины. Такое отличие проще понять посредством того факта, что количество  вторичных  частиц  после  максимума  ливня  начинает уменьшаться,  и  это  снижение  медленнее  в  материалах с высоким {\textit Z}. Например,  высокоэнергетичный  электрон  порождает  в  свинце  в  три  раза больше  позитронов,  чем  в  алюминии.  В  результате  нужно  больше радиационных  длин  свинца,  чем  алюминия,  чтобы  уместить  в  себе  \mbox{99  \%} ливня. К тому же, максимум ливня располагается в материале с высокими {\textit Z} на большей глубине.

%%% energy deposited by electron in different materials (from Wigmans)
\begin{figure}[H]
    \centering
    \includegraphics[width=0.65\textwidth]{3.jpg}
    \caption{Продольные профили электромагнитных ливней, образованных электроном с энергией \mbox{10 ГэВ} в алюминии, железе и \mbox{свинце [4]}}
    \label{fig:elEnDepMat}
\end{figure}

%%% average shower fraction for different energies and materials (from Wigmans)
\begin{figure}[H]
    \centering
    \includegraphics[width=0.65\textwidth]{4.jpg}
    \caption{Средняя доля энергии, выделившейся в блоке материала в результате прохождения его частицей. Показаны результаты для ливней, образованных электронами различных энергий в медном поглотителе (а) и результаты для ливней, образованных электроном с энергией \mbox{100 ГэВ} в различных поглотителях (б). Результат симуляции в программе \textit{EGS4} [4]}
    \label{fig:fracEnAndMat}
\end{figure}

Зависимость  толщины  калориметра,  необходимой  для  того,  чтобы вместить   электромагнитный   ливень,   образованный   электроном,   для различных материалов представлена на рис. \ref{fig:fracEnAndMat}(б). При поглощении \mbox{99 \%} электромагнитного  ливня разница  между  материалом  с  высоким {\textit Z} и материалом  с  низким {\textit Z} может  достигать  десяти $X_0$. И  по  причинам, описанным  выше,  нужно  еще  больше  материала,  чтобы  поглотить  ливень, образованный фотоном. Зависимость доли поглощенной энергии от толщины калориметра изображена на рис. \ref{fig:fracEnAndMat}(а). Для поперечной локализации ливня энергетическая  зависимость  отсутствует,  а различия  в  материале  не  такие значительные,  как  для  продольного. Достаточно  длинный  цилиндр таким образом  поглотит  одинаковую  долю  энергии  от \mbox{1  ГэВ}  ливня  и  от \mbox{1  ТэВ} ливня.

Нарушения пропорциональности, рассмотренные на рис. \ref{fig:elEnDepMat} и \ref{fig:fracEnAndMat}, вызваны явлениями, которые происходят при энергиях ниже критической. Например, в  свинце  более  \mbox{40 \%}  энергии  ливня оставляется  в  веществе частицами  с энергиями ниже \mbox{1 МэВ}, в то время как критическая энергия имеет значение примерно \mbox{7 МэВ}. Только четверть энергии оказывается внесена позитронами, остальное  приносят  электроны.  Эти  факты получены  в  результате  Монте-Карло  симуляции  развития  ливня, и показывают,  что  комптоновское рассеяние  и  явление  образования  пар в  основном  ответственны  за формирование профиля ливня. Оба процесса преобладают на энергиях ниже критической,  а  значит,  недостаточно  точно  описываются  терминами радиационной длины $X_0$ и Мольеровского радиуса $\mathrm{\rho}_M$.

\newpage
\section{Функция отклика калориметра}  \label{chap2} 
\subsection{Абсолютный отклик и его отклонения} \label{chap2.1}

Определим  отклик  калориметра  как  средний  сигнал  калориметра  на единицу выделенной энергии. Таким образом, отклик может быть выражен в терминах числа фотоэлектронов на один ГэВ, в пикокулонах на МэВ и т.д. 

%%% average signal for wired chambers (from Wigmans)
\begin{figure}[!h]
    \centering
    \includegraphics[width=0.75\textwidth]{5.jpg}
    \caption{Средний сигнал гетерогенного калориметра с проволочными камерами, работающими в режиме «насыщенной лавины», на электромагнитный ливень как функция выделившейся энергии (а),сигнал для каждой из пяти секций (б) [4]}
    \label{fig:meanSign}
\end{figure}

В общем случае электромагнитные калориметры линейны только тогда, когда вся энергия пучка выделяется в результате процессов, которые могут производить  сигналы  (возбуждение  или  ионизация  в  слое  поглотителя). Нелинейность обычно свидетельствует об инструментальных проблемах, таких как  насыщение  сигнала и  ливневые  утечки.  На  рис. \ref{fig:meanSign} показан  отклик нелинейного  калориметра.  В  этом  детекторе  проволочные  камеры, использующиеся  для детектирования прохождении  частицы  из  ливня, работают  в  режиме  «насыщенной  лавины»,  это  значит,  что  камеры  не различают прохождение одной частицы и \textit{n} частиц. С повышением энергии ливня или плотности  частиц в ливне  эффект насыщения понижает отклик. Рис. \ref{fig:meanSign}(б), на котором представлена зависимость сигнала сегментированного на пять  секций  калориметра  от  числа  образовавшихся  в  каждой  секции позитронов, показывает,  что  не  столько  выделившаяся  энергия,  сколько плотность  образовавшихся  частиц  ответственна  за  подобные  эффекты,  так как  влияние  эффектов  наиболее  заметно  на  ранней  стадии  развития  ливня (секция  1),  при  которой ливень  еще  сильно  сколлимирован. Описанные эффекты можно избежать, используя камеры в пропорциональном режиме.

По своему устройству калориметры подразделяются на два класса:

\begin{enumerate}[wide]
\item Гомогенные калориметры, в которых поглотитель является также активным материалом, производящим сигнал.
\item Гетерогенные калориметры,  в  которых  каждый  материал выполняет свою функцию.
\end{enumerate}

В инструментах, относящихся ко второму классу, только доля энергии ливня выделяется в активном материале. Эта «сэмплирующая» доля обычно определяется на основании сигнала от минимально ионизирующей частицы (\textit{mip}). \textit{Mip} соответствует частице, удельные ионизирующие потери которой в веществах  поглотителя  и  активного  материала  будут  наименьшими. Например,  в  калориметре \textit{D0},  который  состоит  из  пластин  из $\mathrm{U^{238}}$, разделенных  промежутками  в  \mbox{4,6  мм},  заполненными жидким  аргоном, сэмплирующая   доля для \textit{mip} составляет \mbox{13,7   \%}. Однако для электромагнитных   ливней   сэмплирующая   доля   энергии   составляет приблизительно \mbox{8,2 \%}.

%%% e divided by mip as a function of plates thickness
\begin{figure}[!h]
    \centering
    \includegraphics[width=0.75\textwidth]{6.jpg}
    \caption{Отношение \textit{e/mip} как функция толщины слоёв поглотителей для калориметров уран/оргстекло и уран/жидкий аргон. Результат симуляции в программе \textit{EGS4} [4]}
    \label{fig:eDevMip}
\end{figure}

Причиной такого различия снова является тот факт, что основной вклад в сигнал от электромагнитного ливня вносят очень мягкие частицы. Гамма-кванты с энергиями ниже \mbox{1 МэВ} чрезвычайно неэффективно регистрируются в  детекторах этого класса, потому  что  наиболее  вероятное  взаимодействие при таких энергиях это фотоэффект. Сечение фотоэффекта пропорционально $Z^5$,  поэтому  практически  все  взаимодействия  мягких  гамма-квантов происходят  в  слоях  поглотителя,  и  вклад  в  сигнал  можно  ожидать  только если  взаимодействие  произошло  очень  близко  к  границе  поглотителя  с активным  слоем  (тогда  фотоэлектрон,  чей  пробег  меньше  \mbox{1  мм}, может, выбравшись  из  поглотителя, произвести  сигнал  в  жидком  аргоне). По причине  ключевой  роли  фотоэффекта,  его  влияние на  отношение \textit{e/mip} зависит от значений \textit{Z} пассивного и активного материалов (\textit{e/mip} наименьшее для  поглотителей  с  высоким \textit{Z} и  активных  материалов  с  малыми \textit{Z})  и  от толщины слоёв поглотителя (рис. \ref{fig:eDevMip}). Если поглотитель сделать достаточно тонким, то \textit{e/mip} становится практически равным 1.

\subsection{Флуктуации} \label{chap2.2}

Так  как  принцип  работы  калориметров  основан  на статистических процессах, точность калориметрических измерений определена и ограничена флуктуациями.   На   энергетическое   разрешение   электромагнитного калориметра влияет несколько флуктуационных процессов:

 \begin{itemize}[leftmargin=1.6\parindent, wide]
 	\item[---] квантовые   флуктуации   сигнала,   например   фотоэлектронная статистика;
 		\item[---] флуктуации ливневых утечек;
 			\item[---] флуктуации, обусловленные инструментальными эффектами, такими как  электронные  шумы,  ослабление  светового  потока  и  структурные неоднородности;
 				\item[---] флуктуации сэмплирования.
 \end{itemize}
 
 Последний процесс характерен только для гетерогенных калориметров. В хорошо спроектированных калориметрах этот вид флуктуаций преобладает над  остальными.  В  отличие  от  других  флуктуационных  процессов, флуктуации  сэмплирования  обусловлены  правилами  статистики  Пуассона. Таким  образом,  их  вклад  в  энергетическое  и  позиционное  разрешение описывается слагаемым, который пропорционален $1/\sqrt{E}:\sigma /E \sim E^{- 1/2}$.
 
 Флуктуации  сэмплирования  обусловлены  сэмплирующей  долей (или же   относительным   количеством   активного   материала)   и   частотой сэмплирования  (толщиной  слоёв).  В  электромагнитных  калориметрах  с  не газовым  активным  слоем  они  хорошо  описываются  следующей  формулой[4]:
 
 %%% Samplong fluctuations (from Wigmans)
\begin{equation}\label{eq:sampFluc}
\frac{\sigma}{E}=2,7\ \%\sqrt{ d/f_{sampl} }E^{-1/2},
\end{equation}
в которой \textit{d} – это толщина активного слоя (в мм), а $f_{sampl}$ – это сэмплирующая доля для \textit{mip}. Например, рассчитанная по формуле (\ref{eq:sampFluc}) составляющая энергетического разрешения для свинцово-сцинтилляционного калориметра \textit{KLOE} [5], соответствующая флуктуациям  сэмплирования, равна $6,9\ \%/\sqrt{E}$, что хорошо соответствует  экспериментально  найденному  значению  разрешения $5,7\ \%/\sqrt{E}$.

Среди  калориметров,  разрешение  которых  определяется  в  основном квантовыми  флуктуациями  сигналов,  можно  упомянуть  германиевые детекторы,  используемые  для  ядерной $\gamma$-спектрометрии,  и  калориметры  с кварцевым  волокном,  такие  как \textit{CMS} или \textit{HF}. Количество  энергии, необходимой для образования сигнала, в этих двух примерах различается на девять  порядков  по  величине.  В  то  время  как  достаточно  \mbox{1  эВ}  для образования электронно-дырочной пары в германиевом детекторе, светосбор в  кварцевом  калориметре  обычно  около  \mbox{1  ГэВ}  на  фотоэлектрон.  Таким образом   квантовые   флуктуации   сигнала   ограничивают   разрешение германиевых калориметров до величины 0,1 \% на МэВ, а кварцевых до \mbox{10 \% на ГэВ}.

Влияние    флуктуаций    ливневых    утечек    на    разрешение электромагнитного калориметра демонстрируется на рис. \ref{fig:leak}. Эти флуктуации имеют  не  Пуассоновский  характер  и  поэтому,  их  вклад  в  разрешение  не пропорционален $E^{-1/2}$. Оказывается  также,  что  для  заданного  положения ливня,  влияние  продольных  утечек  значительнее,  чем  поперечных.  Эти отличия объясняются различиями в числах различных частиц, ответственных за  утечки.  Например,  флуктуации  в  точке  начала  порожденного  фотоном ливня,  являются  флуктуациями  утечек,  за  которые  ответственна  только единственная  частица – инициирующий  фотон.  Боковые  утечки – это коллективный феномен, в который вносят вклад множество частиц.

В  отличие  от  продольных  и  поперечных  утечек,  третий  тип  утечек, альбедо,  или  же  обратные  утечки  через  тот  торец  детектора,  в  который попадает  инициирующая  частица,  не  может  быть  уменьшен  посредством изменения конструкции детектора. К счастью, эти утечки крайне невелики во всем  диапазоне  энергий,  кроме  очень  малых  значений. Результаты, показанные на рис. \ref{fig:leak}, получены путем Монте-Карло моделирования, но они подтверждаются многими экспериментами.

%%% lateral, lodgitudal and albedo fluctuations
\begin{figure}[H]
    \centering
    \includegraphics[width=0.75\textwidth]{7.jpg}
    \caption{Сравнение эффектов, вызванных ливневыми утечками различного рода. Результат симуляции в программе \textit{EGS4} [4]}
    \label{fig:leak}
\end{figure}

На практике разрешение конкретного калориметра определяется различными видами флуктуаций, каждый из которых имеет характерную энергетическую зависимость.  Обычно  эти  эффекты  некоррелированны,  значит  могут учитываться  в  виде  отдельных  слагаемых.  По  причине  различных зависимостей  от  энергии  общее  разрешение  калориметра  при  различных энергиях   может   быть   обусловлено   различными   флуктуационными эффектами.

%%% Fluctuations impact in the energy resolution
\begin{figure}[H]
    \centering
    \includegraphics[width=0.5\textwidth]{8.jpg}
    \caption{Энергетическое разрешение электромагнитного калориметра эксперимента \textit{ATLAS} [6]}
    \label{fig:resOnEn}
\end{figure}

Это  иллюстрируется  рис. \ref{fig:resOnEn},  на  котором  изображена  зависимость различных слагаемых, входящих в выражение энергетического разрешения, от  энергии для  электромагнитного  калориметра  эксперимента \textit{ATLAS}. Для энергий ниже \mbox{10 ГэВ} преобладающим фактором является электронный шум, между 10 и \mbox{100 ГэВ} – флуктуации сэмплирования и другие стохастические процессы,  при  энергиях  выше  \mbox{100  ГэВ}  независящие  от  энергии  эффекты влияют на энергетическое разрешение.

\subsection{Форма функции отклика} \label{chap2.3}

Не  все  виды  флуктуаций,  увеличивающие  отклонения  отклика, являются   симметричными   относительно   среднего   значения.   Ниже перечислены примеры эффектов, приводящих к несимметричным функциям отклика:

 \begin{itemize}[leftmargin=1.6\parindent, wide]
 	\item[---] Если сигнал сформирован очень маленьким числом частиц (например фотоэлектронами),  то  распределение  Пуассона  становится  ассиметричным. Эффекты  такого  рода  наблюдаются  в  сигналах  калориметров  с кварцевым оптическим волокном (рис. \ref{fig:poissons}).
 		\item[---] Эффекты ливневых утечек приводят к «хвостам» в распределениях сигналов.  Обычно,  такие  хвосты  возникают  в  распределении  сигналов  с низкоэнергетической стороны, потому что энергия покидает активный объем детектора.  Однако  существуют  примеры  детекторов,  в  которых  утечки приводят   к   усилению   сигнала.   Такое   явление   наблюдается   в сцинтилляционных  калориметрах,  в  которых  сцинтилляции  улавливаются кремниевыми диодами. Попадание в такой диод выбравшегося из детектора электрона может привести к тому, что величина сигнала окажется большей, чем от сцинтилляционного фотона (рис. \ref{fig:photPMT},а).
 \end{itemize}
 
 %%% Simmetric and asimmetric signals
\begin{figure}[H]
    \centering
    \includegraphics[width=1.0\textwidth]{9.jpg}
    \caption{Распределение сигналов для электромагнитных ливней, образованных электронами с энергией \mbox{10 ГэВ (а)} и \mbox{200 ГэВ (б)}, в калориметре эксперимента \textit{CMS} с кварцевыми оптическими волокнами [7]}
    \label{fig:poissons}
\end{figure}

 %%% Photodiode and PMT signals
\begin{figure}[H]
    \centering
    \includegraphics[width=1.0\textwidth]{10.jpg}
    \caption{Распределения сигналов от высокоэнергетичного электрона, образующего электромагнитный ливень в $\mathrm{PbWO}_{\mathrm{4}}$ кристальном калориметре. Для снятия сигнала используется кремниевый фотодиод (а) и ФЭУ (б) [8]}
    \label{fig:photPMT}
\end{figure}



 
\end{document}  
