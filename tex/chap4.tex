\newpage
\section{Параметризация электромагнитного ливня}  \label{chap4}   

Параметризация электромагнитного  ливня заключается  в  получении аналитической зависимости, связывающей пространственное энергетическое распределение   электромагнитного   ливня   с   координатами   частицы, инициирующей ливень при попадании в детектор, и энергией этой частицы. Необходимость   параметризации обычно обусловлена   желанием ускорить процесс симуляции без снижения точности однако, параметризация ливня  может  также  использоваться  для  уточнения  калибровочных коэффициентов. Параметризация в гомогенных калориметрах может использоваться как первое приближение параметризации ливня в моделях для гетерогенных калориметров. Пространственное  энергетическое  распределение  описывается  тремя функциями плотности вероятности:

%%% Spatial energy distrinution (Grindhammer)
\begin{equation}\label{eq:dE}
dE(\vec{r}) = Ef(t)dtf(r)drf(\varphi)d\varphi, 
\end{equation}
соответствующими   продольному,   радиальному   и   азимутальному распределениям  энергии. В  этой  формуле \textit{t} обозначает  продольную  длину ливня  в  единицах  радиационной  длины, \textit{r} – радиальное  расстояние  в Мольеровских  радиусах  от  оси  распространения  ливня, $\varphi$ -- азимутальный угол. Начало ливня определяется точкой в пространстве, в которой первый из электронов или позитронов испускает квант тормозного излучения. Известно, что средний продольный профиль электромагнитного ливня может описываться гамма-распределением [14]:

%%% Longitudal profile (Grindhammer)
\begin{equation}\label{eq:longProf}
\left< 
\frac{1}{E}\frac{dE(t)}{dt}
\right>
= f(t) 
= \frac{ (\beta t )^{ \alpha - 1 } \beta exp(-\beta t) }{\Gamma (\alpha) } 
\end{equation}
Центр тяжести $<t>$,  и  глубина  максимума  ливня \textit{T} могут  быть вычислены  через  параметр  формы $\mathrm{\alpha}$ и  параметр  масштаба $\mathrm{\beta}$ следующим образом

%%% Gravity center (Grindhammer)
\begin{equation}\label{eq:gravCenter}
<t> = \frac{\alpha}{\beta},
\end{equation}

%%% Scale parameter (Grindhammer)
\begin{equation}\label{eq:scalePar}
T = \frac{\alpha - 1}{\beta}.
\end{equation}
Множество различных функций, описывающих среднюю радиальную компоненту ливня

%%% Radial profile (Grindhammer)
\begin{equation}\label{eq:radProf}
f(r) = \frac{1}{dE(t)}\frac{dE(t,r)}{dr},
\end{equation}
могут  быть  найдены  в  различной специализированной литературе. В  этой работе   предлагается   использовать   следующую   двухкомпонентную подстановку, которая является расширением формулы, предложенной в [15]:

%%% Radial profile 2-component Ansatz (Grindhammer)
\begin{equation}\label{eq:ansatz}
f(r)
= \rho f_c(r)
+ (1 - \rho)f_T(r)
= \rho \frac{2rR_c^2}{ (r^2 + R_c^2)^2 }
+ (1 - \rho) \frac{ 2rR_T^2}{ (r^2 + R_T^2)^2 }
\end{equation}
где $0 \leq \rho \leq 1, R_c(R_T)$ -- медиана серединной (хвостовой) компоненты, $\rho$ -- вероятность, задающая серединный вес компоненты.

Азимутальное энергетическое распределение равномерно: $f(\varphi) = 1/{2\pi}$.