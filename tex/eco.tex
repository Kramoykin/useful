 \newpage
\section{Финансовый менеджмент, ресурсоэффективность и ресурсосбережение}  \label{ecochap:1}

\subsection{Оценка  коммерческого  потенциала  и  перспективности проведения научных исследований с позиции ресурсоэффективности и ресурсосбережения} \label{eco.1}

Калибровка детектора является одной из наиболее важных операций, предшествующих физическому эксперименту. Основной целью калибровки является  установление  соответствия  между  энергией,  выделившейся  в материале детектора при прохождении через него частицы или пучка частиц, и  сигналом  детектора.  Неточности  калибровки  могут  привести  к значительным ошибкам при измерении поглощенной энергии, что снижает достоверность результата эксперимента.

\subsubsection{Потенциальные потребители результатов исследования} \label{eco.1.1}

Единственными   потребителями   этой   научно-исследовательской работы   могут   быть   коллаборации   экспериментов,   использующих гетерогенные  электромагнитные  калориметры  схожего  устройства  с калориметром \textit{ECAL} эксперимента \textit{NA64}. Адаптировать  работу  под промышленность не представляется возможным.

\subsubsection{Анализ конкурентных технических решений} \label{eco.1.2}

Существует   несколько   методик   калибровки   гетерогенных калориметров,   применяющихся   в   различных   экспериментах.   Для предварительной  оценки  эффективности  научной  работы  был  проведен детальный  анализ  конкурентных  методик  калибровки. Такой  анализ позволяет оценить сильные и слабые стороны конкурирующих методик, и, если это необходимо, внести своевременные коррективы в рассматриваемую методику для поддержания ее конкурентоспособности. 

Анализ   конкурентных   технических   решений   проводился   с использованием оценочной карты, приведенной в таблице \ref{tab:eco1}. В этой таблице сравниваются    критерии    технической    ресурсоэффективности    и экономической  эффективности  различных  методик  калибровки.  Численное значение каждого критерия выбирается экспертным путем по пятибалльной шкале, где 1 –наиболее низкое значение, а 5 –наиболее высокое. Значения весов  критериев  определяются  экспертным  путем  так,  чтобы  их  сумма равнялась 1. Значения критериев трудоемкости и технической сложности тем выше, чем проще конкретный метод в реализации.

%%% Критерии трудоемкости
\begin{enumerate}[wide]
\item $\textnormal{Б}_\textnormal{ф}$ -- калибровка с использованием параметризации электромагнитного ливня.
\item $\textnormal{Б}_{\textnormal{к}1}$ -- калибровка методом минимизации разности между энергией пучка и суммой реконструированного энерговыделения в каждом сегменте.
\item $\textnormal{Б}_{\textnormal{к}2}$ -- индивидуальная калибровка каждого сегмента. 
\end{enumerate}

% %%
\begin{table}[H]
\small
\centering
\caption{Оценочная карта для сравнения конкурентных технческих разработок}
\label{tab:eco1}
\begin{tabular}{|l|c|c|c|c|c|c|c|}
\hline
\multicolumn{1}{|c|}{\multirow{2}{*}{\textbf{Критерии оценки}}}                  & \multicolumn{1}{|c|}{\multirow{2}{*}{\textbf{Вес критерия}}} & \multicolumn{3}{|c|}{\textbf{Баллы}}                                           & \multicolumn{3}{|c|}{\textbf{Конкурентоспособность}}                           \\ \cline{3-8} 
\multicolumn{1}{|c|}{}                                                           & \multicolumn{1}{|c|}{}                                       & \multicolumn{1}{|c|}{$\textnormal{Б}_\textnormal{ф}$} & \multicolumn{1}{|c|}{$\textnormal{Б}_{\textnormal{к}1}$} & \multicolumn{1}{|c|}{$\textnormal{Б}_{\textnormal{к}2}$} & \multicolumn{1}{|c|}{$\textnormal{К}_\textnormal{ф}$} & \multicolumn{1}{|c|}{$\textnormal{К}_{\textnormal{к}1}$} & \multicolumn{1}{|c|}{$\textnormal{К}_{\textnormal{к}2}$} \\ \hline
\multicolumn{8}{|c|}{\textbf{Технические критерии методики калибровки}}                                                                                                                                                                                                                                        \\ \hline
1. Точность калибровки                                                           & 0,3                                                         & 4                       & 2                        & 3                        & 1,2                     & 0,6                      & 0,9                      \\ \hline
2. Трудоемкость                                                                  & 0,1                                                         & 4                       & 2                        & 1                        & 0,4                     & 0,2                      & 0,1                      \\ \hline
3. Универсальность                                                               & 0,2                                                         & 5                       & 1                        & 2                        & 1                       & 0,2                      & 0,4                      \\ \hline
\begin{tabular}[c]{@{}l@{}}4. Техническая \\ сложность\end{tabular}              & 0,15                                                        & 5                       & 2                        & 1                        & 0,75                    & 0,3                      & 0,15                     \\ \hline
\multicolumn{8}{|c|}{\textbf{Экономические критерии оценки эффективности}}                                                                                                                                                                                                                                     \\ \hline
\begin{tabular}[c]{@{}l@{}}5. Финансирование научной \\ разработки\end{tabular}  & 0,1                                                         & 1                       & 5                        & 5                        & 0,1                     & 0,5                      & 0,5                      \\ \hline
\begin{tabular}[c]{@{}l@{}}6. Стоимость осуществления \\ калибровки\end{tabular} & 0,15                                                        & 1                       & 5                        & 5                        & 0,15                    & 0,75                     & 0,75                     \\ \hline
\textbf{Итого:}                                                                  & \textbf{1}                                                  & \textbf{}               & \textbf{}                & \textbf{}                & \textbf{3,60}           & \textbf{2,55}            & \textbf{2,80}            \\ \hline
\end{tabular}%
\end{table}

По результатам проведенного анализа можно сделать заключение, что метод калибровки с использованием Монте-Карло модели и параметризации ливня  превосходит  конкурирующие  методы.  Причиной  этому  служат следующие особенности:

%%% Отличительные особенности методики
\begin{enumerate}[wide]
\item Тщательное моделирование приводит к точным результатам, не зависящим от инструментальных особенностей;
\item использование  модели  позволяет  путем  небольших  изменений получить  значения  калибровочных  коэффициентов  для  различных  частиц, инициирующих ливень;
\item использование   модели   осуществляется   без   проведения экспериментов,  что  приводит  к  высокой  экономической  эффективности методики и малым трудозатратам.
\end{enumerate}

\subsubsection{\textit{SWOT}-анализ} \label{eco.1.3}

\textit{SWOT} – \textit{Strengths} (сильные стороны), \textit{Weaknesses} (слабые стороны), \textit{Opportunities} (возможности)  и \textit{Threats}  (угрозы) -- комплексный  анализ научно-исследовательского проекта, проводящийся в несколько этапов. Результаты   первого   этапа \textit{SWOT}-анализа,   заключающегося   в выявлении сильных и слабых сторон проекта, возможностей его развития и угроз,представлены в таблице \ref{tab:eco2}.

%%% Первый этап свотанализа
\begin{table}[H]
\centering
\caption{Первый этап \textit{SWOT}-анализа}
\label{tab:eco2}
\begin{tabular}{|l|l|}
\hline
\textbf{Сильные стороны}                                                                                                                                                                                         & \textbf{Слабые стороны}                                                                                                                                                                              \\ \hline
\begin{tabular}[c]{@{}l@{}}1. Отсутствие необходимости \\ проведения эксперимента.\\ 2. Высокая точность.\end{tabular}                                                                                           & \begin{tabular}[c]{@{}l@{}}1. Остутствие возможности \\ учета некоторых технических\\ особенностей.\\ 2. Неустранимая погрешность \\ моделирования.\end{tabular}                                      \\ \hline
\textbf{Возможности}                                                                                                                                                                                             & \textbf{Угрозы}                                                                                                                                                                                       \\ \hline
\begin{tabular}[c]{@{}l@{}}1. Калибровка электромагнитных\\ калориметров с различными\\  геометриями.\\ 2. Возможность калибровки для \\ различных частиц, инициирующих \\ электромагнитный ливень.\end{tabular} & \begin{tabular}[c]{@{}l@{}}1. Получение калибровок другими\\ методами.\\ 2. Невостребованность \\ исследрваний в данном направлении\\ из-за появления новых\\ экспериментальных фактов.\end{tabular}  \\ \hline
\end{tabular}%
\end{table}

Второй  этап  заключается  в  построении  интерактивных  матриц возможностей  и  угроз,  позволяющих  оценить  эффективность  проекта,  а также  надежность  его  реализации,  на  основании  матрицы SWOT. Соотношения параметров представлены в таблицах \ref{tab:eco3},\ref{tab:eco4},\ref{tab:eco5} и \ref{tab:eco6}.

%%% Сильные стороны и возможности
\begin{table}[H]
\centering
\caption{Интерактивная оценка проекта "Сильные стороны и возможности"}
\label{tab:eco3}
\begin{tabular}{|c|c|c|c|}
\hline
\multicolumn{4}{|c|}{Сильные стороны}                                                          \\ \hline
\multirow{3}{*}{\begin{tabular}[c]{@{}c@{}}Возможности \\ проекта\end{tabular}} &    & С1 & С2 \\ \cline{2-4} 
                                                                                & В1 & +  & -  \\ \cline{2-4} 
                                                                                & В2 & +  & +  \\ \hline
\end{tabular}%
\end{table}

%%% Слабые стороны и возможности
\begin{table}[H]
\centering
\caption{Интерактивная оценка проекта "Слабые стороны и возможности"}
\label{tab:eco4}
\begin{tabular}{|c|c|c|c|}
\hline
\multicolumn{4}{|c|}{Слабые стороны}                                                             \\ \hline
\multirow{3}{*}{\begin{tabular}[c]{@{}c@{}}Возможности \\ проекта\end{tabular}} &    & Сл1 & Сл2 \\ \cline{2-4} 
                                                                                & В1 & +   & -   \\ \cline{2-4} 
                                                                                & В2 & +   & -   \\ \hline
\end{tabular}%
\end{table}

%%% Сильные стороны и угрозы
\begin{table}[H]
\centering
\caption{Интерактивная оценка проекта "Сильные стороны и угрозы"}
\label{tab:eco5}
\begin{tabular}{|c|c|c|c|}
\hline
\multicolumn{4}{|c|}{Сильные стороны}   \\ \hline
\multirow{3}{*}{Угрозы} &    & С1 & С2 \\ \cline{2-4} 
                        & У1 & +  & -  \\ \cline{2-4} 
                        & У2 & +  & -  \\ \hline
\end{tabular}%
\end{table}

%%% Слабые стороны и угрозы
\begin{table}[H]
\centering
\caption{Интерактивная оценка проекта "Слабые стороны и угрозы"}
\label{tab:eco6}
\begin{tabular}{|c|c|c|c|}
\hline
\multicolumn{4}{|c|}{Слабые стороны}   \\ \hline
\multirow{3}{*}{Угрозы} &    & С1 & С2 \\ \cline{2-4} 
                        & У1 & -  & -  \\ \cline{2-4} 
                        & У2 & +  & -  \\ \hline
\end{tabular}%
\end{table}

Таким  образом,  в  рамках  третьего  этапа была составлена итоговая матрица SWOT-анализа (таблица \ref{tab:eco7}).

% Итоговая матрица свотанализа
\begin{table}[H]
\scriptsize
\centering
\caption{Итоговая матрица \textit{SWOT}-анализа}
\label{tab:eco7}
\begin{tabular}{|l|l|l|}
\hline
 &
  \begin{tabular}[t]{@{}l@{}}\textbf{Сильные стороны научно-}\\  \textbf{исследовательского проекта}\\ 1. Отсутствие необходимости\\ проведения эксперимента.\\ 2. Высокая точность\end{tabular} &
  \begin{tabular}[t]{@{}l@{}}\textbf{Слабые стороны научно-}\\ \textbf{исследовательского проекта}\\ 1. Отсутствие \\ возможности учета\\ некоторых технических\\ особенностей.\\ 2. Неустранимая \\ погрешность моделирования.\end{tabular} \\ \hline
\begin{tabular}[t]{@{}l@{}}\textbf{Возможности:}\\ 1. Калибровка электромагнитных\\ калориметров с различными\\ геометриями.\\ 2. Возможность\\ калибровки для различных\\ частиц, инициирующих\\ электромагнитный ливень.\end{tabular} &
  \begin{tabular}[t]{@{}l@{}}Отсутствие необходимости \\ в эксперименте позволяет \\ путем небольших изменений \\ применить методику \\ калибровки в различных \\ экспериментах с разными \\ установками и пучками \\ частиц\end{tabular} &
  \begin{tabular}[t]{@{}l@{}}Увеличение сложности \\ модели позволить \\ расширить число ее \\ возможных применений\end{tabular} \\ \hline
\begin{tabular}[t]{@{}l@{}}\textbf{Угрозы:}\\ 1. Получение \\ калибровок другими методами.\\ 2. Невостребованность\\ исследований в данном \\ направлении из-за появления \\ новых экспериментальных\\ фактов.\end{tabular} &
  \begin{tabular}[t]{@{}l@{}}Отсутствие необходимости в эксперименте позволяет \\ предлагать методику \\ калибровки в эксперименты, \\ детекторы которых ранее \\ были откалиброваны \\ недостаточно точно\end{tabular} &
  \begin{tabular}[t]{@{}l@{}}При серьезных изменениях в \\ технологии проведения \\ эксперимента \\ приспособление  такой \\ модели к новым \\ техническим особенностям \\ может оказаться \\ нецелесообразным или даже \\ невозможным\end{tabular} \\ \hline
\end{tabular}%
\end{table}

В результате \textit{SWOT}-анализа показана перспективность работы в виду ее  универсальности  и  отсутствия  привязки  к  конкретному  эксперименту. Наиболее  значимая  уязвимость  заключается  в  том,  что  в  большинстве экспериментов    используются    методы    калибровки,    основанные непосредственно   на   физических   измерениях,   которые   стали   уже классическими.

\subsection{Планирование научно-исследовательских работ} \label{eco.2}

\subsubsection{Структура работ в рамках научного исследования} \label{eco.2.1}

Планирование  призвано  обеспечить  рациональное  использование времени  и  при  формировании  научно-исследовательской  работы  является, несомненно,  важным  этапом.  Планирование  комплекса  предполагаемых работ осуществляется в следующем порядке:

 \begin{itemize}[leftmargin=1.6\parindent, wide]
 	\item[---] определение структуры работ в рамках научного исследования;
 		\item[---] определение участников каждой работы;
 			\item[---] установление продолжительности работ;
 				\item[---] построение графика проведения научных исследований.
 \end{itemize}
 
Для выполнения научных исследований формируется рабочая группа, в состав  которой  могут  входить  научные  сотрудники  и  преподаватели, инженеры, техники и лаборанты, численность групп может варьироваться от 3  до  15  человек.  В  рамках  данной  работы  была  сформирована  рабочая группа, в состав которой вошли: научный руководитель истудент-бакалавр. В  данном  разделе  был  составлен  перечень  этапов  и  работ  по  выполнению НИР, который представлен в таблице \ref{tab:eco8}.

%%% Этапы и исполнители
\begin{table}[H]
\small
\centering
\caption{Перечень этапов, работ и распределение исполниелей}
\label{tab:eco8}
\begin{tabular}{|c|c|c|c|}
\hline
Основные этапы                                                                            & \begin{tabular}[c]{@{}c@{}}№\\ раб\end{tabular} & Содержание работы                                                                        & \begin{tabular}[c]{@{}c@{}}Должность\\ исполнителя\end{tabular}            \\ \hline
\begin{tabular}[c]{@{}c@{}}Разработка \\ технического задания\end{tabular}                & 1                                               & \begin{tabular}[c]{@{}c@{}}Составление и утверждение\\ технического задания\end{tabular} & \begin{tabular}[c]{@{}c@{}}Научный\\ руководитель,\\ бакалавр\end{tabular} \\ \hline
\multirow{4}{*}{\begin{tabular}[c]{@{}c@{}}Выбор направления\\ исследований\end{tabular}} & 2                                               & Выбор направления исследований                                                           & \begin{tabular}[c]{@{}c@{}}Научный\\ руководитель,\\ бакалавр\end{tabular} \\ \cline{2-4} 
                                                                                          & 3                                               & \begin{tabular}[c]{@{}c@{}}Побор и изучение материалов по\\ теме\end{tabular}            & \begin{tabular}[c]{@{}c@{}}Научный\\ руководитель,\\ бакалавр\end{tabular} \\ \cline{2-4} 
                                                                                          & 4                                               & \begin{tabular}[c]{@{}c@{}}Разработка методики выполнения \\ работ\end{tabular}          & \begin{tabular}[c]{@{}c@{}}Научный \\ руководитель\end{tabular}            \\ \cline{2-4} 
                                                                                          & 5                                               & Составление календарного плана                                                           & \begin{tabular}[c]{@{}c@{}}Научный\\ руководитель,\\ бакалавр\end{tabular} \\ \hline
\multirow{2}{*}{\begin{tabular}[c]{@{}c@{}}Теоретическое \\ исследование\end{tabular}}    & 6                                               & Поиск литературы                                                                         & \begin{tabular}[c]{@{}c@{}}Научный\\ руководитель,\\ бакалавр\end{tabular} \\ \cline{2-4} 
                                                                                          & 7                                               & Изучение литературы                                                                      & Бакалавр                                                                   \\ \hline
\multirow{3}{*}{Практическая часть}                                                       & 8                                               & \begin{tabular}[c]{@{}c@{}}Моделирование электромагнитного\\ ливня\end{tabular}          & Бакалавр                                                                   \\ \cline{2-4} 
                                                                                          & 9                                               & \begin{tabular}[c]{@{}c@{}}Расчет энергетического разрешения\\ калориметра\end{tabular}  & Бакалавр                                                                   \\ \cline{2-4} 
                                                                                          & 10                                              & Калибровка калориметра                                                                   & Бакалавр                                                                   \\ \hline
\begin{tabular}[c]{@{}c@{}}Обобщение и оценка \\ результатов\end{tabular}                 & 11                                              & \begin{tabular}[c]{@{}c@{}}Оценка эффективности полученных\\ результатов\end{tabular}    & \begin{tabular}[c]{@{}c@{}}Научный\\ руководитель,\\ бакалавр\end{tabular} \\ \hline
\multicolumn{4}{|c|}{Провередие ВКР}                                                                                                                                                                                                                                                                                \\ \hline
\begin{tabular}[c]{@{}c@{}}Оформление комплекта\\ документации по ВКР\end{tabular}        & 12                                              & Составление отчета                                                                       & Бакалавр                                                                   \\ \hline
\end{tabular}%
\end{table}

\subsubsection{Определение трудоемкости выполнения работ} \label{eco.2.2}

Трудоемкость  выполнения  научного  исследования  оценивается экспертным  путем  в  человеко-днях  и  носит  вероятностный  характер,  т.к. зависит  от  множества  трудно  учитываемых  факторов.  Для  определения ожидаемого (среднего) значения трудоемкости $t_{\textnormal{ож}i}$ используется следующая формула:

%%% Среднее значение трудоемкости
\begin{equation}\label{eq:eq-eco1}
t_{\textnormal{ож}i} 
= \frac{3t_{mini}+2t_{maxi}}{5},
\end{equation}
где $t_{\textnormal{ож}i}$ -- ожидаемая трудоемкость выполнения \textit{i}-ой работы чел.-дн.; $t_{mini}$ -- минимально возможная трудоемкость выполнения заданной \textit{i}-ой работы (оптимистическая оценка: в предположении наиболее благоприятного стечения обстоятельств), чел.-дн.; $t_{maxi}$ -- максимально возможная трудоемкость выполнения заданной \textit{i}-ой работы (пессимистическая   оценка:   в   предположении   наиболее неблагоприятного стечения обстоятельств), чел.-дн.

Исходя   из   ожидаемой   трудоемкости   работ,   определяется продолжительность  каждой  работы  в  рабочих  днях  $Т_р$,  учитывающая параллельность выполнения работ несколькими исполнителями:

%%% Продолжительность работы в рабочих днях
\begin{equation}\label{eq:eq-eco2}
T_{pi}
= \frac{t_{\textnormal{ож}i}}
{\textnormal{Ч}i},
\end{equation}
где  $T_{pi}$ -- продолжительность одной работы, раб.дн.; $\textnormal{Ч}_i$ -- численность исполнителей, выполняющих одновременно одну иту же работу на данном этапе, чел.

\subsubsection{Разработка графика проведения научного исследования} \label{eco.2.3}

В  соответствии  с  календарным  планом  выполнения  работ  был построен  ленточный  график  выполнения  дипломной  работы  в  форме диаграммы Ганта.

Для  удобства  построения  графика,  длительность  каждого  из  этапов работ  из  рабочих  дней  следует  перевести  в  календарные  дни.  Для  этого необходимо воспользоваться следующей формулой:

%%% Перевод рабочих дней в календарные
\begin{equation}\label{eq:eq-eco3}
T_{\textnormal{к}i} = k_{\textnormal{кал}} \cdot T_{pi},
\end{equation}
где $T_{\textnormal{к}i}$ -- продолжительность  выполнения \textit{i}-й  работы  в  календарных днях; $k_{\textnormal{кал}}$ -- коэффициент календарности.

Коэффициент   календарности   определяется   по   следующей формуле:

%%% Коэффициент календарности
\begin{equation}\label{eq:eq-eco4}
k_{\textnormal{кал}} 
= \frac{T_{\textnormal{кал}}}
{T_{\textnormal{кал}}-T_{\textnormal{вых}}-T_{\textnormal{пр}}},
\end{equation}
где $T_{\textnormal{кал}}$ -- количество календарных дней в году; $T_{\textnormal{вых}}$ -- количество выходных дней в году; $T_{\textnormal{пр}}$ -- количество праздничных дней в году.

Таким образом:

%%% Расчет коэффициента календарности
\begin{gather*}
k_{\textnormal{кал}} 
= \frac{T_{\textnormal{кал}}}
{T_{\textnormal{кал}}-T_{\textnormal{вых}}-T_{\textnormal{пр}}}
= \frac{365}{365-14-104} = 1,48.
\end{gather*}

Все рассчитанные значения необходимо свести в таблицу \ref{tab:eco9}.

%%% Временные показатели
\begin{table}[H]
\small
\centering
\caption{Временные показатели проведения научного исследования}
\label{tab:eco9}
\begin{tabular}{|l|l|c|c|c|c|c|c|c|}
\hline
\multicolumn{1}{|c|}{№} & \multicolumn{1}{c|}{\begin{tabular}[c]{@{}c@{}}Содержание\\ работы\end{tabular}}                & Исполнитель                                                       & $t_{min}$ & $t_{max}$ & $t_{\textnormal{ож}}$ & Ч & $\textnormal{Т}_\textnormal{п}$  & $\textnormal{Т}_\textnormal{к}$   \\ \hline
1                       & \begin{tabular}[c]{@{}l@{}}Составление и \\ утверждение \\ технического \\ задания\end{tabular} & \begin{tabular}[c]{@{}c@{}}Руководитель\\ , бакалавр\end{tabular} & 2         & 4         & 2,8      & 2 & 0,7 & 1,0  \\ \hline
2                       & \begin{tabular}[c]{@{}l@{}}Выбор \\ направления \\ исследований\end{tabular}                    & \begin{tabular}[c]{@{}c@{}}Руководитель\\ , бакалавр\end{tabular} & 2         & 3         & 2,4      & 2 & 1,2 & 1,8  \\ \hline
3                       & \begin{tabular}[c]{@{}l@{}}Подбор и изучение \\ материалов по \\ теме\end{tabular}              & \begin{tabular}[c]{@{}c@{}}Руководитель\\ , бакалавр\end{tabular} & 5         & 7         & 5,8      & 2 & 2,9 & 4,3  \\ \hline
4                       & \begin{tabular}[c]{@{}l@{}}Разработка \\ методики \\ выполнения работ\end{tabular}              & \begin{tabular}[c]{@{}c@{}}Руководитель\\ , бакалавр\end{tabular} & 2         & 3         & 2,4      & 2 & 1,2 & 1,8  \\ \hline
5                       & \begin{tabular}[c]{@{}l@{}}Составление \\ календарного \\ плана\end{tabular}                    & Руководитель                                                      & 1         & 2         & 1,4      & 1 & 1,4 & 2,1  \\ \hline
6                       & Поиск литературы                                                                                & \begin{tabular}[c]{@{}c@{}}Руководитель\\ , бакалавр\end{tabular} & 2         & 3         & 2,4      & 2 & 1,2 & 1,8  \\ \hline
7                       & \begin{tabular}[c]{@{}l@{}}Изучение \\ литературы\end{tabular}                                  & Бакалавр                                                          & 7         & 14        & 9,8      & 2 & 4,9 & 7,3  \\ \hline
8                       & \begin{tabular}[c]{@{}l@{}}Моделирование \\ электромагнитного\\ ливня\end{tabular}             & Бакалавр                                                          & 2         & 4         & 4,2      & 1 & 2,8 & 4,1  \\ \hline
9                       & \begin{tabular}[c]{@{}l@{}}Расчет \\ энергетического \\ разрешения \\ калориметра\end{tabular}  & Бакалавр                                                          & 2         & 4         & 4,2      & 1 & 2,8 & 4,1  \\ \hline
10                      & \begin{tabular}[c]{@{}l@{}}Калибровка \\ калориметра\end{tabular}                               & Бакалавр                                                          & 3         & 6         & 7        & 1 & 3,2 & 4,7  \\ \hline
11                      & \begin{tabular}[c]{@{}l@{}}Оценка \\ эффективности \\ полученных \\ результатов\end{tabular}    & Руководитель                                                      & 2         & 3         & 2,4      & 1 & 2,4 & 3,6  \\ \hline
12                      & \begin{tabular}[c]{@{}l@{}}Составление \\ отчета\end{tabular}                                   & Бакалавр                                                          & 7         & 14        & 9,8      & 1 & 8,8 & 14,5 \\ \hline
\end{tabular}%
\end{table}

На  основании полученных данных был  построен  план-график  в  виде диаграммы  Ганта.  График  строится  с  разбивкой  по  месяцам  и неделям (7дней) за период времени дипломирования.

 %%% Диаграмма Ганта
\begin{figure}[H]
    \centering
    \includegraphics[width=0.75\textwidth]{GANT.png}
    \caption{Диаграмма Ганта}
    \label{fig:gant}
\end{figure}

\subsection{Бюджет научно-технического исследования (НТИ)} \label{eco.3}

При  планировании  бюджета  НТИ  должно  быть  обеспечено  полное  и достоверное отражение всех видов расходов, связанных с его выполнением. В  процессе  формирования  бюджета  НТИ  используется  следующая группировка затрат по статьям:

%%% Составные бюджета
\begin{enumerate}[wide]
\item материальные затраты НТИ;
\item затраты на основное оборудование для научно-экспериментальных работ;
\item основная заработная плата исполнителей темы;
\item дополнительная заработная плата исполнителей темы;
\item отчисления во внебюджетные фонты (страховые отчисления);
\item накладные расходы.
\end{enumerate}

\subsubsection{Расчет затрат на оборудование для научно-экспериментальных работ} \label{eco.3.1}

Расчет   затрат   на   оборудование   сводится к   определению амортизационных  отчислений,  так  как  оборудование  было  приобретено  до начала выполнения этой работы. 

Норма амортизации вычисляется по следующей формуле:

%%% Норма амортизации
\begin{equation}\label{eq:eq-eco5}
N_a = \frac{1}{n},
\end{equation}
где \textit{n} -- срок полезного использования, измеряемый в годах.

Амортизация  оборудования  линейным  способом  рассчитывается следующим образом:

%%% Линейная амортизация
\begin{equation}\label{eq:eq-eco6}
A = \frac{N_a \cdot m \cdot N}{12}
\end{equation}
где \textit{N}–итоговая сумма, тыс. руб.;\textit{m}–время использования, мес.

Единственным  оборудованием,  использованным  в  работе  был  ПК \textit{DEXPMarsE320}, приобретенный в декабре 2020 года за 59999 рублей. Срок полезного использования ПК составляет 5 лет. В   итоге   общая   сумма амортизационных отчислений составила:

%%% Расчет амортизации
\begin{gather*}
A = \frac{0,2 \cdot 59999 \cdot 2}{12} 
= 1999,97 \approx 2000\ \textnormal{руб}.
\end{gather*}

\subsubsection{Основная заработная плата исполнителей темы} \label{eco.3.2}

Статья   включает   основную   заработную   плату   работников, непосредственно занятых выполнением НТИ, (включая премии и доплаты) и дополнительную   заработную   плату.   Также   включается   премия, выплачиваемая ежемесячно из фонда заработной платы в размере 20-30 \% от тарифа или оклада:

%%% ОЗП
\begin{equation}\label{eq:eq-eco7}
\textnormal{З}_{\textnormal{Зп}} 
= \textnormal{З}_{\textnormal{осн}}
= \textnormal{З}_{\textnormal{доп}}
\end{equation}
где $\textnormal{З}_{\textnormal{осн}}$ -- основная заработная плата; $\textnormal{З}_{\textnormal{доп}}$ -- дополнительная заработная плата (12-20 \% от $\textnormal{З}_{\textnormal{осн}}$).

Основная  заработная  платаруководителя  (лаборанта,  инженера)  от предприятия (при наличии руководителя от предприятия) рассчитывается по следующей формуле:

%%% ОЗП
\begin{equation}\label{eq:eq-eco8}
\textnormal{З}_{\textnormal{осн}} 
= \textnormal{З}_{\textnormal{дн}} \cdot T_p
\end{equation}
где $T_p$ -- продолжительность работ, выполняемых научно-техническим работником (таблица \ref{tab:eco9}); $\textnormal{З}_{\textnormal{дн}}$ -- среднедневная заработная плата работника, руб.

Среднедневная заработная плата рассчитывается по формуле:

%%% Среднедневная ЗП
\begin{equation}\label{eq:eq-eco9}
\textnormal{З}_{\textnormal{д}} 
= \frac{\textnormal{З}_{\textnormal{м}}}{F_{\textnormal{д}}}
\end{equation}
где   $\textnormal{З}_{\textnormal{м}}$ -- месячный должностной оклад работника, руб.; М – количество месяцев работы без отпуска в течение года: при отпуске в 24 раб.дня М =11,2 месяца, 5-дневная неделя; при отпуске в 48 раб.дней М=10,4 месяца, 6-дневная неделя; $F_{\textnormal{д}}$ – действительный  годовой  фонд  рабочего  времени  научно-технического персонала, раб.дн. (таблица \ref{tab:eco9}).

В таблице \ref{tab:eco10} приведен баланс рабочего времени каждого работника НТИ.

%%% Баланс рабочего времени
\begin{table}[H]
\centering
\caption{Баланс рабочего времени}
\label{tab:eco10}
\begin{tabular}{|l|c|c|}
\hline
Показатели рабочего времени                                                                                & \multicolumn{1}{l|}{Руководитель}                & \multicolumn{1}{l|}{Бакалавр}                    \\ \hline
Календарное число дней                                                                                     & 365                                              & 365                                              \\ \hline
\begin{tabular}[c]{@{}l@{}}Количество нерабочих дней\\  -- выходные дни\\  -- праздничные дни\end{tabular} & \begin{tabular}[c]{@{}c@{}} \\ 104\\ 14\end{tabular} & \begin{tabular}[c]{@{}c@{}} \\ 104\\ 14\end{tabular} \\ \hline
\begin{tabular}[c]{@{}l@{}}Потери рабочего времени\\  -- отпуск\\  -- невыходы по болезни\end{tabular}     & \begin{tabular}[c]{@{}c@{}}\\ 24\\ 7\end{tabular}   & \begin{tabular}[c]{@{}c@{}}\\ 24\\7\end{tabular}   \\ \hline
\begin{tabular}[c]{@{}l@{}}Действительный годовой фонд\\ рабочего времени\end{tabular}                     & 216                                              & 216                                              \\ \hline
\end{tabular}%
\end{table}

Месячный должностной оклад работника:

%%% Месячный оклад
\begin{equation}\label{eq:eq-eco10}
\textnormal{З}_{\textnormal{м}} 
= \textnormal{З}_{\textnormal{тс}} 
\cdot (1 
+ k_{\textnormal{пр}}
+ k_{\textnormal{д}}) \cdot k_p, 
\end{equation}
где   $\textnormal{З}_{\textnormal{тс}} $ -- заработная плата по тарифной ставке, руб.; $k_{\textnormal{пр}}$ -- премиальный коэффициент, равный 0,3; $k_{\textnormal{д}}$ -- коэффициент доплат и надбавок составляет0,2; $k_p$ -- районный коэффициент,для г. Томска равный 1,3.

Расчёт основной заработной платы приведѐн в таблице \ref{tab:eco11}.

%%% Расчет ОЗП
\begin{table}[H]
\centering
\caption{Расчет основной заработной платы}
\label{tab:eco11}
\begin{tabular}{|c|c|c|c|c|c|c|c|c|}
\hline
Категория & \begin{tabular}[c]{@{}c@{}}$\textnormal{З}_{\textnormal{тс}}$,\\ руб\end{tabular} & $k_{\textnormal{д}}$  & $k_{\textnormal{пр}}$ & $k_p$  & \begin{tabular}[c]{@{}c@{}}$\textnormal{З}_{\textnormal{м}} $,\\ руб\end{tabular} & \begin{tabular}[c]{@{}c@{}}$\textnormal{З}_{\textnormal{дн}} $,\\ руб\end{tabular} & \begin{tabular}[c]{@{}c@{}}$T_p$,\\ раб. дн.\end{tabular} & \begin{tabular}[c]{@{}c@{}}$\textnormal{З}_{\textnormal{осн}} $,\\ руб\end{tabular} \\ \hline
\multicolumn{9}{|c|}{Руководитель}                                                                                                                                                                                                                                                                       \\ \hline
ППС3      & 22000                                              & 0,3 & 0,2 & 1,3 & 42900                                             & 2224,4                                             & 17,7                                                   & 39372,7                                             \\ \hline
\multicolumn{9}{|c|}{Бакалавр}                                                                                                                                                                                                                                                                           \\ \hline
ППС1      & 9000                                               & 0,3 & 0,2 & 1,3 & 17550                                             & 910,0                                              & 30,7                                                   & 27937,0                                             \\ \hline
\multicolumn{8}{|c|}{\textbf{Итого}}                                                                                                                                                                                                               & 67309,7                                             \\ \hline
\end{tabular}%
\end{table}

Расчет  дополнительной  заработной  платы  ведется  по  следующей формуле:

%%%Допзп
\begin{equation}\label{eq:eq-eco11}
\textnormal{З}_{\textnormal{доп}} 
= k_{\textnormal{доп}}
\cdot \textnormal{З}_{\textnormal{осн}}, 
\end{equation}
где $k_\textnormal{доп}$ -- коэффициент  дополнительной  заработной  платы  (на  стадии проектирования принимается равным 0,15).

Общая заработная исполнителей работы представлена в таблице \ref{tab:eco12}.

\begin{table}[H]
\centering
\caption{Общая заработная плата исполнителей}
\label{tab:eco12}
\begin{tabular}{|c|c|c|c|}
\hline
Исполнитель    & $\textnormal{З}_{\textnormal{осн}}$, руб        & $\textnormal{З}_{\textnormal{доп}}$, руб        & $\textnormal{З}_{\textnormal{зп}}$, руб         \\ \hline
Руководитель   & 39372,7          & 5905,9           & 45278,6          \\ \hline
Бакалавр       & 27937,0          & 4190,6           & 32127,6          \\ \hline
\textbf{Итого} & \textbf{67309,7} & \textbf{10096,5} & \textbf{77406,1} \\ \hline
\end{tabular}%
\end{table}

\subsubsection{Отчисления во снебюджетные фонды (страховые отчисления)} \label{eco.3.3}

В  данной  статье  расходов  отражаются  обязательные  отчисления  по установленным  законодательством  Российской  Федерации  нормам  органам государственного социального страхования (ФСС), пенсионного фонда (ПФ) и  медицинского  страхования  (ФФОМС)  от  затрат  на  оплату  труда работников.

Величина отчислений во внебюджетные фонды определяется исходя из следующей формулы: 

%%% Отчисления
\begin{equation}\label{eq:eq-eco12}
\textnormal{З}_{\textnormal{внеб}} 
= k_{\textnormal{пр}} 
\cdot (\textnormal{З}_{\textnormal{осн}} 
+ (\textnormal{З}_{\textnormal{доп}}), 
\end{equation}
где  $k_{\textnormal{внеб}}$ – коэффициент  отчислений  на  уплату  во  внебюджетные  фонды (пенсионный  фонд,  фонд  обязательного  медицинского  страхования  и пр.). 

Отчисления во внебюджетные фонды представлены в таблице \ref{tab:eco13}.

%%% Внебюджетные фонды
\begin{table}[H]
\centering
\caption{Отчисления во внебюджетные фонды}
\label{tab:eco13}
\begin{tabular}{|c|c|c|}
\hline
Исполнитель                                                                             & \begin{tabular}[c]{@{}c@{}}Основная заработная \\ плата, руб.\end{tabular} & \begin{tabular}[c]{@{}c@{}}Дополнительная \\ заработная плата,\\ руб\end{tabular} \\ \hline
Руководитель проекта                                                                    & 39372,7                                                                    & 5905,9                                                                            \\ \hline
                                                                                        & 27937,0                                                                    & 4190,6                                                                            \\ \hline
\begin{tabular}[c]{@{}c@{}}Коэффициент отчислений \\ во внебюджетные фонты\end{tabular} & \multicolumn{2}{c|}{0,302}                                                                                                                                     \\ \hline
\textbf{Итого:}                                                                         & \multicolumn{2}{c|}{\textbf{23376,7}}                                                                                                                          \\ \hline
\end{tabular}%
\end{table}

\subsubsection{Накладные расходы} \label{eco.3.4}

Накладные  расходы  учитывают  прочие затраты  организации,  не попавшие  в  предыдущие  статьи  расходов:  печать  и  ксерокопирование материалов  исследования,  оплата  услуг  связи,  электроэнергии,  почтовые  и телеграфные  расходы,  размножение  материалов  и  т.д.  Их  величина определяется по следующей формуле:

%%% Накладные
\begin{equation}\label{eq:eq-eco13}
\textnormal{З}_{\textnormal{накл}} 
= k_{\textnormal{нр}} 
\cdot (\textnormal{\textit{сумма статей}}\ 1 \div 4), 
\end{equation}

где   $k_{\textnormal{нр}}$ – коэффициент, учитывающий накладные расходы.

Величину  коэффициента  накладных  расходов  можно  взять  в  размере 20 \%.

%%% Расчет накладных
\begin{gather*}
\textnormal{З}_{\textnormal{накл}} 
= 0,2 \cdot (2000.0 + 67309,7 + 10096,5 + 22362,1) = 20556,6 \ \textnormal{руб}
\end{gather*}

\subsubsection{Формирование бюджета затрат научно-исследовательского проекта} \label{eco.3.5}

Рассчитанная  величина  затрат  научно-исследовательской  работыявляется основой для формирования бюджета затрат проекта, который при формировании  договора  с  заказчиком защищается  научной  организацией  в качестве  нижнего  предела  затрат  на  разработку  научно-технической продукции.

Определение бюджета затрат на научно-исследовательский проект по каждому варианту исполнения приведен в таблице \ref{tab:eco14}.

%%% Бюджет затрат НТИ
\begin{table}[H]
\centering
\caption{Расчет бюджета затрат НТИ}
\label{tab:eco14}
\begin{tabular}{|c|c|}
\hline
Наименование статьи                                                                                                       & Сумма, руб \\ \hline
\begin{tabular}[c]{@{}c@{}}1. Затраты на специальное\\ оборудование для научных\\ (экспериментальных) работ\end{tabular} & 2000       \\ \hline
\begin{tabular}[c]{@{}c@{}}2. Затраты по оновной \\ заработной плате \\ исполнителей темы\end{tabular}                   & 67309,7    \\ \hline
\begin{tabular}[c]{@{}c@{}}3. Затраты по\\ дополнительной заработной \\ плате исполнителей темы\end{tabular}             & 10096,5    \\ \hline
\begin{tabular}[c]{@{}c@{}}4. Отчисления во\\ внебюджетные фонды\end{tabular}                                            & 23376,7    \\ \hline
5. Накладные расходы                                                                                                     & 20556,6    \\ \hline
6. Бюджет затрат НТИ                                                                                                     & 12339,3    \\ \hline
\end{tabular}%
\end{table}

Как  видно  из  таблицы \ref{tab:eco14} основные  затраты  НТИ  приходятся  назаработную плату исполнителей.

\subsection{Определение  ресурсной  (ресурсосберегающей),  финансовой, бюджетной, социальной и экономической эффективности исследования} \label{eco.4}

Для   определения    эффективности    исследования    рассчитан интегральный  показатель  эффективности  научного  исследования  путем определения  интегральных  показателей  финансовой  эффективности  и ресурсоэффективности.

Интегральный финансовый показатель разработки определяется как:

%%% Инфинпок
\begin{equation}\label{eq:eq-eco14}
I^{\textnormal{исп.i}}_{\textnormal{финр}}
= \frac{\textnormal{Ф}_{pi}}{\textnormal{Ф}_{max}}, 
\end{equation}
где $I^{\textnormal{исп.i}}_{\textnormal{финр}}$ –интегральный финансовый показатель разработки;$\textnormal{Ф}_{pi}$ – стоимость \textit{i}-го варианта исполнения; $\textnormal{Ф}_{max}$ – максимальная стоимость исполнения научно-исследовательского проекта (в т.ч. аналоги).

Интегральный    показатель    ресурсоэффективности    вариантов исполнения объекта исследования можно определить следующим образом:

%%% Интегральный показатель
\begin{equation}\label{eq:eq-eco15}
I_{pi} = \sum a_i \cdot b_i
\end{equation}
где $I_{pi}$ – интегральный показатель ресурсоэффективности для \textit{i}-го варианта исполнения разработки; $a_i$ – весовой коэффициент \textit{i}-го варианта исполнения разработки; $b_i^a$, $b_i^p$ – бальная  оценка \textit{i}-го  варианта  исполнения  разработки, устанавливается экспертным путем по выбранной шкале оценивания; \textit{n} – число параметров сравнения.

Расчет интегрального показателя ресурсоэффективности приведен в форме таблицы \ref{tab:eco15}.

%%%Варианты исполнения
\begin{table}[H]
\small
\centering
\caption{Сравнительная оценка характеристик вариантов исполнения проекта}
\label{tab:eco15}
\begin{tabular}{|c|c|c|c|c|}
\hline
\begin{tabular}[c]{@{}c@{}}\backslashbox{Объект исследования}{Критерии}\end{tabular}          & \begin{tabular}[c]{@{}c@{}}Весовой \\ коэффициент \\ параметра\end{tabular} & \begin{tabular}[c]{@{}c@{}}Текущий \\ проект\end{tabular} & Исп.2 & Исп.3 \\ \hline
1. Универсальность методики                                                     & 0,4                                                                         & 5                                                         & 1     & 2     \\ \hline
2. Трудоемкость реализации                                                      & 0,2                                                                         & 5                                                         & 2     & 1     \\ \hline
\begin{tabular}[c]{@{}c@{}}3. Учет инструментальных\\ особенностей\end{tabular} & 0,4                                                                         & 1                                                         & 5     & 5     \\ \hline
Итого:                                                                          & 1                                                                           & 3,4                                                       & 2,8   & 3     \\ \hline
\end{tabular}
\end{table}

Сравнив  значения  интегральных  показателей  ресурсоэффективности можно сделать вывод, что реализация методики в текущем проекте является более   эффективным   вариантом   для   проектирования   с   позиции ресурсосбережения.

Интегральный  показатель  эффективности  вариантов  исполнения разработки($I_{\textnormal{исп.i}}$)  определяется  на  основании  интегрального  показателя ресурсоэффективности и интегрального финансового показателя по формуле:

%%% Интегральный показатель эффективности
\begin{equation}\label{eq:eq-eco16}
I_{\textnormal{исп.2}} 
= \frac{I_{\textnormal{p-исп.2}}}
{I^{\textnormal{исп.i}}_{\textnormal{финр}}}
,
I_{\textnormal{исп.1}} 
= \frac{I_{\textnormal{p-исп.1}}}
{I^{\textnormal{исп.i}}_{\textnormal{финр}}}
,\ \textnormal{и т. д.}
\end{equation}

Сравнение   интегрального   показателя   эффективности   вариантов исполнения разработки позволит определить сравнительную эффективность проекта  (см.  таблицу  \ref{tab:eco16})  и  выбрать  наиболее  целесообразный  вариант  из предложенных. Сравнительная эффективность проекта ($\textnormal{Э}_{\textnormal{ср}}$):

%%% Сравнительная эффективность проекта
\begin{equation}\label{eq:eq-eco17}
\textnormal{Э}_{\textnormal{ср}} 
= \frac{I_{\textnormal{исп.1}}}
{I_{\textnormal{исп.2}}}.
\end{equation}

%%% Сравнительная эффективность разработки
\begin{table}[H]
\centering
\caption{Сравнительная эффективность разработки}
\label{tab:eco16}
\begin{tabular}{|l|l|c|c|c|}
\hline
\multicolumn{1}{|c|}{\begin{tabular}[c]{@{}c@{}}№\\ п/п\end{tabular}} & \multicolumn{1}{c|}{Показатели}                                                                    & \begin{tabular}[c]{@{}c@{}}Текущий\\ проект\end{tabular} & Исп.2 & Исп.3 \\ \hline
1                                                                     & \begin{tabular}[c]{@{}l@{}}Интегральный финансовый показатель\\ разработки\end{tabular}            & 0,97                                                     & 0,97  & 1,00  \\ \hline
2                                                                     & \begin{tabular}[c]{@{}l@{}}Интегральный ппоказатель\\ ресурсоэффективности разработки\end{tabular} & 3,40                                                     & 2,80  & 3,00  \\ \hline
3                                                                     & Интегральный показатель эффективности                                                              & 3,50                                                     & 2,88  & 3,00  \\ \hline
4                                                                     & \begin{tabular}[c]{@{}l@{}}Сравнительная эффективность вариантов \\ исполнения\end{tabular}        & 1                                                        & 0,82  & 0,86  \\ \hline
\end{tabular}
\end{table}

Вывод: сравнительный    анализ    интегральных    показателей эффективности   показывает,   что   предпочтительным   для   осуществления калибровки является  первый  вариант  исполнения, так  как является наиболее экономичным и ресурсоэффективным.

В результате выполнения целей раздела можно сделать следующие выводы:

%%% Выводы
\begin{enumerate}[wide]
\item В  результате  анализа  конкурентных  решений  выяснили,  что выбранная методика калибровки является наиболее эффективной.
\item В  ходе  планирования  для  руководителя,  консультантов  по социальной ответственности  и  экономической  части  и  бакалаврабыл разработан график реализации этапа работ, который позволяет оценивать и планировать  рабочее  время  исполнителей.  Определено  следующее:  общее количество дней для выполнения работ составляет 48дней.
\item Для  оценки  затрат  на  реализацию  проекта  разработан  проектный бюджет, который составляет 123339,3 руб;
\end{enumerate}

Результат оценки эффективности ИР показывает следующее:

%%% Еще выводы
\begin{enumerate}[wide]
\item значение интегрального финансового показателя ИР составляет 0,97;
\item значение  интегрального  показателя  ресурсоэффективности  ИР составляет 3.40, в то время как при других вариантах исполнения значения показателя составляют 2,80 и 3,00.
\item значение  интегрального  показателя  эффективности  ИР  составляет 3,50 по сравнению с 2.88 и 3,00, и является наиболее высоким, что означает, чтотехническое  решение,  рассматриваемое  в  ИР  является  наиболее эффективным вариантом исполнения.
\end{enumerate}