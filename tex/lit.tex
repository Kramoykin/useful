\newpage
\section*{Литература} \label{sec:lit}
\addcontentsline{toc}{section}{Литература} 

%%% Miscalibrations explanation
 \begin{itemize}[leftmargin=1.6\parindent, wide]
\item [1.] Bertone G., Hooper D. History of dark matter //Reviews of Modern Physics. – 2018. – Т. 90. – №. 4. – С. 045002.
\item [2.] Bednyakov V. A., Klapdor-Kleingrothaus H. V. Direct search for dark matter—striking the balance—and the future //Physics of Particles and Nuclei. – 2009. – Т. 40. – №. 5. – С. 583-611.
\item [3.] Bjorken J. D. et al. New fixed-target experiments to search for dark gauge forces //Physical Review D. – 2009. – Т. 80. – №. 7. – С. 075018.
\item [4.] Wigmans   R.,   Wigmans   R.   Calorimetry:   Energy   measurement   in particle physics. –Oxford University Press, 2000. – Т. 107.
\item [5.] Adinolfi  M.  et  al.  The  KLOE  electromagnetic  calorimeter  //Nuclear Instruments   and   Methods   in   Physics   Research   Section   A:   Accelerators, Spectrometers, Detectors and Associated Equipment. – 2002. – Т. 482. – No. 1-2. – С. 364-386.
\item [6.] Gingrich D. M. et al. Performance of a large scale prototype of the ATLAS accordion electromagnetic calorimeter //Nuclear Instruments and Methods in Physics Research Section A: Accelerators, Spectrometers, Detectors and Associated Equipment. – 1995. – Т. 364. – №. 2. – С. 290-306.
\item[7.] Akchurin N. et al. Beam test results from a fine-sampling quartz fiber calorimeter for electron, photon and hadron detection //Nuclear Instruments and Methods in Physics Research Section A: Accelerators, Spectrometers, Detectors and Associated Equipment. – 1997. – Т. 399. – №. 2-3. – С. 202-226.
\item[8.] Peigneux J. P. et al. Results from tests on matrices of lead tungstate crystals using high energy beams //Nuclear Instruments and Methods in Physics Research Section A: Accelerators, Spectrometers, Detectors and Associated Equipment. – 1996. – Т. 378. – №. 3. – С. 410-426.
\item[9.] Åkesson T. et al. Performance of the uranium/plastic scintillator calorimeter for the HELIOS experiment at CERN //Nuclear Instruments and Methods in Physics Research Section A: Accelerators, Spectrometers, Detectors and Associated Equipment. – 1987. – Т. 262. – №. 2-3. – С. 243-263.
\item[10.] Cervelli F. et al. A reduced scale em calorimeter prototype for the AMS-02 experiment //Nuclear Instruments and Methods in Physics Research Section A: Accelerators, Spectrometers, Detectors and Associated Equipment. – 2002. – Т. 490. – №. 1-2. – С. 132-139.
\item[11.] Aharrouche M. et al. Energy linearity and resolution of the ATLAS electromagnetic barrel calorimeter in an electron test-beam //Nuclear Instruments and Methods in Physics Research Section A: Accelerators, Spectrometers, Detectors and Associated Equipment. – 2006. – Т. 568. – №. 2. – С. 601-623.
\item[12.] Ganel O., Wigmans R. On the calibration of longitudinally segmented calorimeter systems //Nuclear Instruments and Methods in Physics Research Section A: Accelerators, Spectrometers, Detectors and Associated Equipment. – 1998. – Т. 409. – №. 1-3. – С. 621-628.
\item[13.] Albrow  M.  et  al.  Intercalibration  of  the  longitudinal  segments  of  a calorimeter   system   //Nuclear   Instruments   and   Methods   in   Physics   Research Section  A:  Accelerators,Spectrometers,  Detectors  and  Associated  Equipment. – 2002. – Т. 487. – №. 3. – С. 381-395.
\item[14.] Grindhammer G., Peters S. The parameterized simulation of electromagnetic showers in homogeneous and sampling calorimeters //arXiv preprint hep-ex/0001020. – 2000.
\item[15.] Grindhammer  G.,  Rudowicz  M.,  Peters  S.The  fast  simulation  of electromagnetic  and  hadronic  showers  //Nuclear  Instruments  and  Methods  in Physics   Research   Section   A:   Accelerators,   Spectrometers,   Detectors   and Associated Equipment. – 1990. – Т. 290. – No. 2-3. – С. 469-488.
\item[16.] Del Peso J., Ros E. On the energy resolution of electromagnetic sampling calorimeters //Nuclear Instruments and Methods in Physics Research Section A: Accelerators, Spectrometers, Detectors and Associated Equipment. – 1989. – Т. 276. – №. 3. – С. 456-467.
\item[17.] Galassi M. et al. GNU scientific library //Reference Manual. Edition 1.4, for GSL Version 1.4. – 2003.
\item[18.] ГОСТ  12.0.003-2015  Система  стандартов  безопасности  труда  (ССБТ). Опасные  и  вредные  производственные  факторы.  Классификация Режим доступа: https://docs.cntd.ru/document/1200136071/ (дата обращения: 14.02.21)
\item[19.] ГОСТ  30494-96 Здания  жилые  и общественные.  Параметры микроклимата в помещениях Режим доступа: http://docs.cntd.ru/document/1200003003 (дата обращения: 15.02.21)
\item[20.] ГОСТ 12.1.003-83 Система стандартов безопасности труда (ССБТ). Шум. Общие  требования  безопасности  (с  Изменением  № 1) Режим  доступа: http://docs.cntd.ru/document/5200291 (дата обращения: 15.02.21)
\item[21.] СНиП  23-05-95*  Естественное  и  искусственное  освещение  (с Изменением № 1) Режим доступа:http://docs.cntd.ru/document/871001026 (дата обращения: 15.02.21)
\item[22.] СП  12.13130.2009.  Определение  категорий  помещений,  зданий  и наружных установок по взрывопожарной и пожарной опасности (в ред. изм. №  1, утв. приказом МЧС России от 09.12.2010 № 643). [Электронный ресурс]. Доступ из сборника НСИС ПБ. –2011. – № 2 (45).
\item[23.] ГОСТ 12.1.004-91  Система  стандартов  безопасности  труда.  Пожарная безопасность. Общие требования Режим доступа: https://docs.cntd.ru/document/9051953 (дата обращения: 03.03.2021)
\item[24.] ГОСТ  12.1.009-76  Система  стандартов  безопасности  труда  (ССБТ). Электробезопасность.   Термины   и   определения Режим   доступа: http://docs.cntd.ru/document/5200278 (дата обращения: 18.02.21)
\item[25.] ГОСТ 12.1.019-2017  ССБТ  Электробезопасность Режим  доступа: https://beta.docs.cntd.ru/document/1200161238 (дата обращения: 19.02.21)
\item[26.] ГОСТ Р МЭК 61140-2000 Защита  от  поражения  электрическим  током. Общие положения по безопасности, обеспечиваемой электрооборудованием и   электроустановками   в   их   взаимосвязи.   Режим   доступа: https://docs.cntd.ru/document/1200017996 (дата обращения: 05.03.2021)
\item[27.] СанПиН 2.6.1.2523-09 Нормы радиационной безопасности НРБ-99/2009 Режим доступа: https://base.garant.ru/4188851/53f89421bbdaf741eb2d1ecc4ddb4c33/ (дата обращения: 21.02.21)
\item[28.] СанПиН  2.2.4.548-96  Гигиенические  требования  к  микроклимату производственных помещений Режим доступа: http://docs.cntd.ru/document/901704046 (дата обращения: 15.02.21)
\item[29.] СНиП 41-01-2003 Отопление, вентиляция и кондиционирование Режим доступа: http://docs.cntd.ru/document/1200035579 (дата обращения: 15.02.21)
\item[30.] ГОСТ 12.1.029-80 Средства и методы защиты от шума. Режим доступа: http://docs.cntd.ru/document/5200292 (дата обращения: 15.02.21)
\item[31.] ГОСТ  12.4.026-76*  Система  стандартов  безопасности  труда.  Цвета сигнальные    и    знаки    безопасности Режим    доступа: http://docs.cntd.ru/document/1200003391 (дата обращения: 15.02.21)
\item[32.] СанПиН  2.2.2/2.4.1340-03  О  введении  в  действие  санитарно-эпидемиологических   правил   и   нормативов Режим   доступа: http://docs.cntd.ru/document/901865498 (дата обращения: 16.02.21)
\item[33.] ГОСТ  12.1.006-84  Электромагнитные поля  радиочастот.  Допустимые уровни  на  рабочих  местах  и  требования  к  проведению  контроля Режим доступа: http://docs.cntd.ru/document/5200272 (дата обращения: 16.02.21)
\item[34.] ГОСТ Р 22.0.02-2016  Безопасность  в  чрезвычайных  ситуациях. Термины и определения Режим доступа: https://docs.cntd.ru/document/1200139176 (дата обращения: 11.03.2021)
 \end{itemize}