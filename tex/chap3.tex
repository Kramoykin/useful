\newpage
\section{Методики калибровки электромагнитных калориметров}  \label{chap3}  

Калибровка,  или  же  установление  отношения  между  выделившейся энергией  и  сигналом  калориметра  является,  возможно,  наиболее  важным аспектом  работы. Во  втором  разделе  было  упомянуто,  что  устройство калориметра  определяется  процессами,  происходящими  на  последних стадиях развития ливня. Эту особенность нужно учитывать при калибровке продольно сегментированных устройств. 

В  электромагнитном  ливне,  развивающемся  внутри гетерогенного калориметра, сэмплирующая доля для мягких гамма-квантов отличается от аналогичной  для \textit{mip}.  Поэтому  общая  сэмплирующая  доля  является  функцией глубины или возраста ливня. Это иллюстрируется на рис. \ref{fig:eDivMipDepth}. Этот эффект зависит не только от \textit{Z} активного и пассивного материалов, но также и от энергии ливня.

%%% e/mip as depth function (from Wigmans)
\begin{figure}[H]
    \centering
    \includegraphics[width=0.75\textwidth]{11.jpg}
    \caption{Отношение \textit{e/mip} как функция глубины ливня, образованного электроном с энергией \mbox{1 ГэВ} в гетерогенных электромагнитных калориметров различных конфигураций. Результат симуляции в программе \textit{EGS4} [4]}
    \label{fig:eDivMipDepth}
\end{figure}

Чем  ниже  энергия  ливня,  тем  раньше  преобладающими  станут  мягкие частицы  Комптоновского  рассеяния  и  фотоэффекта.  Если  калориметр продольно  сегментирован,  то  отношение между  выделившейся  энергией  и результирующим  сигналом  калориметра  будет  различным  для  разных сегментов.  Как  результат,  энергия,  выделившаяся  в  различных  сегментах, систематически  недооценивается (или  переоценивается),  причем  величина ошибки зависит  от  энергии  инициирующей  ливень  частицы. Это демонстрируется рис. \ref{fig:helCal}, на котором изображен размер просчетов для двух секций,  составляющих  калориметр \textit{HELIOS} [9]. Энергия  в  первой  секции (глубиной 6,6 $X_0$) систематически переоценивается, энергия во второй секции систематически недооценивается.

%%% energy fraction in the 1st and 2nd segments of HELIOS calorimeter
\begin{figure}[H]
    \centering
    \includegraphics[width=0.75\textwidth]{12.jpg}
    \caption{Ошибка при измерении энергии, выделившейся в конкретной секции продольно сегментированного калориметра \textit{HELIOS}, как функция энергии электронов, образующих ливень (нижняя ось), или доли энергии, выделившейся в первом сегменте (верхняя ось) [4]}
    \label{fig:helCal}
\end{figure}

На практике в подобных случаях принято определять калибровочные константы для этих двух секций и возникает вопрос о том, как это сделать. Практически все используемые методы не приводят к должному результату. Например,  для калибровки  упомянутого  выше  калориметра \textit{HELIOS}, используется  метод,в  котором  калибровочные  константы  выбираются  так, чтобы  минимизировать  ширину  сигнала,  соответствующего  следующей формуле [4]:

%%% HELIOS calibration condition
\begin{equation}\label{eq:helCalCond}
Q = \sum_{j=1}^N
\left[
E-A\sum_{i=1}^n S_{ij}^A
- B\sum_{i=1}^n S_{ij}^B
\right]^2,
\end{equation}

где \textit{A} – калибровочная  константа  для  первой  секции, \textit{B} – калибровочная константа для второй секции.

Однако выяснилось,  что  значения  констант \textit{A} и \textit{B}, а  особенно  их отношение  зависит  от  энергии  электронов,  которые  использовались  для калибровки детектора. Эта зависимость показана на рис. \ref{fig:helCal},а. В частности было установлено, что эти константы отличаются  от единицы, то есть обе секции  недокалиброваны,  относительно  мюонов,  которые  сэмплируются обеими  секциями  совершенно  одинаково.  Установлено,  что  такой  метод калибровки приводит к следующим нежелательным последствиям:

\begin{itemize}[leftmargin=1.6\parindent, wide]
\item[---] зависимость калибровочных констант от энергии;
\item[---] нелинейность электромагнитного отклика;
\item[---] систематические отличия 	в откликах электронов, гамма-квантов и пионов.
\end{itemize}

Это  приводит,  практически  во  всех  случаях,  к  зависимости реконструируемой  энергии  от  точки  начала  ливня  и  к  систематическим просчетам  при  измерении  энергии  джетов.  Далее  приведены  несколько методов, которые широко используются на практике для того, чтобы снизить влияние подобных нежелательных эффектов. 

Существует  множество  экспериментов,  в  которых  продольная сегментация  является  причиной проблем,  возникающих  при калибровке. В качестве примера можно упомянуть эксперимент \textit{AMS} [10]. Его свинцово-сцинтилляционный калориметр с оптическими волокнами по длине разделен на 18 сегментов, каждый толщиной примерно 1 $X_0$. Каждый из этих сегментов  калибруется  на \textit{mip} и  энергетический  эквивалент  прохождения одного \textit{mip} через сегмент был установлен равным \mbox{11,7 МэВ}. Как бы то ни было,  полной  длины  калориметра  (17 $X_0$)  недостаточно  для  того,  чтобы полностью поглотить энергию ливня, что демонстрируется на рис. \ref{fig:leakAMS}(а). Как результат, общий сигнал калориметра не пропорционален энергии пучка. Чем больше энергия пучка, тем больше утечки (рис. \ref{fig:leakAMS},б). Путем интегрирования функции,  аппроксимирующей  профиль  пучка,  от  нуля  до  бесконечности, оказывается возможным восстановить лишь часть потерянной энергии.

%%% AMS calibration with energy leakege correction
\begin{figure}[H]
    \centering
    \includegraphics[width=0.75\textwidth]{13.jpg}
    \caption{Средние сигналы, вызванные электроном с энергией \mbox{100 ГэВ} в 18 продольных секциях калориметра эксперимента \textit{AMS} (а). Усредненная разность между измеренной энергией и энергией пучка до и после поправок, учитывающих утечки (б) }
    \label{fig:leakAMS}
\end{figure}

Это объясняется тем, что сигналы, снятые с участков за максимумом ливня соответствуют  значительно  большей  энергии,  чем  сигналы  из сегментов,  в которых происходит раннее развитие ливня. При использовании одинаковых коэффициентов преобразования сигнала в энергию для всех участков модуля, поперечные утечки систематически будут недооцениваться. Таким образом, реконструированная  энергия  ливня  систематически  оказывается  слишком маленькой, особенно, если доля утечек велика. 

Коллаборацией эксперимента \textit{ATLAS} был разработан подход, который на  практике  приводит  к  достаточно  хорошим  результатам.  В  этом эксперименте  используется  калориметр  из  свинца  и  жидкого  аргона, состоящий  из  трех  продольных  сегментов  (толщиной 4,3 $X_0$,  16 $X_0$ и 2 $X_0$ соответственно).  Сэмплирующая  доля  энерговыделения  в  этом  детекторе значительно  понижается  с  ростом  длины,  несмотря  на  его  однородную структуру.  Калибровочные  коэффициенты  были  найдены  на  основании тщательной  Монте-Карло  симуляции  таким  образом,  чтобы  достичь одновременно   хорошего   разрешения   и   линейности   сигнала [11]. Реконструкция  энергии  по  измеренному  сигналу  производилась  на основании  формулы,  содержащей  по  крайней  мере  4  параметра,  которые нелинейно зависели от энергии электронов пучка. Благодаря этой формуле авторы  достигли линейности  в  интервале  энергий  \mbox{15-180 ГэВ}. В то же время, рассчитанные подобным образом коэффиценты корректны только для единственной псевдобыстроты, в то время как значения параметров должны отличаться в том случае, когда сигнал производят фотоны, а не электроны. Возникает  также  вопрос  о  том, как  экстраполировать  эти  результаты  для значений,  лежащих  за  границами  интервала  энергий,  в  котором  они  были получены.  Для  \mbox{10  ГэВ}  уже  наблюдались  серьезные  отклонения  от линейности сигнала. В  заключение  стоит  отметить,  что  калибровка  в  первую  очередь должна  приводить  к  корректной  реконструкции  энергии  порождающей ливень частицы. Это условие серьезно отличается от требований касательно ширины распределения сигналов, линейности сигнала или других желаемых особенностей, которые часто формируют основу методики калибровки [12, 13]. 
