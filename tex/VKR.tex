% !TEX program = xelatex
%----------------------- Преамбула -----------------------
\documentclass[utf8x, 14pt, oneside, a4paper]{article}

\usepackage{extsizes} % Для добавления в параметры класса документа 14pt

% Для работы с несколькими языками и шрифтом Times New Roman по-умолчанию
\usepackage[english,russian]{babel}
\usepackage{fontspec}
\setmainfont{Times New Roman}

% ГОСТовские настройки для полей и абзацев
\usepackage[left=30mm,right=15mm,top=20mm,bottom=20mm]{geometry}
\usepackage{misccorr}
\usepackage{indentfirst}
\usepackage{enumitem}
\setlength{\parindent}{1.25cm}
\linespread{1.3}
\setlist{nolistsep} % Отсутствие отступов между элементами \enumerate и \itemize

% Дополнительное окружения для подписей
\usepackage{array}
\newenvironment{signstabular}[1][1]{
	\renewcommand*{\arraystretch}{#1}
	\tabular
}{
	\endtabular
}

% Переопределение стандартных \section, \subsection, \subsubsection по ГОСТу;
% Переопределение их отступов до и после для 1.5 интервала во всем документе
\usepackage{titlesec}

\titleformat{\section}[block]
    {\bfseries\normalsize}
    {\hspace{1.25cm}\thesection}
    {1ex}{}   
    
\titleformat{name=\section,numberless}[block]
  {\bfseries\normalsize}
  {}
  {0pt}
  {\hspace*{1.25cm}}

\titleformat{\subsection}[block]
    {\bfseries\normalsize}
    {\hspace{1.25cm}\thesubsection}
    {1ex}{}

\titleformat{\subsubsection}[block]
    {\bfseries\normalsize}
    {\hspace{1.25cm}\thesubsubsection}
    {1ex}{}

% Работа с изображениями и таблицами; переопределение названий по ГОСТу
\usepackage[justification=centering]{caption}
\captionsetup[figure]{name={Рисунок},labelsep=endash}
\captionsetup[table]{singlelinecheck=false, labelsep=endash}

\usepackage{graphicx}
\usepackage{slashbox} % Диагональное разделение первой ячейки в таблицах
\usepackage{multirow}

% Цвета для гиперссылок и листингов
\usepackage{color}

% Гиперссылки \toc с кликабельностью
\usepackage{hyperref}

\hypersetup{
	linktoc=all,
	linkcolor=black,
	colorlinks=true,
}

% Листинги
\setsansfont{Arial}
\setmonofont{Courier New}

\usepackage{color} % Цвета для гиперссылок и листингов
\definecolor{comment}{rgb}{0,0.5,0}
\definecolor{plain}{rgb}{0.2,0.2,0.2}
\definecolor{string}{rgb}{0.91,0.45,0.32}

\usepackage{listings}
\lstset{
	basicstyle=\footnotesize\ttfamily,
	language=[Sharp]C, % Или другой ваш язык -- см. документацию пакета
	commentstyle=\color{comment},
	numbers=left,
	numberstyle=\tiny\color{plain},
	numbersep=5pt,
	tabsize=4,
	extendedchars=\true,
	breaklines=true,
	keywordstyle=\color{blue},
	frame=b,
	stringstyle=\ttfamily\color{string}\ttfamily,
	showspaces=false,
	showtabs=false,
	xleftmargin=17pt,
	framexleftmargin=17pt,
	framexrightmargin=5pt,
	framexbottommargin=4pt,
	showstringspaces=false,
	inputencoding=utf8x,
	keepspaces=true
}

\DeclareCaptionLabelSeparator{line}{\ --\ }
\DeclareCaptionFont{white}{\color{white}}
\DeclareCaptionFormat{listing}{\colorbox[cmyk]{0.43,0.35,0.35,0.01}{\parbox{\textwidth}{\hspace{15pt}#1#2#3}}}
\captionsetup[lstlisting]{
	format=listing,
	labelfont=white,
	textfont=white,
	singlelinecheck=false,
	margin=0pt,
	font={bf,footnotesize},
	labelsep=line
}
%%% Правильные отступы в подписях таблиц
\newlength\myindention
\DeclareCaptionFormat{myformat}%
{\hspace*{\myindention}#1#2#3}
\setlength\myindention{1.25cm}
\captionsetup{format=myformat}
\captionsetup[table]{justification = justified,format=myformat}

% Годные пакеты для обычных действий
\usepackage{ulem} % Нормальное нижнее подчеркивание
\usepackage{hhline} % Двойная горизонтальная линия в таблицах
\usepackage[figure,table]{totalcount} % Подсчет изображений, таблиц
\usepackage{rotating} % Поворот изображения вместе с названием
\usepackage{lastpage} % Для подсчета числа страниц

% Быстрый и грязный способ избежать ошибки с титульником
\renewcommand\maketitle{}

% Директория с рисуночками
\graphicspath{ {./pic/after} }

%%% Запрет деления слов (перенос слова на новую строку)
\tolerance=1
\emergencystretch=\maxdimen
\hyphenpenalty=10000
\hbadness=10000

%%% Делаем рисунки плавающими
\usepackage{float}

%%% Доп математика
\usepackage{mathtools}

%%% Косая черта в таблицах
\usepackage{diagbox}

% Значок градусов
\usepackage{gensymb}

% Вставяем титульники
\usepackage{pdfpages}

% ---------------------- Документ ---------------------- 
\begin{document}

%%% Меняем "Рис._" на "Рисунок_"
\def\figurename{Рисунок}

%%%\maketitle

\includepdf[page=-]{1Tit.pdf}

\includepdf[page=-]{2Osv.pdf}

\includepdf[page=-]{3Zad.pdf}

\includepdf[page=-]{4Graf.pdf}

\includepdf[page=-]{5Titeco.pdf}

\includepdf[page=-]{6Titsoc.pdf}

\includepdf[page=-]{7Ref.pdf}

\tableofcontents

\newpage
\section*{Введение}  \label{sec:intro}
\addcontentsline{toc}{section}{Введение}

В 30-х годах XX века наблюдения за движением материи во Вселенной стали указывать на наличие во Вселенной значительной массы, недоступной для прямой регистрации и проявляющей себя только в гравитационном взаимодействии [1]. Эта скрытая масса была названа тёмной материей. Однако, за прошедшее время темную материю не удалось зарегестрировать непосредственно, и до сих пор она остается одной из важнейших загадок современных ксомологии и астрофизики.

Есть основания полагать [2], что поиски темной материи можно проводить не только методами астрофизики, но и на ускорителях частиц..  Предполагается, что при столкновении высокоэнергетичных частиц между собой или с веществом мишени могут образовываться частицы тёмной материи. Такие эксперименты обычно подразумевают использование калориметров для регистрации полной энергии в различных реакциях (герметичная постановка эксперимента со сбросом пучка -- \textit{beam dump}) [3].

Изначально калориметры производились как довольно грубые, но дешевые инструменты для специализированного применения. Например, для детектирования нейтринных взаимодействий. В современных коллайдерных экспериментах они являются одними из ключевых элементов в телескопах детекторов. Они подходят для множества задач, начиная от отбора событий и заканчивая точными измерениями четырехвекторов отдельных частиц и сгустков частиц и получением информации об энерговыделении в результате различных событий.
	
Вклад калориметрической информации в анализ данных заключается, в основном, в идентификации частиц и измерении энергии частиц, порождающих электромагнитные ливни. Ожидается, что значение адронной калориметрии будет расти с дальнейшим ростом энергий сталкивающихся частиц.
            
\textit{NA64} – это эксперимент с фиксированной мишенью, на протонном суперсинхротроне (\textit{SPS}), расположенном в Европейском центре ядерных исследований (\textit{CERN}), и направленный на поиск экзотических легких бозонов. Ключевым детектором в постановке эксперимента является гетерогенный электромагнитный калориметр \textit{ECAL}. Для получения достоверных результатов необходима тщательная калибровка калориметра. 

Одним из наиболее результативных методов калибровки является калибровка, основанная на компьютерном моделировании. Целью настоящей работы является построение модели электромагнитного ливня для реконструкции энерговыделения в калориметре \textit{ECAL}.

Задачи:

 \begin{itemize}[leftmargin=1.6\parindent, wide]
 	\item[---] определение структуры работ в рамках научного исследования;
 		\item[---] обзор литературы с целью определения наиболее подходящей параметризации электромагнитного ливня для известных конструкции калориметра и диапазона энергий;
 			\item[---] получение профиля электромагнитного ливня на основе МК-модели, достижение согласия результатов моделирования с предсказаниями параметризации;
 				\item[---] определение энергетического разрешения электромагнитного калориметра.
 \end{itemize}


\newpage
\section{Физические процессы в калориметрах}  \label{chap1}
 
 Существует несколько процессов, играющих роль в развитии электромагнитного ливня. Электроны и позитроны теряют энергию на ионизацию и излучение. Первый процесс преобладает при низких энергиях, второй – при высоких. Критическая энергия, при которой роль обоих процессов одинакова в первом приближении обратно пропорциональна зарядовому числу вещества поглотителя [4]:
 
%%% Kritical energy formula (from Wigmans)
\begin{equation}\label{eq:kriten}
\epsilon_c = \frac{610\ \mathrm{MeV}}{Z+1,24}
\end{equation}
 
Фотоны взаимодействуют с веществом преимущественно посредством фотоэффекта, Комптоновского рассеяния и эффекта образования электро-позитронных пар. Фотоэффект преобладает при низких энергиях, а образование пар при высоких. Соответствующие сечения зависят от {\textit Z}. Например, сечение фотоэффекта пропорционально $Z^{5}$ и $E^{-3}$, в то время как сечение образования пар постепенно возрастает с ростом {\textit Z} и {\textit E}, асимптотически приближаясь к определенному значению при значениях энергии порядка \mbox{1 ГэВ}. Угловое распределение более или менее изотропно для фото- и комптоновских электронов, но имеет строго преобладающее направление для электронов и позитронов, рожденных в результате образования пар.

Начиная с энергий от \mbox{1 ГэВ} и выше, электроны и фотоны образуют электромагнитный ливень в веществе, в которое они проникают. Электроны теряют свою энергию преимущественно на излучение, фотоны с наибольшей энергией, рожденные в этом процессе, конвертируются в электрон-позитронные пары, которые также излучают фотоны и т. д.  Число частиц, образующихся в развивающемся ливне достигает максимума на определенной глубине внутри поглотителя, после чего постепенно убывает. Глубина, на которой ливень достигает своего максимума, логарифмически возрастает с увеличением энергии инициирующей ливень частицы.

%%% Deposited energy graphic (from Wigmans)
\begin{figure}[H]
    \centering
    \includegraphics[width=0.75\textwidth]{1.jpg}
    \caption{Энерговыделение как функция глубины для электромагнитных ливней, образованных 1, 10, \mbox{100 ГэВ} и \mbox{1 ТэВ} электроном в медном поглотителе (а). Радиальное распределение энергии, оставленной \mbox{10 ГэВ} электроном в меди на различных глубинах (б). Результат симуляции в программе \textit{EGS4} [4]}
    \label{fig:enDep}
\end{figure}

Поперечное развитие ливня обусловлено множественным рассеянием электронов и позитронов, а так же изотропным и несоосным с ливнем характером рождения фотонов и позитронов.

Первый процесс преобладает на ранних стадиях развития ливня, а второй после того, как ливень преодолевает максимум. Влияние обоих процессов хорошо демонстрируется рис.  \ref{fig:enDep}(б), на котором изображена радиальная плотность энергии ливня, порожденного электроном в меди, на трех различных глубинах внутри калориметра.

Развитие ливня может быть описано более или менее независимо от материала поглотителя в терминах радиационной длины $X_0$ (в продольном направлении) и Мольеровского радиуса $\mathrm{\rho}_M$ (в поперечном направлении). Радиационная длина различается для фотонов и электронов достаточно существенно. Ливни, инициированные высокоэнергетичным  электроном или же фотоном, имеют существенные отличия. Попадая  в материал, высокоэнергетичные электроны начинают излучать немедленно. На их пути, через первые несколько миллиметров материала, они могут испускать тысячи фотонов тормозного излучения. С другой стороны, высокоэнергетичный фотон может как провзаимодействовать с веществом на такой же длине, так и не провзаимодействовать.  В последнем случае он не потеряет энергии совсем, а в случае взаимодействия, может потерять даже больше, чем электрон в таком же количестве материала. Такая разница иллюстрируется на рис. \ref{fig:gamAndEnFraq}. В одинаковом количестве материала электроны теряют большую долю своей энергии, чем фотоны, но разброс энергетических потерь у фотонов больше.

%%% gamma and electron en fraction (from Wigmans)
\begin{figure}[H]
    \centering
    \includegraphics[width=0.75\textwidth]{2.jpg}
    \caption{Распределение доли энергии, оставленной при прохождении пяти радиационных длин электроном и фотоном с энергиями \mbox{10 ГэВ} [4]}
    \label{fig:gamAndEnFraq}
\end{figure}

В первом приближении профиль электромагнитного ливня в основном определяется значениями $X_0$ и $\mathrm{\rho}_M$,  но  такое  приближение неидеально,  что демонстрируется на рис. \ref{fig:elEnDepMat},  где  показано  энерговыделение  на  единице радиационной длины. Такое отличие проще понять посредством того факта, что количество  вторичных  частиц  после  максимума  ливня  начинает уменьшаться,  и  это  снижение  медленнее  в  материалах с высоким {\textit Z}. Например,  высокоэнергетичный  электрон  порождает  в  свинце  в  три  раза больше  позитронов,  чем  в  алюминии.  В  результате  нужно  больше радиационных  длин  свинца,  чем  алюминия,  чтобы  уместить  в  себе  \mbox{99  \%} ливня. К тому же, максимум ливня располагается в материале с высокими {\textit Z} на большей глубине.

%%% energy deposited by electron in different materials (from Wigmans)
\begin{figure}[H]
    \centering
    \includegraphics[width=0.65\textwidth]{3.jpg}
    \caption{Продольные профили электромагнитных ливней, образованных электроном с энергией \mbox{10 ГэВ} в алюминии, железе и \mbox{свинце [4]}}
    \label{fig:elEnDepMat}
\end{figure}

%%% average shower fraction for different energies and materials (from Wigmans)
\begin{figure}[H]
    \centering
    \includegraphics[width=0.65\textwidth]{4.jpg}
    \caption{Средняя доля энергии, выделившейся в блоке материала в результате прохождения его частицей. Показаны результаты для ливней, образованных электронами различных энергий в медном поглотителе (а) и результаты для ливней, образованных электроном с энергией \mbox{100 ГэВ} в различных поглотителях (б). Результат симуляции в программе \textit{EGS4} [4]}
    \label{fig:fracEnAndMat}
\end{figure}

Зависимость  толщины  калориметра,  необходимой  для  того,  чтобы вместить   электромагнитный   ливень,   образованный   электроном,   для различных материалов представлена на рис. \ref{fig:fracEnAndMat}(б). При поглощении \mbox{99 \%} электромагнитного  ливня разница  между  материалом  с  высоким {\textit Z} и материалом  с  низким {\textit Z} может  достигать  десяти $X_0$. И  по  причинам, описанным  выше,  нужно  еще  больше  материала,  чтобы  поглотить  ливень, образованный фотоном. Зависимость доли поглощенной энергии от толщины калориметра изображена на рис. \ref{fig:fracEnAndMat}(а). Для поперечной локализации ливня энергетическая  зависимость  отсутствует,  а различия  в  материале  не  такие значительные,  как  для  продольного. Достаточно  длинный  цилиндр таким образом  поглотит  одинаковую  долю  энергии  от \mbox{1  ГэВ}  ливня  и  от \mbox{1  ТэВ} ливня.

Нарушения пропорциональности, рассмотренные на рис. \ref{fig:elEnDepMat} и \ref{fig:fracEnAndMat}, вызваны явлениями, которые происходят при энергиях ниже критической. Например, в  свинце  более  \mbox{40 \%}  энергии  ливня оставляется  в  веществе частицами  с энергиями ниже \mbox{1 МэВ}, в то время как критическая энергия имеет значение примерно \mbox{7 МэВ}. Только четверть энергии оказывается внесена позитронами, остальное  приносят  электроны.  Эти  факты получены  в  результате  Монте-Карло  симуляции  развития  ливня, и показывают,  что  комптоновское рассеяние  и  явление  образования  пар в  основном  ответственны  за формирование профиля ливня. Оба процесса преобладают на энергиях ниже критической,  а  значит,  недостаточно  точно  описываются  терминами радиационной длины $X_0$ и Мольеровского радиуса $\mathrm{\rho}_M$.

\newpage
\section{Функция отклика калориметра}  \label{chap2} 
\subsection{Абсолютный отклик и его отклонения} \label{chap2.1}

Определим  отклик  калориметра  как  средний  сигнал  калориметра  на единицу выделенной энергии. Таким образом, отклик может быть выражен в терминах числа фотоэлектронов на один ГэВ, в пикокулонах на МэВ и т.д. 

%%% average signal for wired chambers (from Wigmans)
\begin{figure}[!h]
    \centering
    \includegraphics[width=0.75\textwidth]{5.jpg}
    \caption{Средний сигнал гетерогенного калориметра с проволочными камерами, работающими в режиме «насыщенной лавины», на электромагнитный ливень как функция выделившейся энергии (а),сигнал для каждой из пяти секций (б) [4]}
    \label{fig:meanSign}
\end{figure}

В общем случае электромагнитные калориметры линейны только тогда, когда вся энергия пучка выделяется в результате процессов, которые могут производить  сигналы  (возбуждение  или  ионизация  в  слое  поглотителя). Нелинейность обычно свидетельствует об инструментальных проблемах, таких как  насыщение  сигнала и  ливневые  утечки.  На  рис. \ref{fig:meanSign} показан  отклик нелинейного  калориметра.  В  этом  детекторе  проволочные  камеры, использующиеся  для детектирования прохождении  частицы  из  ливня, работают  в  режиме  «насыщенной  лавины»,  это  значит,  что  камеры  не различают прохождение одной частицы и \textit{n} частиц. С повышением энергии ливня или плотности  частиц в ливне  эффект насыщения понижает отклик. Рис. \ref{fig:meanSign}(б), на котором представлена зависимость сигнала сегментированного на пять  секций  калориметра  от  числа  образовавшихся  в  каждой  секции позитронов, показывает,  что  не  столько  выделившаяся  энергия,  сколько плотность  образовавшихся  частиц  ответственна  за  подобные  эффекты,  так как  влияние  эффектов  наиболее  заметно  на  ранней  стадии  развития  ливня (секция  1),  при  которой ливень  еще  сильно  сколлимирован. Описанные эффекты можно избежать, используя камеры в пропорциональном режиме.

По своему устройству калориметры подразделяются на два класса:

\begin{enumerate}[wide]
\item Гомогенные калориметры, в которых поглотитель является также активным материалом, производящим сигнал.
\item Гетерогенные калориметры,  в  которых  каждый  материал выполняет свою функцию.
\end{enumerate}

В инструментах, относящихся ко второму классу, только доля энергии ливня выделяется в активном материале. Эта «сэмплирующая» доля обычно определяется на основании сигнала от минимально ионизирующей частицы (\textit{mip}). \textit{Mip} соответствует частице, удельные ионизирующие потери которой в веществах  поглотителя  и  активного  материала  будут  наименьшими. Например,  в  калориметре \textit{D0},  который  состоит  из  пластин  из $\mathrm{U^{238}}$, разделенных  промежутками  в  \mbox{4,6  мм},  заполненными жидким  аргоном, сэмплирующая   доля для \textit{mip} составляет \mbox{13,7   \%}. Однако для электромагнитных   ливней   сэмплирующая   доля   энергии   составляет приблизительно \mbox{8,2 \%}.

%%% e divided by mip as a function of plates thickness
\begin{figure}[!h]
    \centering
    \includegraphics[width=0.75\textwidth]{6.jpg}
    \caption{Отношение \textit{e/mip} как функция толщины слоёв поглотителей для калориметров уран/оргстекло и уран/жидкий аргон. Результат симуляции в программе \textit{EGS4} [4]}
    \label{fig:eDevMip}
\end{figure}

Причиной такого различия снова является тот факт, что основной вклад в сигнал от электромагнитного ливня вносят очень мягкие частицы. Гамма-кванты с энергиями ниже \mbox{1 МэВ} чрезвычайно неэффективно регистрируются в  детекторах этого класса, потому  что  наиболее  вероятное  взаимодействие при таких энергиях это фотоэффект. Сечение фотоэффекта пропорционально $Z^5$,  поэтому  практически  все  взаимодействия  мягких  гамма-квантов происходят  в  слоях  поглотителя,  и  вклад  в  сигнал  можно  ожидать  только если  взаимодействие  произошло  очень  близко  к  границе  поглотителя  с активным  слоем  (тогда  фотоэлектрон,  чей  пробег  меньше  \mbox{1  мм}, может, выбравшись  из  поглотителя, произвести  сигнал  в  жидком  аргоне). По причине  ключевой  роли  фотоэффекта,  его  влияние на  отношение \textit{e/mip} зависит от значений \textit{Z} пассивного и активного материалов (\textit{e/mip} наименьшее для  поглотителей  с  высоким \textit{Z} и  активных  материалов  с  малыми \textit{Z})  и  от толщины слоёв поглотителя (рис. \ref{fig:eDevMip}). Если поглотитель сделать достаточно тонким, то \textit{e/mip} становится практически равным 1.

\subsection{Флуктуации} \label{chap2.2}

Так  как  принцип  работы  калориметров  основан  на статистических процессах, точность калориметрических измерений определена и ограничена флуктуациями.   На   энергетическое   разрешение   электромагнитного калориметра влияет несколько флуктуационных процессов:

 \begin{itemize}[leftmargin=1.6\parindent, wide]
 	\item[---] квантовые   флуктуации   сигнала,   например   фотоэлектронная статистика;
 		\item[---] флуктуации ливневых утечек;
 			\item[---] флуктуации, обусловленные инструментальными эффектами, такими как  электронные  шумы,  ослабление  светового  потока  и  структурные неоднородности;
 				\item[---] флуктуации сэмплирования.
 \end{itemize}
 
 Последний процесс характерен только для гетерогенных калориметров. В хорошо спроектированных калориметрах этот вид флуктуаций преобладает над  остальными.  В  отличие  от  других  флуктуационных  процессов, флуктуации  сэмплирования  обусловлены  правилами  статистики  Пуассона. Таким  образом,  их  вклад  в  энергетическое  и  позиционное  разрешение описывается слагаемым, который пропорционален $1/\sqrt{E}:\sigma /E \sim E^{- 1/2}$.
 
 Флуктуации  сэмплирования  обусловлены  сэмплирующей  долей (или же   относительным   количеством   активного   материала)   и   частотой сэмплирования  (толщиной  слоёв).  В  электромагнитных  калориметрах  с  не газовым  активным  слоем  они  хорошо  описываются  следующей  формулой[4]:
 
 %%% Samplong fluctuations (from Wigmans)
\begin{equation}\label{eq:sampFluc}
\frac{\sigma}{E}=2,7\ \%\sqrt{ d/f_{sampl} }E^{-1/2},
\end{equation}
в которой \textit{d} – это толщина активного слоя (в мм), а $f_{sampl}$ – это сэмплирующая доля для \textit{mip}. Например, рассчитанная по формуле (\ref{eq:sampFluc}) составляющая энергетического разрешения для свинцово-сцинтилляционного калориметра \textit{KLOE} [5], соответствующая флуктуациям  сэмплирования, равна $6,9\ \%/\sqrt{E}$, что хорошо соответствует  экспериментально  найденному  значению  разрешения $5,7\ \%/\sqrt{E}$.

Среди  калориметров,  разрешение  которых  определяется  в  основном квантовыми  флуктуациями  сигналов,  можно  упомянуть  германиевые детекторы,  используемые  для  ядерной $\gamma$-спектрометрии,  и  калориметры  с кварцевым  волокном,  такие  как \textit{CMS} или \textit{HF}. Количество  энергии, необходимой для образования сигнала, в этих двух примерах различается на девять  порядков  по  величине.  В  то  время  как  достаточно  \mbox{1  эВ}  для образования электронно-дырочной пары в германиевом детекторе, светосбор в  кварцевом  калориметре  обычно  около  \mbox{1  ГэВ}  на  фотоэлектрон.  Таким образом   квантовые   флуктуации   сигнала   ограничивают   разрешение германиевых калориметров до величины 0,1 \% на МэВ, а кварцевых до \mbox{10 \% на ГэВ}.

Влияние    флуктуаций    ливневых    утечек    на    разрешение электромагнитного калориметра демонстрируется на рис. \ref{fig:leak}. Эти флуктуации имеют  не  Пуассоновский  характер  и  поэтому,  их  вклад  в  разрешение  не пропорционален $E^{-1/2}$. Оказывается  также,  что  для  заданного  положения ливня,  влияние  продольных  утечек  значительнее,  чем  поперечных.  Эти отличия объясняются различиями в числах различных частиц, ответственных за  утечки.  Например,  флуктуации  в  точке  начала  порожденного  фотоном ливня,  являются  флуктуациями  утечек,  за  которые  ответственна  только единственная  частица – инициирующий  фотон.  Боковые  утечки – это коллективный феномен, в который вносят вклад множество частиц.

В  отличие  от  продольных  и  поперечных  утечек,  третий  тип  утечек, альбедо,  или  же  обратные  утечки  через  тот  торец  детектора,  в  который попадает  инициирующая  частица,  не  может  быть  уменьшен  посредством изменения конструкции детектора. К счастью, эти утечки крайне невелики во всем  диапазоне  энергий,  кроме  очень  малых  значений. Результаты, показанные на рис. \ref{fig:leak}, получены путем Монте-Карло моделирования, но они подтверждаются многими экспериментами.

%%% lateral, lodgitudal and albedo fluctuations
\begin{figure}[H]
    \centering
    \includegraphics[width=0.75\textwidth]{7.jpg}
    \caption{Сравнение эффектов, вызванных ливневыми утечками различного рода. Результат симуляции в программе \textit{EGS4} [4]}
    \label{fig:leak}
\end{figure}

На практике разрешение конкретного калориметра определяется различными видами флуктуаций, каждый из которых имеет характерную энергетическую зависимость.  Обычно  эти  эффекты  некоррелированны,  значит  могут учитываться  в  виде  отдельных  слагаемых.  По  причине  различных зависимостей  от  энергии  общее  разрешение  калориметра  при  различных энергиях   может   быть   обусловлено   различными   флуктуационными эффектами.

%%% Fluctuations impact in the energy resolution
\begin{figure}[H]
    \centering
    \includegraphics[width=0.5\textwidth]{8.jpg}
    \caption{Энергетическое разрешение электромагнитного калориметра эксперимента \textit{ATLAS} [6]}
    \label{fig:resOnEn}
\end{figure}

Это  иллюстрируется  рис. \ref{fig:resOnEn},  на  котором  изображена  зависимость различных слагаемых, входящих в выражение энергетического разрешения, от  энергии для  электромагнитного  калориметра  эксперимента \textit{ATLAS}. Для энергий ниже \mbox{10 ГэВ} преобладающим фактором является электронный шум, между 10 и \mbox{100 ГэВ} – флуктуации сэмплирования и другие стохастические процессы,  при  энергиях  выше  \mbox{100  ГэВ}  независящие  от  энергии  эффекты влияют на энергетическое разрешение.

\subsection{Форма функции отклика} \label{chap2.3}

Не  все  виды  флуктуаций,  увеличивающие  отклонения  отклика, являются   симметричными   относительно   среднего   значения.   Ниже перечислены примеры эффектов, приводящих к несимметричным функциям отклика:

 \begin{itemize}[leftmargin=1.6\parindent, wide]
 	\item[---] Если сигнал сформирован очень маленьким числом частиц (например фотоэлектронами),  то  распределение  Пуассона  становится  ассиметричным. Эффекты  такого  рода  наблюдаются  в  сигналах  калориметров  с кварцевым оптическим волокном (рис. \ref{fig:poissons}).
 		\item[---] Эффекты ливневых утечек приводят к «хвостам» в распределениях сигналов.  Обычно,  такие  хвосты  возникают  в  распределении  сигналов  с низкоэнергетической стороны, потому что энергия покидает активный объем детектора.  Однако  существуют  примеры  детекторов,  в  которых  утечки приводят   к   усилению   сигнала.   Такое   явление   наблюдается   в сцинтилляционных  калориметрах,  в  которых  сцинтилляции  улавливаются кремниевыми диодами. Попадание в такой диод выбравшегося из детектора электрона может привести к тому, что величина сигнала окажется большей, чем от сцинтилляционного фотона (рис. \ref{fig:photPMT},а).
 \end{itemize}
 
 %%% Simmetric and asimmetric signals
\begin{figure}[H]
    \centering
    \includegraphics[width=1.0\textwidth]{9.jpg}
    \caption{Распределение сигналов для электромагнитных ливней, образованных электронами с энергией \mbox{10 ГэВ (а)} и \mbox{200 ГэВ (б)}, в калориметре эксперимента \textit{CMS} с кварцевыми оптическими волокнами [7]}
    \label{fig:poissons}
\end{figure}

 %%% Photodiode and PMT signals
\begin{figure}[H]
    \centering
    \includegraphics[width=1.0\textwidth]{10.jpg}
    \caption{Распределения сигналов от высокоэнергетичного электрона, образующего электромагнитный ливень в $\mathrm{PbWO}_{\mathrm{4}}$ кристальном калориметре. Для снятия сигнала используется кремниевый фотодиод (а) и ФЭУ (б) [8]}
    \label{fig:photPMT}
\end{figure}




\newpage
\section{Методики калибровки электромагнитных калориметров}  \label{chap3}  

Калибровка,  или  же  установление  отношения  между  выделившейся энергией  и  сигналом  калориметра  является,  возможно,  наиболее  важным аспектом  работы. Во  втором  разделе  было  упомянуто,  что  устройство калориметра  определяется  процессами,  происходящими  на  последних стадиях развития ливня. Эту особенность нужно учитывать при калибровке продольно сегментированных устройств. 

В  электромагнитном  ливне,  развивающемся  внутри гетерогенного калориметра, сэмплирующая доля для мягких гамма-квантов отличается от аналогичной  для \textit{mip}.  Поэтому  общая  сэмплирующая  доля  является  функцией глубины или возраста ливня. Это иллюстрируется на рис. \ref{fig:eDivMipDepth}. Этот эффект зависит не только от \textit{Z} активного и пассивного материалов, но также и от энергии ливня.

%%% e/mip as depth function (from Wigmans)
\begin{figure}[H]
    \centering
    \includegraphics[width=0.75\textwidth]{11.jpg}
    \caption{Отношение \textit{e/mip} как функция глубины ливня, образованного электроном с энергией \mbox{1 ГэВ} в гетерогенных электромагнитных калориметров различных конфигураций. Результат симуляции в программе \textit{EGS4} [4]}
    \label{fig:eDivMipDepth}
\end{figure}

Чем  ниже  энергия  ливня,  тем  раньше  преобладающими  станут  мягкие частицы  Комптоновского  рассеяния  и  фотоэффекта.  Если  калориметр продольно  сегментирован,  то  отношение между  выделившейся  энергией  и результирующим  сигналом  калориметра  будет  различным  для  разных сегментов.  Как  результат,  энергия,  выделившаяся  в  различных  сегментах, систематически  недооценивается (или  переоценивается),  причем  величина ошибки зависит  от  энергии  инициирующей  ливень  частицы. Это демонстрируется рис. \ref{fig:helCal}, на котором изображен размер просчетов для двух секций,  составляющих  калориметр \textit{HELIOS} [9]. Энергия  в  первой  секции (глубиной 6,6 $X_0$) систематически переоценивается, энергия во второй секции систематически недооценивается.

%%% energy fraction in the 1st and 2nd segments of HELIOS calorimeter
\begin{figure}[H]
    \centering
    \includegraphics[width=0.75\textwidth]{12.jpg}
    \caption{Ошибка при измерении энергии, выделившейся в конкретной секции продольно сегментированного калориметра \textit{HELIOS}, как функция энергии электронов, образующих ливень (нижняя ось), или доли энергии, выделившейся в первом сегменте (верхняя ось) [4]}
    \label{fig:helCal}
\end{figure}

На практике в подобных случаях принято определять калибровочные константы для этих двух секций и возникает вопрос о том, как это сделать. Практически все используемые методы не приводят к должному результату. Например,  для калибровки  упомянутого  выше  калориметра \textit{HELIOS}, используется  метод,в  котором  калибровочные  константы  выбираются  так, чтобы  минимизировать  ширину  сигнала,  соответствующего  следующей формуле [4]:

%%% HELIOS calibration condition
\begin{equation}\label{eq:helCalCond}
Q = \sum_{j=1}^N
\left[
E-A\sum_{i=1}^n S_{ij}^A
- B\sum_{i=1}^n S_{ij}^B
\right]^2,
\end{equation}

где \textit{A} – калибровочная  константа  для  первой  секции, \textit{B} – калибровочная константа для второй секции.

Однако выяснилось,  что  значения  констант \textit{A} и \textit{B}, а  особенно  их отношение  зависит  от  энергии  электронов,  которые  использовались  для калибровки детектора. Эта зависимость показана на рис. \ref{fig:helCal},а. В частности было установлено, что эти константы отличаются  от единицы, то есть обе секции  недокалиброваны,  относительно  мюонов,  которые  сэмплируются обеими  секциями  совершенно  одинаково.  Установлено,  что  такой  метод калибровки приводит к следующим нежелательным последствиям:

\begin{itemize}[leftmargin=1.6\parindent, wide]
\item[---] зависимость калибровочных констант от энергии;
\item[---] нелинейность электромагнитного отклика;
\item[---] систематические отличия 	в откликах электронов, гамма-квантов и пионов.
\end{itemize}

Это  приводит,  практически  во  всех  случаях,  к  зависимости реконструируемой  энергии  от  точки  начала  ливня  и  к  систематическим просчетам  при  измерении  энергии  джетов.  Далее  приведены  несколько методов, которые широко используются на практике для того, чтобы снизить влияние подобных нежелательных эффектов. 

Существует  множество  экспериментов,  в  которых  продольная сегментация  является  причиной проблем,  возникающих  при калибровке. В качестве примера можно упомянуть эксперимент \textit{AMS} [10]. Его свинцово-сцинтилляционный калориметр с оптическими волокнами по длине разделен на 18 сегментов, каждый толщиной примерно 1 $X_0$. Каждый из этих сегментов  калибруется  на \textit{mip} и  энергетический  эквивалент  прохождения одного \textit{mip} через сегмент был установлен равным \mbox{11,7 МэВ}. Как бы то ни было,  полной  длины  калориметра  (17 $X_0$)  недостаточно  для  того,  чтобы полностью поглотить энергию ливня, что демонстрируется на рис. \ref{fig:leakAMS}(а). Как результат, общий сигнал калориметра не пропорционален энергии пучка. Чем больше энергия пучка, тем больше утечки (рис. \ref{fig:leakAMS},б). Путем интегрирования функции,  аппроксимирующей  профиль  пучка,  от  нуля  до  бесконечности, оказывается возможным восстановить лишь часть потерянной энергии.

%%% AMS calibration with energy leakege correction
\begin{figure}[H]
    \centering
    \includegraphics[width=0.75\textwidth]{13.jpg}
    \caption{Средние сигналы, вызванные электроном с энергией \mbox{100 ГэВ} в 18 продольных секциях калориметра эксперимента \textit{AMS} (а). Усредненная разность между измеренной энергией и энергией пучка до и после поправок, учитывающих утечки (б) }
    \label{fig:leakAMS}
\end{figure}

Это объясняется тем, что сигналы, снятые с участков за максимумом ливня соответствуют  значительно  большей  энергии,  чем  сигналы  из сегментов,  в которых происходит раннее развитие ливня. При использовании одинаковых коэффициентов преобразования сигнала в энергию для всех участков модуля, поперечные утечки систематически будут недооцениваться. Таким образом, реконструированная  энергия  ливня  систематически  оказывается  слишком маленькой, особенно, если доля утечек велика. 

Коллаборацией эксперимента \textit{ATLAS} был разработан подход, который на  практике  приводит  к  достаточно  хорошим  результатам.  В  этом эксперименте  используется  калориметр  из  свинца  и  жидкого  аргона, состоящий  из  трех  продольных  сегментов  (толщиной 4,3 $X_0$,  16 $X_0$ и 2 $X_0$ соответственно).  Сэмплирующая  доля  энерговыделения  в  этом  детекторе значительно  понижается  с  ростом  длины,  несмотря  на  его  однородную структуру.  Калибровочные  коэффициенты  были  найдены  на  основании тщательной  Монте-Карло  симуляции  таким  образом,  чтобы  достичь одновременно   хорошего   разрешения   и   линейности   сигнала [11]. Реконструкция  энергии  по  измеренному  сигналу  производилась  на основании  формулы,  содержащей  по  крайней  мере  4  параметра,  которые нелинейно зависели от энергии электронов пучка. Благодаря этой формуле авторы  достигли линейности  в  интервале  энергий  \mbox{15-180 ГэВ}. В то же время, рассчитанные подобным образом коэффиценты корректны только для единственной псевдобыстроты, в то время как значения параметров должны отличаться в том случае, когда сигнал производят фотоны, а не электроны. Возникает  также  вопрос  о  том, как  экстраполировать  эти  результаты  для значений,  лежащих  за  границами  интервала  энергий,  в  котором  они  были получены.  Для  \mbox{10  ГэВ}  уже  наблюдались  серьезные  отклонения  от линейности сигнала. В  заключение  стоит  отметить,  что  калибровка  в  первую  очередь должна  приводить  к  корректной  реконструкции  энергии  порождающей ливень частицы. Это условие серьезно отличается от требований касательно ширины распределения сигналов, линейности сигнала или других желаемых особенностей, которые часто формируют основу методики калибровки [12, 13]. 


\newpage
\section{Параметризация электромагнитного ливня}  \label{chap4}   

Параметризация электромагнитного  ливня заключается  в  получении аналитической зависимости, связывающей пространственное энергетическое распределение   электромагнитного   ливня   с   координатами   частицы, инициирующей ливень при попадании в детектор, и энергией этой частицы. Необходимость   параметризации обычно обусловлена   желанием ускорить процесс симуляции без снижения точности однако, параметризация ливня  может  также  использоваться  для  уточнения  калибровочных коэффициентов. Параметризация в гомогенных калориметрах может использоваться как первое приближение параметризации ливня в моделях для гетерогенных калориметров. Пространственное  энергетическое  распределение  описывается  тремя функциями плотности вероятности:

%%% Spatial energy distrinution (Grindhammer)
\begin{equation}\label{eq:dE}
dE(\vec{r}) = Ef(t)dtf(r)drf(\varphi)d\varphi, 
\end{equation}
соответствующими   продольному,   радиальному   и   азимутальному распределениям  энергии. В  этой  формуле \textit{t} обозначает  продольную  длину ливня  в  единицах  радиационной  длины, \textit{r} – радиальное  расстояние  в Мольеровских  радиусах  от  оси  распространения  ливня, $\varphi$ -- азимутальный угол. Начало ливня определяется точкой в пространстве, в которой первый из электронов или позитронов испускает квант тормозного излучения. Известно, что средний продольный профиль электромагнитного ливня может описываться гамма-распределением [14]:

%%% Longitudal profile (Grindhammer)
\begin{equation}\label{eq:longProf}
\left< 
\frac{1}{E}\frac{dE(t)}{dt}
\right>
= f(t) 
= \frac{ (\beta t )^{ \alpha - 1 } \beta exp(-\beta t) }{\Gamma (\alpha) } 
\end{equation}
Центр тяжести $<t>$,  и  глубина  максимума  ливня \textit{T} могут  быть вычислены  через  параметр  формы $\mathrm{\alpha}$ и  параметр  масштаба $\mathrm{\beta}$ следующим образом

%%% Gravity center (Grindhammer)
\begin{equation}\label{eq:gravCenter}
<t> = \frac{\alpha}{\beta},
\end{equation}

%%% Scale parameter (Grindhammer)
\begin{equation}\label{eq:scalePar}
T = \frac{\alpha - 1}{\beta}.
\end{equation}
Множество различных функций, описывающих среднюю радиальную компоненту ливня

%%% Radial profile (Grindhammer)
\begin{equation}\label{eq:radProf}
f(r) = \frac{1}{dE(t)}\frac{dE(t,r)}{dr},
\end{equation}
могут  быть  найдены  в  различной специализированной литературе. В  этой работе   предлагается   использовать   следующую   двухкомпонентную подстановку, которая является расширением формулы, предложенной в [15]:

%%% Radial profile 2-component Ansatz (Grindhammer)
\begin{equation}\label{eq:ansatz}
f(r)
= \rho f_c(r)
+ (1 - \rho)f_T(r)
= \rho \frac{2rR_c^2}{ (r^2 + R_c^2)^2 }
+ (1 - \rho) \frac{ 2rR_T^2}{ (r^2 + R_T^2)^2 }
\end{equation}
где $0 \leq \rho \leq 1, R_c(R_T)$ -- медиана серединной (хвостовой) компоненты, $\rho$ -- вероятность, задающая серединный вес компоненты.

Азимутальное энергетическое распределение равномерно: $f(\varphi) = 1/{2\pi}$.

\newpage
 \section{Моделирование энергетического разрешения калориметра}  \label{chap5}  
 
 Моделирование электронного ливня в калориметре \textit{ECAL} производилось с использованием программного пакета \textit{GEANT4}. Ранее в научной группе ТПУ/\textit{NA64} была создана цифровая модель гетерогенного электромагнитного калориметра (рис. \ref{fig:ecal}), состоящего из чередующихся пластин из свинца, толщиной \mbox{1,5 мм}, и сцинтиллятора из полиметилметакрилата, толщиной \mbox{1,55 мм}. Геометрия калориметра и сцентарии моделирования настраивались через соотвествующие конфигурационные файлы. 
 
 %%% Geant4 ECAL model
\begin{figure}[H]
    \centering
    \includegraphics[width=0.75\textwidth]{ECAL-model.jpg}
    \caption{Модель калориметра \textit{ECAL}, \textit{GEANT4}}
    \label{fig:ecal}
\end{figure}

Для вычисления энергетического разрешения калориметра необходимо знать среднеквадратичное отклонение совокупного энерговыделения (высвеченная энергия) от среднего значения (квази-)моноэнергетического пучка, которое выражается как корень из суммы квадратов энерговыделения в ячейках:

%%% STD as sum of std
\begin{equation}\label{eq:sigmaCal}
\sigma = \sqrt{\sum_{i=1}^{36} \sigma_i^2} .
\end{equation}

В свою очередь выражение для среднеквадратичного отклонения энерговыделения в ячейке: 

%%% STD classic formula
\begin{equation}\label{eq:classicSTD}
\sigma_i = \sqrt{\frac{1}{n} \sum_{k=1}^n (x_k - \bar{x})^2},
\end{equation}
где \textit{n} -- число событий; $x_k$ -- энерговыделение в \textit{k}-м событии; $\bar{x}$ -- среднее энерговыделение в ячейке.

Выражение (\ref{eq:classicSTD}) путём простых преобразований сводится к 

%%% STD formula in use
\begin{equation}\label{eq:mySTD}
\sigma_i = \frac{1}{n} 
\left\{ 
\sum_{k=1}^n x_i^2 - \frac{1}{n} 
\left(
 \sum_{k=1}^n x_k \right) ^2 
\right\} .
\end{equation}

Проводилось моделирование попадания электрона в детектор для 500 событий в диапазоне энергий электрона от 10 до \mbox{100 ГэВ}. После чего по формуле (\ref{eq:mySTD}) вычислялось значение энергетического разрешения для заданной энергии. Полученная зависимость энергетического разрешения калориметра ECAL представлена на рис. \ref{fig:enResFit} звездочками.

%%% Energy resolution on the energy dependence
\begin{figure}[H]
    \centering
    \includegraphics[width=0.75\textwidth]{14.jpg}
    \caption{Зависимость энергетического разрешения электромагнитного калориметра \textit{ECAL} от энергии электрона, инициирующего ливень}
    \label{fig:enResFit}
\end{figure}

Зависимость энергетического разрешения сэмплирующего калориметра от энергии пучка можно описать так [16]:

%%% Energy resolution from delPeso
\begin{equation}\label{eq:delPeso}
\frac{\sigma_s}{E_s} = \frac{\sigma_0}{\sqrt{E}}
\left(
\frac{t}{X_t}
\right)
^{\alpha}
\left(
\frac{s}{X_s}
\right)
^{-\beta},
\end{equation}
где $X_t$ и $X_s$ -- радиационные длины поглотителя и сцинтиллятора соответственно, $\mathrm{\sigma_s}$ -- среднеквадратичное отклонение высвеченной энергии, $E_s$ -- высвеченная энергия, $\mathrm{\alpha}$, $\mathrm{\beta}$ и $\mathrm{\sigma_0}$ -- параметры аппроксимирующей функции. 
 
Однако, зависимость, изображенную на рис. \ref{fig:enResFit} удается аппроксимировать формулой (\ref{eq:delPeso}) с достаточной точностью только введя дополнительное постоянное слагаемое. Ключевой особенностью такого слагаемого является то что флуктуация не зависит от энергии ливня. Такая аппроксимация изображена на рис. \ref{fig:enResFit} красной кривой. Необходимость добавления постоянного слагаемого можно объяснить влиянием таких факторов как энергетические утечки, неустранимые флуктуации и ошибка вычисления высвеченной энергии, возникающая при суммировании множества маленьких значений.



\newpage  
\section{Моделирование энергетических утечек}  \label{chap6}  

Как было упомянуто в предыдущем разделе, одной из причин, обуславливающих необходимость добавления константного слагаемого к формуле (\ref{eq:delPeso}), могут являться неустранимые энергетические утечки. 

Увеличение геометрических размеров калориметра должно приводить к росту энергии, поглощенной в материале детектора. И, при достаточно больших размерах калориметра, ожидается достижение равенства между поглощенной энергией и энергией пучка, порождающего электромагнитный ливень. 

Было установлено, что при увеличении числа продольных сегментов от 150 до 1000 и увеличении радиуса детектора от \mbox{40 мм} до \mbox{200 мм} энергетические утечки в среднем уменьшились от \mbox{10 ГэВ} до \mbox{200 МэВ}. Однако, дальнейшее увеличение размеров детектора не приводило к снижению величины утечки. 

Геометрическая особенность калориметров типа "шашлык"{}, а именно -- чередование плоских пластин из материалов, физические свойства которых отличаются очень значительно, приводит к предположению о том, что гамма-кванты, попадая в сцинтиллятор под достаточно большими углами, относительно продольной оси калориметра могут, в результате многократного отражения от границы раздела сред сцинтиллятора и свинца, покидать детектор без поглощения (в особенности это касается мягких фотонов падающих на границу раздела под углом большим угла полного внутреннего отражения). 

Для проверки этой гипотезы были получены угловые распределения гамма-квантов, покидающих объем детектора для различных размеров детектора. Угловые распределения при энергии пучка \mbox{100 ГэВ} и \mbox{1000 продольных слоев} для малых и больших размеров детектора показаны на рис. \ref{fig:leakSmall}  и рис. \ref{fig:leakBig} соответственно.

 %%% Gamma leakege in small ECAl
\begin{figure}[H]
    \centering
    \includegraphics[width=0.5\textwidth]{leakage-small.png}
    \caption{Угловое распределение гамма-квантов, покидающих калориметр с радиусом \mbox{40 мм} (на графике указана ошибка, соответствующая одному $\sigma$)}
    \label{fig:leakSmall}
\end{figure}

 %%% Gamma leakege in big ECAl
\begin{figure}[H]
    \centering
    \includegraphics[width=0.5\textwidth]{leakage-big.png}
    \caption{Угловое распределение гамма-квантов, покидающих калориметр с радиусом \mbox{200 мм} (на графике указана ошибка, соответствующая одному $\sigma$)}
    \label{fig:leakBig}
\end{figure}

Можем заметить, что частицы действительно покидают калориметр под большими углами к продольной оси детектора, что подверждает догадку о многократном отражении внутри сцинтилляторных слоёв. Однако, среняя энергия, покидающая детектор, уменьшается более чем на два порядка при увеличении радиуса детектора от 40 до \mbox{200 мм}. При дальнейшем увеличении радиуса калориметра достигается полное отсутствие утечки гамма-квантов.

Таким образом, утечка гамма-квантов из калориметра не может приводить к независящим от энергии пучка флуктуациям. Единственной оставшейся причиной неустранимых утечек могут быть слабо взаимодействующие частицы, рождающиеся в фотоядерных реакциях, например -- нейтроны. 

\newpage
\section{Аппроксимация профиля ливня}  \label{chap7}  

В результате моделирования электромагнитного ливня были получены значения энерговыделения для каждого сегмента калориметра. После чего осуществлялась аппроксимация профиля ливня функциями (\ref{eq:longProf}) и (\ref{eq:ansatz}), используя процедуры библиотеки \textit{GSL} [17]. Аппроксимированные радиальные профили представлены на рис. \ref{fig:rad1}, \ref{fig:rad40} , \ref{fig:rad85}, продольный -- на рис. \ref{fig:longProf}.

 %%% Radial profile for the 1st cell
\begin{figure}[H]
    \centering
    \includegraphics[width=0.5\textwidth]{rad1.jpg}
    \caption{Радиальный профиль ливня в 1-й ячейке вдоль оси распространения}
    \label{fig:rad1}
\end{figure}

Радиальный профиль в самом начале ливня представляет собой резко убывающую кривую, напоминающую график экспоненты. Это объясняется тем, что в начале ливня число рожденных частиц невелико и образуются они близко к оси, вдоль которой происходит развитие ливня.

 %%% Radial profile for the 40 cell
\begin{figure}[H]
    \centering
    \includegraphics[width=0.5\textwidth]{rad40.jpg}
    \caption{ Радиальный профиль ливня в 40-й ячейке вдоль оси распространения (спадание у нуля является численным артефактом фитирующей функции)}
    \label{fig:rad40}
\end{figure}

Распределение энергии в области интенсивного энерговыделения становится шире и появляется максимум энерговыделения, соответствующий концу первой ячейки в направлении от центра детектора к периферии.

 %%% Radial profile for the 85 cell
\begin{figure}[H]
    \centering
    \includegraphics[width=0.5\textwidth]{rad85.jpg}
    \caption{Радиальный профиль ливня в 85-й ячейке вдоль оси распространения}
    \label{fig:rad85}
\end{figure}

Радиальное распределение энергии в области затухания ливня становится очень гладким со слабо выраженным максимумом во второй ячейке.

 %%% Longitudal profile
\begin{figure}[H]
    \centering
    \includegraphics[width=0.75\textwidth]{long.jpg}
    \caption{Продольный профиль ливня}
    \label{fig:longProf}
\end{figure}

 Трехмерный график энерговыделения в калориметре и аппроксимация энерговыеления факторизованной функцией показаны на рис. \ref{fig:shower3D}.

 %%% Electromagnetic shower profile with fit
\begin{figure}[H]
    \centering
    \includegraphics[width=0.75\textwidth]{shower3D.png}
    \caption{ Аппроксимация электромагнитного ливня факторизованной функцией }
    \label{fig:shower3D}
\end{figure}

Осуществили интегрирование функции, аппроксимирующей профиль ливня, в областях, ограничивающих  ячейку, на которую приходится ось развития ливня, ячейку, имеющую с центральной смежную грань, и ячейку, расположенную относительно центральной по диагонали, для предливневой и основной части калориметра ECAL. Значения, полученные в результате интегрирования, отнесенные к энергии пучка, представлены в таблице \ref{tab:enFrac}.

%%% Table with energy deposit fractions 
\begin{table}[H]
\centering
\caption{Энерговыделение в ячейках калориметра ECAL}
\label{tab:enFrac}
\begin{tabular}{|c|c|c|}
\hline
            & \multicolumn{2}{c|}{$E/E_0$, \%}        \\ \hline
Ячейка      & Предливневая часть & Основная часть \\ \hline
Центральная & 1,55               & 37,79          \\ \hline
Смежная     & 0,16               & 3,93           \\ \hline
Диагональная     & 0,09               & 2,27           \\ \hline
\end{tabular}
\end{table}

\newpage
\section{Извлечение калибровочных коэффициентов}  \label{chap8}

На основе набора данных калибровочной статистики, можно получить значения энерговыделения в калориметре в результате бомбардировки электронами каждой из 36 ячеек. После чего, осуществляя отбор по отдельным ячейкам, получили распределения энергии, поглощенной в каждой ячейке. Такое распределение для одной из центральных ячеек основной части калориметра показано на рис. \ref{fig:areaNoCat}.

%%% Raw energy distribution for 1-3-3
\begin{figure}[H]
    \centering
    \includegraphics[width=0.75\textwidth]{areaNoCat133Log.png}
    \caption{Распределение энерговыделения в одной из центральных ячеек осовной части калориметра}
    \label{fig:areaNoCat}
\end{figure}

В распределении выделяются три области:
\begin{enumerate}[wide]
\item Область низких значений, первый пик. Ей соответствуют события, в которых основное энерговыделение произошло в других ячейках, а сигнал в рассматриваемой ячейке обусловлен поглощением энергии достигших ее вторичных частиц.
\item Центральная область. Соответствует сигналу ячейки, через которую проходит ось распрстранения электромагнитного ливня при нормальном попадании пучка электронов в детектор. Исключительно значения из этой области должны учитываться при вычислении калибровочных коэффициентов.
\item Область высоких значений. Вклад в эту область вносят два или более попадания частицы в детектор, разнесенные во времени на достаточно малый промежуток, из-за чего они обрабатываются как одно событие. Такие сигналы следует дискриминировать как зашумляющие.
\end{enumerate}

Для того, чтобы обрабатывать события, соответствующие только центральному пику основного сигнала, был реализован алгоритм, поиска и отбора пиков, результат использования которого показан на рис. \ref{fig:areaCat}.
%%% Central peak from energy distribution for 1-3-3
\begin{figure}[H]
    \centering
    \includegraphics[width=0.75\textwidth]{areaCat133.png}
    \caption{Распределение энерговыделения в одной из центральных ячеек осовной части калориметра после дискриминации вторичных пиков}
    \label{fig:areaCat}
\end{figure}
После выделения центрального пика распределения, для каждой ячейки был получен интервал значений энерговыделения, подходящий для расчета калибровочных коэффициентов. 

Калибровочный коэффициент для каждой ячейки вычислялся как отношение долей энерговыделения, расчет которых описан в предыдущем разделе, к значению энерговыделения из распределения. Таким образом были получены распределения калибровочных коэффициентов (рис. \ref{fig:coeffDist}). Распределения коэффициентов аппроксимировались нормальным распределением, положение максимума которого принималось за усредненное значение коэффициента.

%%% Coefficients distribution for 1-3-3, fitted by gaussian
\begin{figure}[H]
    \centering
    \includegraphics[width=0.75\textwidth]{coef133.jpg}
    \caption{Распределение калибровочных коэффициентов для одной из центральных ячеек осовной части калориметра}
    \label{fig:coeffDist}
\end{figure}

Рассчитанные значение калибровочных коэффициентов для каждой ячейки предливневой и главной части электромагнитного калориметра ECAL представлены в таблице \ref{tab:calib-coeffs}.

%%% Calib-coefficient and RMS table
\begin{table}[H]
\footnotesize
\centering
\caption{Калибровочные коэффициенты электромагнитного калориметра ECAL}
\label{tab:calib-coeffs}
\begin{tabular}{|c|c|c|c|c|c|c|c|}
\hline
\multicolumn{2}{|c|}{\begin{tabular}[c]{@{}c@{}}Индексы \\ ячейки\end{tabular}} & \begin{tabular}[c]{@{}c@{}}Калибровочный\\ коэффициент\end{tabular} & \begin{tabular}[c]{@{}c@{}}Среднеквадратичное\\ отклонение\end{tabular} & \multicolumn{2}{c|}{\begin{tabular}[c]{@{}c@{}}Индексы\\  ячейки\end{tabular}} & \begin{tabular}[c]{@{}c@{}}Калибровочный\\ коэффициент\end{tabular} & \begin{tabular}[c]{@{}c@{}}Среднеквадратичное\\ отклонение\end{tabular} \\ \hline
\textit{x}                                      & \textit{y}                                      & \multicolumn{2}{c|}{Предливневая часть}                                                                                                       & \textit{x}                                      & \textit{y}                                     & \multicolumn{2}{c|}{Основная часть}                                                                                                           \\ \hline
0                                      & 0                                      & 0.00036                                                             & 0.00030                                                                 & 0                                      & 0                                     & 0.00412                                                             & 0.00048                                                                 \\ \hline
0                                      & 1                                      & 0.00035                                                             & 0.00028                                                                 & 0                                      & 1                                     & 0.00402                                                             & 0.00051                                                                 \\ \hline
0                                      & 2                                      & 0.00036                                                             & 0.00021                                                                 & 0                                      & 2                                     & 0.00394                                                             & 0.00046                                                                 \\ \hline
0                                      & 3                                      & 0.00030                                                             & 0.00018                                                                 & 0                                      & 3                                     & 0.00451                                                             & 0.00046                                                                 \\ \hline
0                                      & 4                                      & 0.00026                                                             & 0.00017                                                                 & 0                                      & 4                                     & 0.00435                                                             & 0.00046                                                                 \\ \hline
0                                      & 5                                      & 0.00029                                                             & 0.00019                                                                 & 0                                      & 5                                     & 0.00392                                                             & 0.00041                                                                 \\ \hline
1                                      & 0                                      & 0.00036                                                             & 0.00029                                                                 & 1                                      & 0                                     & 0.00379                                                             & 0.00046                                                                 \\ \hline
1                                      & 1                                      & 0.00032                                                             & 0.00027                                                                 & 1                                      & 1                                     & 0.00377                                                             & 0.00041                                                                 \\ \hline
1                                      & 2                                      & 0.00028                                                             & 0.00018                                                                 & 1                                      & 2                                     & 0.00459                                                             & 0.00043                                                                 \\ \hline
1                                      & 3                                      & 0.00032                                                             & 0.00020                                                                 & 1                                      & 3                                     & 0.00439                                                             & 0.00048                                                                 \\ \hline
1                                      & 4                                      & 0.00029                                                             & 0.00018                                                                 & 1                                      & 4                                     & 0.00401                                                             & 0.00038                                                                 \\ \hline
1                                      & 5                                      & 0.00025                                                             & 0.00017                                                                 & 1                                      & 5                                     & 0.00438                                                             & 0.00045                                                                 \\ \hline
2                                      & 0                                      & 0.01000                                                             & 0.00001                                                                 & 2                                      & 0                                     & 0.00389                                                             & 0.00050                                                                 \\ \hline
2                                      & 1                                      & 0.00037                                                             & 0.00030                                                                 & 2                                      & 1                                     & 0.00378                                                             & 0.00030                                                                 \\ \hline
2                                      & 2                                      & 0.00032                                                             & 0.00028                                                                 & 2                                      & 2                                     & 0.00405                                                             & 0.00038                                                                 \\ \hline
2                                      & 3                                      & 0.00031                                                             & 0.00019                                                                 & 2                                      & 3                                     & 0.00385                                                             & 0.00036                                                                 \\ \hline
2                                      & 4                                      & 0.00032                                                             & 0.00019                                                                 & 2                                      & 4                                     & 0.00384                                                             & 0.00038                                                                 \\ \hline
2                                      & 5                                      & 0.00029                                                             & 0.00025                                                                 & 2                                      & 5                                     & 0.00415                                                             & 0.00040                                                                 \\ \hline
3                                      & 0                                      & 0.00033                                                             & 0.00028                                                                 & 3                                      & 0                                     & 0.00462                                                             & 0.00051                                                                 \\ \hline
3                                      & 1                                      & 0.00030                                                             & 0.00025                                                                 & 3                                      & 1                                     & 0.00420                                                             & 0.00048                                                                 \\ \hline
3                                      & 2                                      & 0.00030                                                             & 0.00019                                                                 & 3                                      & 2                                     & 0.00402                                                             & 0.00044                                                                 \\ \hline
3                                      & 3                                      & 0.00029                                                             & 0.00019                                                                 & 3                                      & 3                                     & 0.00406                                                             & 0.00041                                                                 \\ \hline
3                                      & 4                                      & 0.00028                                                             & 0.00018                                                                 & 3                                      & 4                                     & 0.00418                                                             & 0.00038                                                                 \\ \hline
3                                      & 5                                      & 0.00039                                                             & 0.00030                                                                 & 3                                      & 5                                     & 0.00456                                                             & 0.00043                                                                 \\ \hline
4                                      & 0                                      & 0.01000                                                             & 0.00000                                                                 & 4                                      & 0                                     & 0.00399                                                             & 0.00049                                                                 \\ \hline
4                                      & 1                                      & 0.00043                                                             & 0.00032                                                                 & 4                                      & 1                                     & 0.00367                                                             & 0.00025                                                                 \\ \hline
4                                      & 2                                      & 0.00035                                                             & 0.00028                                                                 & 4                                      & 2                                     & 0.00398                                                             & 0.00048                                                                 \\ \hline
4                                      & 3                                      & 0.00028                                                             & 0.00023                                                                 & 4                                      & 3                                     & 0.00379                                                             & 0.00044                                                                 \\ \hline
4                                      & 4                                      & 0.00029                                                             & 0.00025                                                                 & 4                                      & 4                                     & 0.00421                                                             & 0.00039                                                                 \\ \hline
4                                      & 5                                      & 0.00027                                                             & 0.00017                                                                 & 4                                      & 5                                     & 0.00422                                                             & 0.00049                                                                 \\ \hline
5                                      & 0                                      & 0.00031                                                             & 0.00028                                                                 & 5                                      & 0                                     & 0.00364                                                             & 0.00027                                                                 \\ \hline
5                                      & 1                                      & 0.00032                                                             & 0.00027                                                                 & 5                                      & 1                                     & 0.00444                                                             & 0.00054                                                                 \\ \hline
5                                      & 2                                      & 0.00026                                                             & 0.00017                                                                 & 5                                      & 2                                     & 0.00423                                                             & 0.00047                                                                 \\ \hline
5                                      & 3                                      & 0.00030                                                             & 0.00025                                                                 & 5                                      & 3                                     & 0.00420                                                             & 0.00045                                                                 \\ \hline
5                                      & 4                                      & 0.00056                                                             & 0.00038                                                                 & 5                                      & 4                                     & 0.00415                                                             & 0.00046                                                                 \\ \hline
5                                      & 5                                      & 0.00032                                                             & 0.00026                                                                 & 5                                      & 5                                     & 0.00465                                                             & 0.00041                                                                 \\ \hline
\end{tabular}%
\end{table}

После применения полученных калибровочных коэффициентов к экспериментальным данным 2017 года, было получено распределение реконструированного энерговыделения в электромагнитном калориметре (рис. \ref{fig:overall}).

%%% ECAL overall energy distribution
\begin{figure}[H]
    \centering
    \includegraphics[width=0.75\textwidth]{overall.jpg}
    \caption{Распределение реконструированного энерговыделения}
    \label{fig:overall}
\end{figure}

Калибровочным моноэнергетическим событиям, не зашумленным попаданиями нескольких частиц во временной интервал записи АЦП, соответствует первый пик распределения, приходящийся на \mbox{40 ГэВ}. Второй и третий пик соответствуют \textit{pile-up} событиям от двух и трех частиц соответственно. Очевидно, что результат, при котором энерговыделение большинства событий от попадания электронного пучка с энергией \mbox{100 ГэВ} в электромагнитный калориметр оценивается в \mbox{40 ГэВ}, свидетельствует о недооценке.

Вероятные причины ошибки:
%%% Miscalibrations explanation
\begin{enumerate}[wide]
\item Неточность \textit{GEANT4} модели электромагнитного калориметра. 
\item Грубое приближение при аппроксимации профиля ливня формулой, полученной для гомогенного калориметра, приводящее к зависимости радиального профиля от глубины ливня.
\item Нелинейность преобразования энергии в веществе сцинтиллятора по отношению к энергии поглощённой в свинце.
\end{enumerate}

Последнюю причину следует разобрать более подробно. Сигнал калориметра образует преимущественно энергия, поглощенная в сцинтилляторе. Это значит, что для калибровки следовало бы использовать только энерговыделение в слоях полиметилметакрилата. При дальнейшем усовершенствовании процедуры калибровки предполагается исключать из рассмотрения энерговыделение в свинце путём введения дополнительного множителя в формулу, аппроксимирующую профиль ливня. Однако понимание этой методической ошибки приводит к некоторым интересным заключениям.

Отбирая события с наименьшим энерговыделением в адронном калориметре \textit{HCAL}, расположенном в телескопе детекторов позади калориметра \textit{ECAL} можно добиться того, что совокупная энергия, поглощенная в электромагнитном калориметре, будет соответствовать событиям с наибольшей герметичностью. Распределения энерговыделения с возрастающим ограничением по энергии, поглощенной в адронном калориметре, представлены на рис. \ref{fig:cutHCAL0}, \ref{fig:cutHCAL1}, \ref{fig:cutHCAL2} и \ref{fig:cutHCAL3}.

%%% No HCAl cut
\begin{figure}[H]
    \centering
    \includegraphics[width=0.75\textwidth]{HCALcut0.png}
    \caption{Распределение реконструированного энерговыделения (ограничение энергии, выделившейся в адронном калориметре 20000 вспом. ед.)}
    \label{fig:cutHCAL0}
\end{figure}

В распределении (рис. \ref{fig:cutHCAL0}) различимы области, соотвествующие аппаратурным шумам, одиночным событиям, двойным и тройным \textit{pile-up} событиям.

%%% The first cut - kill noize
\begin{figure}[H]
    \centering
    \includegraphics[width=0.75\textwidth]{HCALcut1.png}
    \caption{Распределение реконструированного энерговыделения (ограничение энергии, выделившейся в адронном калориметре 2000 вспом. ед.)}
    \label{fig:cutHCAL1}
\end{figure}

В распределении (рис. \ref{fig:cutHCAL1}) пропадает участок аппаратурного шума.

%%% The first cut - kill pileup (almost)
\begin{figure}[H]
    \centering
    \includegraphics[width=0.75\textwidth]{HCALcut2.png}
    \caption{Распределение реконструированного энерговыделения (ограничение энергии, выделившейся в адронном калориметре 400 вспом. ед.)}
    \label{fig:cutHCAL2}
\end{figure}

В распределении (рис. \ref{fig:cutHCAL2}) пропадает область, соответствующая тройным \textit{pile-up} событиям.

%%% The first cut - kill noize
\begin{figure}[H]
    \centering
    \includegraphics[width=0.75\textwidth]{HCALcut3.png}
    \caption{Распределение реконструированного энерговыделения (ограничение энергии, выделившейся в адронном калориметре 300 вспом. ед.)}
    \label{fig:cutHCAL3}
\end{figure}

В распределении (рис. \ref{fig:cutHCAL3}) практически не остается \textit{pile-up}-событий, центральный пик, приходящийся на \mbox{55 ГэВ}, -- основной сигнал. Можно отметить, что положение пика и его форма не изменяются после дискриминации зашумляющих событий. Такому поведению соответствует случай, в котором алгоритм калибровки выстроен верно, но значения коэффициентов определены с методологической ошибкой, обусловленной недооценкой влияния пространственной анизотропии преобразования энергии выделившейся в веществе калориметра в световую.

 \newpage
\section{Финансовый менеджмент, ресурсоэффективность и ресурсосбережение}  \label{ecochap:1}

\subsection{Оценка  коммерческого  потенциала  и  перспективности проведения научных исследований с позиции ресурсоэффективности и ресурсосбережения} \label{eco.1}

Калибровка детектора является одной из наиболее важных операций, предшествующих физическому эксперименту. Основной целью калибровки является  установление  соответствия  между  энергией,  выделившейся  в материале детектора при прохождении через него частицы или пучка частиц, и  сигналом  детектора.  Неточности  калибровки  могут  привести  к значительным ошибкам при измерении поглощенной энергии, что снижает достоверность результата эксперимента.

\subsubsection{Потенциальные потребители результатов исследования} \label{eco.1.1}

Единственными   потребителями   этой   научно-исследовательской работы   могут   быть   коллаборации   экспериментов,   использующих гетерогенные  электромагнитные  калориметры  схожего  устройства  с калориметром \textit{ECAL} эксперимента \textit{NA64}. Адаптировать  работу  под промышленность не представляется возможным.

\subsubsection{Анализ конкурентных технических решений} \label{eco.1.2}

Существует   несколько   методик   калибровки   гетерогенных калориметров,   применяющихся   в   различных   экспериментах.   Для предварительной  оценки  эффективности  научной  работы  был  проведен детальный  анализ  конкурентных  методик  калибровки. Такой  анализ позволяет оценить сильные и слабые стороны конкурирующих методик, и, если это необходимо, внести своевременные коррективы в рассматриваемую методику для поддержания ее конкурентоспособности. 

Анализ   конкурентных   технических   решений   проводился   с использованием оценочной карты, приведенной в таблице \ref{tab:eco1}. В этой таблице сравниваются    критерии    технической    ресурсоэффективности    и экономической  эффективности  различных  методик  калибровки.  Численное значение каждого критерия выбирается экспертным путем по пятибалльной шкале, где 1 –наиболее низкое значение, а 5 –наиболее высокое. Значения весов  критериев  определяются  экспертным  путем  так,  чтобы  их  сумма равнялась 1. Значения критериев трудоемкости и технической сложности тем выше, чем проще конкретный метод в реализации.

%%% Критерии трудоемкости
\begin{enumerate}[wide]
\item $\textnormal{Б}_\textnormal{ф}$ -- калибровка с использованием параметризации электромагнитного ливня.
\item $\textnormal{Б}_{\textnormal{к}1}$ -- калибровка методом минимизации разности между энергией пучка и суммой реконструированного энерговыделения в каждом сегменте.
\item $\textnormal{Б}_{\textnormal{к}2}$ -- индивидуальная калибровка каждого сегмента. 
\end{enumerate}

% %%
\begin{table}[H]
\small
\centering
\caption{Оценочная карта для сравнения конкурентных технческих разработок}
\label{tab:eco1}
\begin{tabular}{|l|c|c|c|c|c|c|c|}
\hline
\multicolumn{1}{|c|}{\multirow{2}{*}{\textbf{Критерии оценки}}}                  & \multicolumn{1}{|c|}{\multirow{2}{*}{\textbf{Вес критерия}}} & \multicolumn{3}{|c|}{\textbf{Баллы}}                                           & \multicolumn{3}{|c|}{\textbf{Конкурентоспособность}}                           \\ \cline{3-8} 
\multicolumn{1}{|c|}{}                                                           & \multicolumn{1}{|c|}{}                                       & \multicolumn{1}{|c|}{$\textnormal{Б}_\textnormal{ф}$} & \multicolumn{1}{|c|}{$\textnormal{Б}_{\textnormal{к}1}$} & \multicolumn{1}{|c|}{$\textnormal{Б}_{\textnormal{к}2}$} & \multicolumn{1}{|c|}{$\textnormal{К}_\textnormal{ф}$} & \multicolumn{1}{|c|}{$\textnormal{К}_{\textnormal{к}1}$} & \multicolumn{1}{|c|}{$\textnormal{К}_{\textnormal{к}2}$} \\ \hline
\multicolumn{8}{|c|}{\textbf{Технические критерии методики калибровки}}                                                                                                                                                                                                                                        \\ \hline
1. Точность калибровки                                                           & 0,3                                                         & 4                       & 2                        & 3                        & 1,2                     & 0,6                      & 0,9                      \\ \hline
2. Трудоемкость                                                                  & 0,1                                                         & 4                       & 2                        & 1                        & 0,4                     & 0,2                      & 0,1                      \\ \hline
3. Универсальность                                                               & 0,2                                                         & 5                       & 1                        & 2                        & 1                       & 0,2                      & 0,4                      \\ \hline
\begin{tabular}[c]{@{}l@{}}4. Техническая \\ сложность\end{tabular}              & 0,15                                                        & 5                       & 2                        & 1                        & 0,75                    & 0,3                      & 0,15                     \\ \hline
\multicolumn{8}{|c|}{\textbf{Экономические критерии оценки эффективности}}                                                                                                                                                                                                                                     \\ \hline
\begin{tabular}[c]{@{}l@{}}5. Финансирование научной \\ разработки\end{tabular}  & 0,1                                                         & 1                       & 5                        & 5                        & 0,1                     & 0,5                      & 0,5                      \\ \hline
\begin{tabular}[c]{@{}l@{}}6. Стоимость осуществления \\ калибровки\end{tabular} & 0,15                                                        & 1                       & 5                        & 5                        & 0,15                    & 0,75                     & 0,75                     \\ \hline
\textbf{Итого:}                                                                  & \textbf{1}                                                  & \textbf{}               & \textbf{}                & \textbf{}                & \textbf{3,60}           & \textbf{2,55}            & \textbf{2,80}            \\ \hline
\end{tabular}%
\end{table}

По результатам проведенного анализа можно сделать заключение, что метод калибровки с использованием Монте-Карло модели и параметризации ливня  превосходит  конкурирующие  методы.  Причиной  этому  служат следующие особенности:

%%% Отличительные особенности методики
\begin{enumerate}[wide]
\item Тщательное моделирование приводит к точным результатам, не зависящим от инструментальных особенностей;
\item использование  модели  позволяет  путем  небольших  изменений получить  значения  калибровочных  коэффициентов  для  различных  частиц, инициирующих ливень;
\item использование   модели   осуществляется   без   проведения экспериментов,  что  приводит  к  высокой  экономической  эффективности методики и малым трудозатратам.
\end{enumerate}

\subsubsection{\textit{SWOT}-анализ} \label{eco.1.3}

\textit{SWOT} – \textit{Strengths} (сильные стороны), \textit{Weaknesses} (слабые стороны), \textit{Opportunities} (возможности)  и \textit{Threats}  (угрозы) -- комплексный  анализ научно-исследовательского проекта, проводящийся в несколько этапов. Результаты   первого   этапа \textit{SWOT}-анализа,   заключающегося   в выявлении сильных и слабых сторон проекта, возможностей его развития и угроз,представлены в таблице \ref{tab:eco2}.

%%% Первый этап свотанализа
\begin{table}[H]
\centering
\caption{Первый этап \textit{SWOT}-анализа}
\label{tab:eco2}
\begin{tabular}{|l|l|}
\hline
\textbf{Сильные стороны}                                                                                                                                                                                         & \textbf{Слабые стороны}                                                                                                                                                                              \\ \hline
\begin{tabular}[c]{@{}l@{}}1. Отсутствие необходимости \\ проведения эксперимента.\\ 2. Высокая точность.\end{tabular}                                                                                           & \begin{tabular}[c]{@{}l@{}}1. Остутствие возможности \\ учета некоторых технических\\ особенностей.\\ 2. Неустранимая погрешность \\ моделирования.\end{tabular}                                      \\ \hline
\textbf{Возможности}                                                                                                                                                                                             & \textbf{Угрозы}                                                                                                                                                                                       \\ \hline
\begin{tabular}[c]{@{}l@{}}1. Калибровка электромагнитных\\ калориметров с различными\\  геометриями.\\ 2. Возможность калибровки для \\ различных частиц, инициирующих \\ электромагнитный ливень.\end{tabular} & \begin{tabular}[c]{@{}l@{}}1. Получение калибровок другими\\ методами.\\ 2. Невостребованность \\ исследрваний в данном направлении\\ из-за появления новых\\ экспериментальных фактов.\end{tabular}  \\ \hline
\end{tabular}%
\end{table}

Второй  этап  заключается  в  построении  интерактивных  матриц возможностей  и  угроз,  позволяющих  оценить  эффективность  проекта,  а также  надежность  его  реализации,  на  основании  матрицы SWOT. Соотношения параметров представлены в таблицах \ref{tab:eco3},\ref{tab:eco4},\ref{tab:eco5} и \ref{tab:eco6}.

%%% Сильные стороны и возможности
\begin{table}[H]
\centering
\caption{Интерактивная оценка проекта "Сильные стороны и возможности"}
\label{tab:eco3}
\begin{tabular}{|c|c|c|c|}
\hline
\multicolumn{4}{|c|}{Сильные стороны}                                                          \\ \hline
\multirow{3}{*}{\begin{tabular}[c]{@{}c@{}}Возможности \\ проекта\end{tabular}} &    & С1 & С2 \\ \cline{2-4} 
                                                                                & В1 & +  & -  \\ \cline{2-4} 
                                                                                & В2 & +  & +  \\ \hline
\end{tabular}%
\end{table}

%%% Слабые стороны и возможности
\begin{table}[H]
\centering
\caption{Интерактивная оценка проекта "Слабые стороны и возможности"}
\label{tab:eco4}
\begin{tabular}{|c|c|c|c|}
\hline
\multicolumn{4}{|c|}{Слабые стороны}                                                             \\ \hline
\multirow{3}{*}{\begin{tabular}[c]{@{}c@{}}Возможности \\ проекта\end{tabular}} &    & Сл1 & Сл2 \\ \cline{2-4} 
                                                                                & В1 & +   & -   \\ \cline{2-4} 
                                                                                & В2 & +   & -   \\ \hline
\end{tabular}%
\end{table}

%%% Сильные стороны и угрозы
\begin{table}[H]
\centering
\caption{Интерактивная оценка проекта "Сильные стороны и угрозы"}
\label{tab:eco5}
\begin{tabular}{|c|c|c|c|}
\hline
\multicolumn{4}{|c|}{Сильные стороны}   \\ \hline
\multirow{3}{*}{Угрозы} &    & С1 & С2 \\ \cline{2-4} 
                        & У1 & +  & -  \\ \cline{2-4} 
                        & У2 & +  & -  \\ \hline
\end{tabular}%
\end{table}

%%% Слабые стороны и угрозы
\begin{table}[H]
\centering
\caption{Интерактивная оценка проекта "Слабые стороны и угрозы"}
\label{tab:eco6}
\begin{tabular}{|c|c|c|c|}
\hline
\multicolumn{4}{|c|}{Слабые стороны}   \\ \hline
\multirow{3}{*}{Угрозы} &    & С1 & С2 \\ \cline{2-4} 
                        & У1 & -  & -  \\ \cline{2-4} 
                        & У2 & +  & -  \\ \hline
\end{tabular}%
\end{table}

Таким  образом,  в  рамках  третьего  этапа была составлена итоговая матрица SWOT-анализа (таблица \ref{tab:eco7}).

% Итоговая матрица свотанализа
\begin{table}[H]
\scriptsize
\centering
\caption{Итоговая матрица \textit{SWOT}-анализа}
\label{tab:eco7}
\begin{tabular}{|l|l|l|}
\hline
 &
  \begin{tabular}[t]{@{}l@{}}\textbf{Сильные стороны научно-}\\  \textbf{исследовательского проекта}\\ 1. Отсутствие необходимости\\ проведения эксперимента.\\ 2. Высокая точность\end{tabular} &
  \begin{tabular}[t]{@{}l@{}}\textbf{Слабые стороны научно-}\\ \textbf{исследовательского проекта}\\ 1. Отсутствие \\ возможности учета\\ некоторых технических\\ особенностей.\\ 2. Неустранимая \\ погрешность моделирования.\end{tabular} \\ \hline
\begin{tabular}[t]{@{}l@{}}\textbf{Возможности:}\\ 1. Калибровка электромагнитных\\ калориметров с различными\\ геометриями.\\ 2. Возможность\\ калибровки для различных\\ частиц, инициирующих\\ электромагнитный ливень.\end{tabular} &
  \begin{tabular}[t]{@{}l@{}}Отсутствие необходимости \\ в эксперименте позволяет \\ путем небольших изменений \\ применить методику \\ калибровки в различных \\ экспериментах с разными \\ установками и пучками \\ частиц\end{tabular} &
  \begin{tabular}[t]{@{}l@{}}Увеличение сложности \\ модели позволить \\ расширить число ее \\ возможных применений\end{tabular} \\ \hline
\begin{tabular}[t]{@{}l@{}}\textbf{Угрозы:}\\ 1. Получение \\ калибровок другими методами.\\ 2. Невостребованность\\ исследований в данном \\ направлении из-за появления \\ новых экспериментальных\\ фактов.\end{tabular} &
  \begin{tabular}[t]{@{}l@{}}Отсутствие необходимости в эксперименте позволяет \\ предлагать методику \\ калибровки в эксперименты, \\ детекторы которых ранее \\ были откалиброваны \\ недостаточно точно\end{tabular} &
  \begin{tabular}[t]{@{}l@{}}При серьезных изменениях в \\ технологии проведения \\ эксперимента \\ приспособление  такой \\ модели к новым \\ техническим особенностям \\ может оказаться \\ нецелесообразным или даже \\ невозможным\end{tabular} \\ \hline
\end{tabular}%
\end{table}

В результате \textit{SWOT}-анализа показана перспективность работы в виду ее  универсальности  и  отсутствия  привязки  к  конкретному  эксперименту. Наиболее  значимая  уязвимость  заключается  в  том,  что  в  большинстве экспериментов    используются    методы    калибровки,    основанные непосредственно   на   физических   измерениях,   которые   стали   уже классическими.

\subsection{Планирование научно-исследовательских работ} \label{eco.2}

\subsubsection{Структура работ в рамках научного исследования} \label{eco.2.1}

Планирование  призвано  обеспечить  рациональное  использование времени  и  при  формировании  научно-исследовательской  работы  является, несомненно,  важным  этапом.  Планирование  комплекса  предполагаемых работ осуществляется в следующем порядке:

 \begin{itemize}[leftmargin=1.6\parindent, wide]
 	\item[---] определение структуры работ в рамках научного исследования;
 		\item[---] определение участников каждой работы;
 			\item[---] установление продолжительности работ;
 				\item[---] построение графика проведения научных исследований.
 \end{itemize}
 
Для выполнения научных исследований формируется рабочая группа, в состав  которой  могут  входить  научные  сотрудники  и  преподаватели, инженеры, техники и лаборанты, численность групп может варьироваться от 3  до  15  человек.  В  рамках  данной  работы  была  сформирована  рабочая группа, в состав которой вошли: научный руководитель истудент-бакалавр. В  данном  разделе  был  составлен  перечень  этапов  и  работ  по  выполнению НИР, который представлен в таблице \ref{tab:eco8}.

%%% Этапы и исполнители
\begin{table}[H]
\small
\centering
\caption{Перечень этапов, работ и распределение исполниелей}
\label{tab:eco8}
\begin{tabular}{|c|c|c|c|}
\hline
Основные этапы                                                                            & \begin{tabular}[c]{@{}c@{}}№\\ раб\end{tabular} & Содержание работы                                                                        & \begin{tabular}[c]{@{}c@{}}Должность\\ исполнителя\end{tabular}            \\ \hline
\begin{tabular}[c]{@{}c@{}}Разработка \\ технического задания\end{tabular}                & 1                                               & \begin{tabular}[c]{@{}c@{}}Составление и утверждение\\ технического задания\end{tabular} & \begin{tabular}[c]{@{}c@{}}Научный\\ руководитель,\\ бакалавр\end{tabular} \\ \hline
\multirow{4}{*}{\begin{tabular}[c]{@{}c@{}}Выбор направления\\ исследований\end{tabular}} & 2                                               & Выбор направления исследований                                                           & \begin{tabular}[c]{@{}c@{}}Научный\\ руководитель,\\ бакалавр\end{tabular} \\ \cline{2-4} 
                                                                                          & 3                                               & \begin{tabular}[c]{@{}c@{}}Побор и изучение материалов по\\ теме\end{tabular}            & \begin{tabular}[c]{@{}c@{}}Научный\\ руководитель,\\ бакалавр\end{tabular} \\ \cline{2-4} 
                                                                                          & 4                                               & \begin{tabular}[c]{@{}c@{}}Разработка методики выполнения \\ работ\end{tabular}          & \begin{tabular}[c]{@{}c@{}}Научный \\ руководитель\end{tabular}            \\ \cline{2-4} 
                                                                                          & 5                                               & Составление календарного плана                                                           & \begin{tabular}[c]{@{}c@{}}Научный\\ руководитель,\\ бакалавр\end{tabular} \\ \hline
\multirow{2}{*}{\begin{tabular}[c]{@{}c@{}}Теоретическое \\ исследование\end{tabular}}    & 6                                               & Поиск литературы                                                                         & \begin{tabular}[c]{@{}c@{}}Научный\\ руководитель,\\ бакалавр\end{tabular} \\ \cline{2-4} 
                                                                                          & 7                                               & Изучение литературы                                                                      & Бакалавр                                                                   \\ \hline
\multirow{3}{*}{Практическая часть}                                                       & 8                                               & \begin{tabular}[c]{@{}c@{}}Моделирование электромагнитного\\ ливня\end{tabular}          & Бакалавр                                                                   \\ \cline{2-4} 
                                                                                          & 9                                               & \begin{tabular}[c]{@{}c@{}}Расчет энергетического разрешения\\ калориметра\end{tabular}  & Бакалавр                                                                   \\ \cline{2-4} 
                                                                                          & 10                                              & Калибровка калориметра                                                                   & Бакалавр                                                                   \\ \hline
\begin{tabular}[c]{@{}c@{}}Обобщение и оценка \\ результатов\end{tabular}                 & 11                                              & \begin{tabular}[c]{@{}c@{}}Оценка эффективности полученных\\ результатов\end{tabular}    & \begin{tabular}[c]{@{}c@{}}Научный\\ руководитель,\\ бакалавр\end{tabular} \\ \hline
\multicolumn{4}{|c|}{Провередие ВКР}                                                                                                                                                                                                                                                                                \\ \hline
\begin{tabular}[c]{@{}c@{}}Оформление комплекта\\ документации по ВКР\end{tabular}        & 12                                              & Составление отчета                                                                       & Бакалавр                                                                   \\ \hline
\end{tabular}%
\end{table}

\subsubsection{Определение трудоемкости выполнения работ} \label{eco.2.2}

Трудоемкость  выполнения  научного  исследования  оценивается экспертным  путем  в  человеко-днях  и  носит  вероятностный  характер,  т.к. зависит  от  множества  трудно  учитываемых  факторов.  Для  определения ожидаемого (среднего) значения трудоемкости $t_{\textnormal{ож}i}$ используется следующая формула:

%%% Среднее значение трудоемкости
\begin{equation}\label{eq:eq-eco1}
t_{\textnormal{ож}i} 
= \frac{3t_{mini}+2t_{maxi}}{5},
\end{equation}
где $t_{\textnormal{ож}i}$ -- ожидаемая трудоемкость выполнения \textit{i}-ой работы чел.-дн.; $t_{mini}$ -- минимально возможная трудоемкость выполнения заданной \textit{i}-ой работы (оптимистическая оценка: в предположении наиболее благоприятного стечения обстоятельств), чел.-дн.; $t_{maxi}$ -- максимально возможная трудоемкость выполнения заданной \textit{i}-ой работы (пессимистическая   оценка:   в   предположении   наиболее неблагоприятного стечения обстоятельств), чел.-дн.

Исходя   из   ожидаемой   трудоемкости   работ,   определяется продолжительность  каждой  работы  в  рабочих  днях  $Т_р$,  учитывающая параллельность выполнения работ несколькими исполнителями:

%%% Продолжительность работы в рабочих днях
\begin{equation}\label{eq:eq-eco2}
T_{pi}
= \frac{t_{\textnormal{ож}i}}
{\textnormal{Ч}i},
\end{equation}
где  $T_{pi}$ -- продолжительность одной работы, раб.дн.; $\textnormal{Ч}_i$ -- численность исполнителей, выполняющих одновременно одну иту же работу на данном этапе, чел.

\subsubsection{Разработка графика проведения научного исследования} \label{eco.2.3}

В  соответствии  с  календарным  планом  выполнения  работ  был построен  ленточный  график  выполнения  дипломной  работы  в  форме диаграммы Ганта.

Для  удобства  построения  графика,  длительность  каждого  из  этапов работ  из  рабочих  дней  следует  перевести  в  календарные  дни.  Для  этого необходимо воспользоваться следующей формулой:

%%% Перевод рабочих дней в календарные
\begin{equation}\label{eq:eq-eco3}
T_{\textnormal{к}i} = k_{\textnormal{кал}} \cdot T_{pi},
\end{equation}
где $T_{\textnormal{к}i}$ -- продолжительность  выполнения \textit{i}-й  работы  в  календарных днях; $k_{\textnormal{кал}}$ -- коэффициент календарности.

Коэффициент   календарности   определяется   по   следующей формуле:

%%% Коэффициент календарности
\begin{equation}\label{eq:eq-eco4}
k_{\textnormal{кал}} 
= \frac{T_{\textnormal{кал}}}
{T_{\textnormal{кал}}-T_{\textnormal{вых}}-T_{\textnormal{пр}}},
\end{equation}
где $T_{\textnormal{кал}}$ -- количество календарных дней в году; $T_{\textnormal{вых}}$ -- количество выходных дней в году; $T_{\textnormal{пр}}$ -- количество праздничных дней в году.

Таким образом:

%%% Расчет коэффициента календарности
\begin{gather*}
k_{\textnormal{кал}} 
= \frac{T_{\textnormal{кал}}}
{T_{\textnormal{кал}}-T_{\textnormal{вых}}-T_{\textnormal{пр}}}
= \frac{365}{365-14-104} = 1,48.
\end{gather*}

Все рассчитанные значения необходимо свести в таблицу \ref{tab:eco9}.

%%% Временные показатели
\begin{table}[H]
\small
\centering
\caption{Временные показатели проведения научного исследования}
\label{tab:eco9}
\begin{tabular}{|l|l|c|c|c|c|c|c|c|}
\hline
\multicolumn{1}{|c|}{№} & \multicolumn{1}{c|}{\begin{tabular}[c]{@{}c@{}}Содержание\\ работы\end{tabular}}                & Исполнитель                                                       & $t_{min}$ & $t_{max}$ & $t_{\textnormal{ож}}$ & Ч & $\textnormal{Т}_\textnormal{п}$  & $\textnormal{Т}_\textnormal{к}$   \\ \hline
1                       & \begin{tabular}[c]{@{}l@{}}Составление и \\ утверждение \\ технического \\ задания\end{tabular} & \begin{tabular}[c]{@{}c@{}}Руководитель\\ , бакалавр\end{tabular} & 2         & 4         & 2,8      & 2 & 0,7 & 1,0  \\ \hline
2                       & \begin{tabular}[c]{@{}l@{}}Выбор \\ направления \\ исследований\end{tabular}                    & \begin{tabular}[c]{@{}c@{}}Руководитель\\ , бакалавр\end{tabular} & 2         & 3         & 2,4      & 2 & 1,2 & 1,8  \\ \hline
3                       & \begin{tabular}[c]{@{}l@{}}Подбор и изучение \\ материалов по \\ теме\end{tabular}              & \begin{tabular}[c]{@{}c@{}}Руководитель\\ , бакалавр\end{tabular} & 5         & 7         & 5,8      & 2 & 2,9 & 4,3  \\ \hline
4                       & \begin{tabular}[c]{@{}l@{}}Разработка \\ методики \\ выполнения работ\end{tabular}              & \begin{tabular}[c]{@{}c@{}}Руководитель\\ , бакалавр\end{tabular} & 2         & 3         & 2,4      & 2 & 1,2 & 1,8  \\ \hline
5                       & \begin{tabular}[c]{@{}l@{}}Составление \\ календарного \\ плана\end{tabular}                    & Руководитель                                                      & 1         & 2         & 1,4      & 1 & 1,4 & 2,1  \\ \hline
6                       & Поиск литературы                                                                                & \begin{tabular}[c]{@{}c@{}}Руководитель\\ , бакалавр\end{tabular} & 2         & 3         & 2,4      & 2 & 1,2 & 1,8  \\ \hline
7                       & \begin{tabular}[c]{@{}l@{}}Изучение \\ литературы\end{tabular}                                  & Бакалавр                                                          & 7         & 14        & 9,8      & 2 & 4,9 & 7,3  \\ \hline
8                       & \begin{tabular}[c]{@{}l@{}}Моделирование \\ электромагнитного\\ ливня\end{tabular}             & Бакалавр                                                          & 2         & 4         & 4,2      & 1 & 2,8 & 4,1  \\ \hline
9                       & \begin{tabular}[c]{@{}l@{}}Расчет \\ энергетического \\ разрешения \\ калориметра\end{tabular}  & Бакалавр                                                          & 2         & 4         & 4,2      & 1 & 2,8 & 4,1  \\ \hline
10                      & \begin{tabular}[c]{@{}l@{}}Калибровка \\ калориметра\end{tabular}                               & Бакалавр                                                          & 3         & 6         & 7        & 1 & 3,2 & 4,7  \\ \hline
11                      & \begin{tabular}[c]{@{}l@{}}Оценка \\ эффективности \\ полученных \\ результатов\end{tabular}    & Руководитель                                                      & 2         & 3         & 2,4      & 1 & 2,4 & 3,6  \\ \hline
12                      & \begin{tabular}[c]{@{}l@{}}Составление \\ отчета\end{tabular}                                   & Бакалавр                                                          & 7         & 14        & 9,8      & 1 & 8,8 & 14,5 \\ \hline
\end{tabular}%
\end{table}

На  основании полученных данных был  построен  план-график  в  виде диаграммы  Ганта.  График  строится  с  разбивкой  по  месяцам  и неделям (7дней) за период времени дипломирования.

 %%% Диаграмма Ганта
\begin{figure}[H]
    \centering
    \includegraphics[width=0.75\textwidth]{GANT.png}
    \caption{Диаграмма Ганта}
    \label{fig:gant}
\end{figure}

\subsection{Бюджет научно-технического исследования (НТИ)} \label{eco.3}

При  планировании  бюджета  НТИ  должно  быть  обеспечено  полное  и достоверное отражение всех видов расходов, связанных с его выполнением. В  процессе  формирования  бюджета  НТИ  используется  следующая группировка затрат по статьям:

%%% Составные бюджета
\begin{enumerate}[wide]
\item материальные затраты НТИ;
\item затраты на основное оборудование для научно-экспериментальных работ;
\item основная заработная плата исполнителей темы;
\item дополнительная заработная плата исполнителей темы;
\item отчисления во внебюджетные фонты (страховые отчисления);
\item накладные расходы.
\end{enumerate}

\subsubsection{Расчет затрат на оборудование для научно-экспериментальных работ} \label{eco.3.1}

Расчет   затрат   на   оборудование   сводится к   определению амортизационных  отчислений,  так  как  оборудование  было  приобретено  до начала выполнения этой работы. 

Норма амортизации вычисляется по следующей формуле:

%%% Норма амортизации
\begin{equation}\label{eq:eq-eco5}
N_a = \frac{1}{n},
\end{equation}
где \textit{n} -- срок полезного использования, измеряемый в годах.

Амортизация  оборудования  линейным  способом  рассчитывается следующим образом:

%%% Линейная амортизация
\begin{equation}\label{eq:eq-eco6}
A = \frac{N_a \cdot m \cdot N}{12}
\end{equation}
где \textit{N}–итоговая сумма, тыс. руб.;\textit{m}–время использования, мес.

Единственным  оборудованием,  использованным  в  работе  был  ПК \textit{DEXPMarsE320}, приобретенный в декабре 2020 года за 59999 рублей. Срок полезного использования ПК составляет 5 лет. В   итоге   общая   сумма амортизационных отчислений составила:

%%% Расчет амортизации
\begin{gather*}
A = \frac{0,2 \cdot 59999 \cdot 2}{12} 
= 1999,97 \approx 2000\ \textnormal{руб}.
\end{gather*}

\subsubsection{Основная заработная плата исполнителей темы} \label{eco.3.2}

Статья   включает   основную   заработную   плату   работников, непосредственно занятых выполнением НТИ, (включая премии и доплаты) и дополнительную   заработную   плату.   Также   включается   премия, выплачиваемая ежемесячно из фонда заработной платы в размере 20-30 \% от тарифа или оклада:

%%% ОЗП
\begin{equation}\label{eq:eq-eco7}
\textnormal{З}_{\textnormal{Зп}} 
= \textnormal{З}_{\textnormal{осн}}
= \textnormal{З}_{\textnormal{доп}}
\end{equation}
где $\textnormal{З}_{\textnormal{осн}}$ -- основная заработная плата; $\textnormal{З}_{\textnormal{доп}}$ -- дополнительная заработная плата (12-20 \% от $\textnormal{З}_{\textnormal{осн}}$).

Основная  заработная  платаруководителя  (лаборанта,  инженера)  от предприятия (при наличии руководителя от предприятия) рассчитывается по следующей формуле:

%%% ОЗП
\begin{equation}\label{eq:eq-eco8}
\textnormal{З}_{\textnormal{осн}} 
= \textnormal{З}_{\textnormal{дн}} \cdot T_p
\end{equation}
где $T_p$ -- продолжительность работ, выполняемых научно-техническим работником (таблица \ref{tab:eco9}); $\textnormal{З}_{\textnormal{дн}}$ -- среднедневная заработная плата работника, руб.

Среднедневная заработная плата рассчитывается по формуле:

%%% Среднедневная ЗП
\begin{equation}\label{eq:eq-eco9}
\textnormal{З}_{\textnormal{д}} 
= \frac{\textnormal{З}_{\textnormal{м}}}{F_{\textnormal{д}}}
\end{equation}
где   $\textnormal{З}_{\textnormal{м}}$ -- месячный должностной оклад работника, руб.; М – количество месяцев работы без отпуска в течение года: при отпуске в 24 раб.дня М =11,2 месяца, 5-дневная неделя; при отпуске в 48 раб.дней М=10,4 месяца, 6-дневная неделя; $F_{\textnormal{д}}$ – действительный  годовой  фонд  рабочего  времени  научно-технического персонала, раб.дн. (таблица \ref{tab:eco9}).

В таблице \ref{tab:eco10} приведен баланс рабочего времени каждого работника НТИ.

%%% Баланс рабочего времени
\begin{table}[H]
\centering
\caption{Баланс рабочего времени}
\label{tab:eco10}
\begin{tabular}{|l|c|c|}
\hline
Показатели рабочего времени                                                                                & \multicolumn{1}{l|}{Руководитель}                & \multicolumn{1}{l|}{Бакалавр}                    \\ \hline
Календарное число дней                                                                                     & 365                                              & 365                                              \\ \hline
\begin{tabular}[c]{@{}l@{}}Количество нерабочих дней\\  -- выходные дни\\  -- праздничные дни\end{tabular} & \begin{tabular}[c]{@{}c@{}} \\ 104\\ 14\end{tabular} & \begin{tabular}[c]{@{}c@{}} \\ 104\\ 14\end{tabular} \\ \hline
\begin{tabular}[c]{@{}l@{}}Потери рабочего времени\\  -- отпуск\\  -- невыходы по болезни\end{tabular}     & \begin{tabular}[c]{@{}c@{}}\\ 24\\ 7\end{tabular}   & \begin{tabular}[c]{@{}c@{}}\\ 24\\7\end{tabular}   \\ \hline
\begin{tabular}[c]{@{}l@{}}Действительный годовой фонд\\ рабочего времени\end{tabular}                     & 216                                              & 216                                              \\ \hline
\end{tabular}%
\end{table}

Месячный должностной оклад работника:

%%% Месячный оклад
\begin{equation}\label{eq:eq-eco10}
\textnormal{З}_{\textnormal{м}} 
= \textnormal{З}_{\textnormal{тс}} 
\cdot (1 
+ k_{\textnormal{пр}}
+ k_{\textnormal{д}}) \cdot k_p, 
\end{equation}
где   $\textnormal{З}_{\textnormal{тс}} $ -- заработная плата по тарифной ставке, руб.; $k_{\textnormal{пр}}$ -- премиальный коэффициент, равный 0,3; $k_{\textnormal{д}}$ -- коэффициент доплат и надбавок составляет0,2; $k_p$ -- районный коэффициент,для г. Томска равный 1,3.

Расчёт основной заработной платы приведѐн в таблице \ref{tab:eco11}.

%%% Расчет ОЗП
\begin{table}[H]
\centering
\caption{Расчет основной заработной платы}
\label{tab:eco11}
\begin{tabular}{|c|c|c|c|c|c|c|c|c|}
\hline
Категория & \begin{tabular}[c]{@{}c@{}}$\textnormal{З}_{\textnormal{тс}}$,\\ руб\end{tabular} & $k_{\textnormal{д}}$  & $k_{\textnormal{пр}}$ & $k_p$  & \begin{tabular}[c]{@{}c@{}}$\textnormal{З}_{\textnormal{м}} $,\\ руб\end{tabular} & \begin{tabular}[c]{@{}c@{}}$\textnormal{З}_{\textnormal{дн}} $,\\ руб\end{tabular} & \begin{tabular}[c]{@{}c@{}}$T_p$,\\ раб. дн.\end{tabular} & \begin{tabular}[c]{@{}c@{}}$\textnormal{З}_{\textnormal{осн}} $,\\ руб\end{tabular} \\ \hline
\multicolumn{9}{|c|}{Руководитель}                                                                                                                                                                                                                                                                       \\ \hline
ППС3      & 22000                                              & 0,3 & 0,2 & 1,3 & 42900                                             & 2224,4                                             & 17,7                                                   & 39372,7                                             \\ \hline
\multicolumn{9}{|c|}{Бакалавр}                                                                                                                                                                                                                                                                           \\ \hline
ППС1      & 9000                                               & 0,3 & 0,2 & 1,3 & 17550                                             & 910,0                                              & 30,7                                                   & 27937,0                                             \\ \hline
\multicolumn{8}{|c|}{\textbf{Итого}}                                                                                                                                                                                                               & 67309,7                                             \\ \hline
\end{tabular}%
\end{table}

Расчет  дополнительной  заработной  платы  ведется  по  следующей формуле:

%%%Допзп
\begin{equation}\label{eq:eq-eco11}
\textnormal{З}_{\textnormal{доп}} 
= k_{\textnormal{доп}}
\cdot \textnormal{З}_{\textnormal{осн}}, 
\end{equation}
где $k_\textnormal{доп}$ -- коэффициент  дополнительной  заработной  платы  (на  стадии проектирования принимается равным 0,15).

Общая заработная исполнителей работы представлена в таблице \ref{tab:eco12}.

\begin{table}[H]
\centering
\caption{Общая заработная плата исполнителей}
\label{tab:eco12}
\begin{tabular}{|c|c|c|c|}
\hline
Исполнитель    & $\textnormal{З}_{\textnormal{осн}}$, руб        & $\textnormal{З}_{\textnormal{доп}}$, руб        & $\textnormal{З}_{\textnormal{зп}}$, руб         \\ \hline
Руководитель   & 39372,7          & 5905,9           & 45278,6          \\ \hline
Бакалавр       & 27937,0          & 4190,6           & 32127,6          \\ \hline
\textbf{Итого} & \textbf{67309,7} & \textbf{10096,5} & \textbf{77406,1} \\ \hline
\end{tabular}%
\end{table}

\subsubsection{Отчисления во снебюджетные фонды (страховые отчисления)} \label{eco.3.3}

В  данной  статье  расходов  отражаются  обязательные  отчисления  по установленным  законодательством  Российской  Федерации  нормам  органам государственного социального страхования (ФСС), пенсионного фонда (ПФ) и  медицинского  страхования  (ФФОМС)  от  затрат  на  оплату  труда работников.

Величина отчислений во внебюджетные фонды определяется исходя из следующей формулы: 

%%% Отчисления
\begin{equation}\label{eq:eq-eco12}
\textnormal{З}_{\textnormal{внеб}} 
= k_{\textnormal{пр}} 
\cdot (\textnormal{З}_{\textnormal{осн}} 
+ (\textnormal{З}_{\textnormal{доп}}), 
\end{equation}
где  $k_{\textnormal{внеб}}$ – коэффициент  отчислений  на  уплату  во  внебюджетные  фонды (пенсионный  фонд,  фонд  обязательного  медицинского  страхования  и пр.). 

Отчисления во внебюджетные фонды представлены в таблице \ref{tab:eco13}.

%%% Внебюджетные фонды
\begin{table}[H]
\centering
\caption{Отчисления во внебюджетные фонды}
\label{tab:eco13}
\begin{tabular}{|c|c|c|}
\hline
Исполнитель                                                                             & \begin{tabular}[c]{@{}c@{}}Основная заработная \\ плата, руб.\end{tabular} & \begin{tabular}[c]{@{}c@{}}Дополнительная \\ заработная плата,\\ руб\end{tabular} \\ \hline
Руководитель проекта                                                                    & 39372,7                                                                    & 5905,9                                                                            \\ \hline
                                                                                        & 27937,0                                                                    & 4190,6                                                                            \\ \hline
\begin{tabular}[c]{@{}c@{}}Коэффициент отчислений \\ во внебюджетные фонты\end{tabular} & \multicolumn{2}{c|}{0,302}                                                                                                                                     \\ \hline
\textbf{Итого:}                                                                         & \multicolumn{2}{c|}{\textbf{23376,7}}                                                                                                                          \\ \hline
\end{tabular}%
\end{table}

\subsubsection{Накладные расходы} \label{eco.3.4}

Накладные  расходы  учитывают  прочие затраты  организации,  не попавшие  в  предыдущие  статьи  расходов:  печать  и  ксерокопирование материалов  исследования,  оплата  услуг  связи,  электроэнергии,  почтовые  и телеграфные  расходы,  размножение  материалов  и  т.д.  Их  величина определяется по следующей формуле:

%%% Накладные
\begin{equation}\label{eq:eq-eco13}
\textnormal{З}_{\textnormal{накл}} 
= k_{\textnormal{нр}} 
\cdot (\textnormal{\textit{сумма статей}}\ 1 \div 4), 
\end{equation}

где   $k_{\textnormal{нр}}$ – коэффициент, учитывающий накладные расходы.

Величину  коэффициента  накладных  расходов  можно  взять  в  размере 20 \%.

%%% Расчет накладных
\begin{gather*}
\textnormal{З}_{\textnormal{накл}} 
= 0,2 \cdot (2000.0 + 67309,7 + 10096,5 + 22362,1) = 20556,6 \ \textnormal{руб}
\end{gather*}

\subsubsection{Формирование бюджета затрат научно-исследовательского проекта} \label{eco.3.5}

Рассчитанная  величина  затрат  научно-исследовательской  работыявляется основой для формирования бюджета затрат проекта, который при формировании  договора  с  заказчиком защищается  научной  организацией  в качестве  нижнего  предела  затрат  на  разработку  научно-технической продукции.

Определение бюджета затрат на научно-исследовательский проект по каждому варианту исполнения приведен в таблице \ref{tab:eco14}.

%%% Бюджет затрат НТИ
\begin{table}[H]
\centering
\caption{Расчет бюджета затрат НТИ}
\label{tab:eco14}
\begin{tabular}{|c|c|}
\hline
Наименование статьи                                                                                                       & Сумма, руб \\ \hline
\begin{tabular}[c]{@{}c@{}}1. Затраты на специальное\\ оборудование для научных\\ (экспериментальных) работ\end{tabular} & 2000       \\ \hline
\begin{tabular}[c]{@{}c@{}}2. Затраты по оновной \\ заработной плате \\ исполнителей темы\end{tabular}                   & 67309,7    \\ \hline
\begin{tabular}[c]{@{}c@{}}3. Затраты по\\ дополнительной заработной \\ плате исполнителей темы\end{tabular}             & 10096,5    \\ \hline
\begin{tabular}[c]{@{}c@{}}4. Отчисления во\\ внебюджетные фонды\end{tabular}                                            & 23376,7    \\ \hline
5. Накладные расходы                                                                                                     & 20556,6    \\ \hline
6. Бюджет затрат НТИ                                                                                                     & 12339,3    \\ \hline
\end{tabular}%
\end{table}

Как  видно  из  таблицы \ref{tab:eco14} основные  затраты  НТИ  приходятся  назаработную плату исполнителей.

\subsection{Определение  ресурсной  (ресурсосберегающей),  финансовой, бюджетной, социальной и экономической эффективности исследования} \label{eco.4}

Для   определения    эффективности    исследования    рассчитан интегральный  показатель  эффективности  научного  исследования  путем определения  интегральных  показателей  финансовой  эффективности  и ресурсоэффективности.

Интегральный финансовый показатель разработки определяется как:

%%% Инфинпок
\begin{equation}\label{eq:eq-eco14}
I^{\textnormal{исп.i}}_{\textnormal{финр}}
= \frac{\textnormal{Ф}_{pi}}{\textnormal{Ф}_{max}}, 
\end{equation}
где $I^{\textnormal{исп.i}}_{\textnormal{финр}}$ –интегральный финансовый показатель разработки;$\textnormal{Ф}_{pi}$ – стоимость \textit{i}-го варианта исполнения; $\textnormal{Ф}_{max}$ – максимальная стоимость исполнения научно-исследовательского проекта (в т.ч. аналоги).

Интегральный    показатель    ресурсоэффективности    вариантов исполнения объекта исследования можно определить следующим образом:

%%% Интегральный показатель
\begin{equation}\label{eq:eq-eco15}
I_{pi} = \sum a_i \cdot b_i
\end{equation}
где $I_{pi}$ – интегральный показатель ресурсоэффективности для \textit{i}-го варианта исполнения разработки; $a_i$ – весовой коэффициент \textit{i}-го варианта исполнения разработки; $b_i^a$, $b_i^p$ – бальная  оценка \textit{i}-го  варианта  исполнения  разработки, устанавливается экспертным путем по выбранной шкале оценивания; \textit{n} – число параметров сравнения.

Расчет интегрального показателя ресурсоэффективности приведен в форме таблицы \ref{tab:eco15}.

%%%Варианты исполнения
\begin{table}[H]
\small
\centering
\caption{Сравнительная оценка характеристик вариантов исполнения проекта}
\label{tab:eco15}
\begin{tabular}{|c|c|c|c|c|}
\hline
\begin{tabular}[c]{@{}c@{}}\backslashbox{Объект исследования}{Критерии}\end{tabular}          & \begin{tabular}[c]{@{}c@{}}Весовой \\ коэффициент \\ параметра\end{tabular} & \begin{tabular}[c]{@{}c@{}}Текущий \\ проект\end{tabular} & Исп.2 & Исп.3 \\ \hline
1. Универсальность методики                                                     & 0,4                                                                         & 5                                                         & 1     & 2     \\ \hline
2. Трудоемкость реализации                                                      & 0,2                                                                         & 5                                                         & 2     & 1     \\ \hline
\begin{tabular}[c]{@{}c@{}}3. Учет инструментальных\\ особенностей\end{tabular} & 0,4                                                                         & 1                                                         & 5     & 5     \\ \hline
Итого:                                                                          & 1                                                                           & 3,4                                                       & 2,8   & 3     \\ \hline
\end{tabular}
\end{table}

Сравнив  значения  интегральных  показателей  ресурсоэффективности можно сделать вывод, что реализация методики в текущем проекте является более   эффективным   вариантом   для   проектирования   с   позиции ресурсосбережения.

Интегральный  показатель  эффективности  вариантов  исполнения разработки($I_{\textnormal{исп.i}}$)  определяется  на  основании  интегрального  показателя ресурсоэффективности и интегрального финансового показателя по формуле:

%%% Интегральный показатель эффективности
\begin{equation}\label{eq:eq-eco16}
I_{\textnormal{исп.2}} 
= \frac{I_{\textnormal{p-исп.2}}}
{I^{\textnormal{исп.i}}_{\textnormal{финр}}}
,
I_{\textnormal{исп.1}} 
= \frac{I_{\textnormal{p-исп.1}}}
{I^{\textnormal{исп.i}}_{\textnormal{финр}}}
,\ \textnormal{и т. д.}
\end{equation}

Сравнение   интегрального   показателя   эффективности   вариантов исполнения разработки позволит определить сравнительную эффективность проекта  (см.  таблицу  \ref{tab:eco16})  и  выбрать  наиболее  целесообразный  вариант  из предложенных. Сравнительная эффективность проекта ($\textnormal{Э}_{\textnormal{ср}}$):

%%% Сравнительная эффективность проекта
\begin{equation}\label{eq:eq-eco17}
\textnormal{Э}_{\textnormal{ср}} 
= \frac{I_{\textnormal{исп.1}}}
{I_{\textnormal{исп.2}}}.
\end{equation}

%%% Сравнительная эффективность разработки
\begin{table}[H]
\centering
\caption{Сравнительная эффективность разработки}
\label{tab:eco16}
\begin{tabular}{|l|l|c|c|c|}
\hline
\multicolumn{1}{|c|}{\begin{tabular}[c]{@{}c@{}}№\\ п/п\end{tabular}} & \multicolumn{1}{c|}{Показатели}                                                                    & \begin{tabular}[c]{@{}c@{}}Текущий\\ проект\end{tabular} & Исп.2 & Исп.3 \\ \hline
1                                                                     & \begin{tabular}[c]{@{}l@{}}Интегральный финансовый показатель\\ разработки\end{tabular}            & 0,97                                                     & 0,97  & 1,00  \\ \hline
2                                                                     & \begin{tabular}[c]{@{}l@{}}Интегральный ппоказатель\\ ресурсоэффективности разработки\end{tabular} & 3,40                                                     & 2,80  & 3,00  \\ \hline
3                                                                     & Интегральный показатель эффективности                                                              & 3,50                                                     & 2,88  & 3,00  \\ \hline
4                                                                     & \begin{tabular}[c]{@{}l@{}}Сравнительная эффективность вариантов \\ исполнения\end{tabular}        & 1                                                        & 0,82  & 0,86  \\ \hline
\end{tabular}
\end{table}

Вывод: сравнительный    анализ    интегральных    показателей эффективности   показывает,   что   предпочтительным   для   осуществления калибровки является  первый  вариант  исполнения, так  как является наиболее экономичным и ресурсоэффективным.

В результате выполнения целей раздела можно сделать следующие выводы:

%%% Выводы
\begin{enumerate}[wide]
\item В  результате  анализа  конкурентных  решений  выяснили,  что выбранная методика калибровки является наиболее эффективной.
\item В  ходе  планирования  для  руководителя,  консультантов  по социальной ответственности  и  экономической  части  и  бакалаврабыл разработан график реализации этапа работ, который позволяет оценивать и планировать  рабочее  время  исполнителей.  Определено  следующее:  общее количество дней для выполнения работ составляет 48дней.
\item Для  оценки  затрат  на  реализацию  проекта  разработан  проектный бюджет, который составляет 123339,3 руб;
\end{enumerate}

Результат оценки эффективности ИР показывает следующее:

%%% Еще выводы
\begin{enumerate}[wide]
\item значение интегрального финансового показателя ИР составляет 0,97;
\item значение  интегрального  показателя  ресурсоэффективности  ИР составляет 3.40, в то время как при других вариантах исполнения значения показателя составляют 2,80 и 3,00.
\item значение  интегрального  показателя  эффективности  ИР  составляет 3,50 по сравнению с 2.88 и 3,00, и является наиболее высоким, что означает, чтотехническое  решение,  рассматриваемое  в  ИР  является  наиболее эффективным вариантом исполнения.
\end{enumerate}

\newpage
\addcontentsline{toc}{section}{10\ \ \ Социальная ответственность}
\addcontentsline{toc}{subsection}{10.1\ \ \ Оценка вредных и опасных факторов}
\includepdf[pages=1]{8Soc.pdf}

\addcontentsline{toc}{subsubsection}{10.1.1\ \ \ Микроклимат}
\includepdf[pages={2-4}]{8Soc.pdf}

\addcontentsline{toc}{subsubsection}{10.1.2\ \ \ Шум}
\includepdf[pages={5}]{8Soc.pdf}

\addcontentsline{toc}{subsubsection}{10.1.3\ \ \ Освещение}
\includepdf[pages={6-7}]{8Soc.pdf}

\addcontentsline{toc}{subsubsection}{10.1.4\ \ \ Электромагнитные поля}
\includepdf[pages={8}]{8Soc.pdf}

\addcontentsline{toc}{subsubsection}{10.1.5\ \ \ Пожароопасность }
\includepdf[pages={9}]{8Soc.pdf}

\includepdf[pages={10}]{8Soc.pdf}
\addcontentsline{toc}{subsubsection}{10.1.6\ \ \ Электробезопасность }

\includepdf[pages={11}]{8Soc.pdf}
\addcontentsline{toc}{subsubsection}{10.1.7\ \ \ Радиационная безопасность }

\includepdf[pages={12}]{8Soc.pdf}
\addcontentsline{toc}{subsection}{10.2\ \ \  Аварийные ситуации }
\includepdf[pages={13}]{8Soc.pdf}
\includepdf[pages={14}]{8Soc.pdf}

\includepdf[pages={15}]{8Soc.pdf}
\addcontentsline{toc}{subsection}{10.3\ \ \   Выводы по главе }
\includepdf[pages={16}]{8Soc.pdf}

\section*{Заключение}  \label{sec:concl} 
\addcontentsline{toc}{section}{Заключение}

С целью калибровки электромагнитного калориметра \textit{ECAL} эксперимента \textit{NA64}, а также получения зависимости предельного энергетического разрешения калориметра от энергии пучка было осуществлено численное моделирование с помощью пакета \textit{GEANT4}. Помимо этого проводилось изучение энергетических утечек.

В результате моделирования энергетического разрешения выяснилось, что зависимость разрешения от энергии инициирующей частицы содержит постоянный член, неучтённый в моделях работы [16]. Это можно объяснить неустранимыми энергетическими утечками в материале калориметра или ошибкой округления при суммировании низкоэнергетической компоненты спектра. С увеличением размеров калориметра, энергия поглощенная в калориметре увеличивается, однако недостаток энергии около \mbox{200 МэВ} не удается снизить таким образом. Предположительно, его формируют слабо взаимодействующие частицы, рождающиеся в фотоядерных реакциях.

Рассмативаемый алгоритм извлечения калибровочных коэффициентов предполагает параметризацию электромагнитного ливня, развивающегося в детекторе. Для этого было осуществлено моделирование, после чего полученный профиль ливня аппроксимировался факторизованной функцией, полученной в приближении гомогенного калориметра. После применения калибровки на экспериментальных данных, был получен результат, свидетельствующий о значительной количественной ошибке в калибровке, однако в дальнейшем планируется учесть рассмотренные в работе эффекты связанные с анизотропией излучения в веществе калориметра, конверсией типов частиц и светосбором с целью повышения точности реконструкции.


\newpage
\section*{Литература} \label{sec:lit}
\addcontentsline{toc}{section}{Литература} 

%%% Miscalibrations explanation
 \begin{itemize}[leftmargin=1.6\parindent, wide]
\item [1.] Bertone G., Hooper D. History of dark matter //Reviews of Modern Physics. – 2018. – Т. 90. – №. 4. – С. 045002.
\item [2.] Bednyakov V. A., Klapdor-Kleingrothaus H. V. Direct search for dark matter—striking the balance—and the future //Physics of Particles and Nuclei. – 2009. – Т. 40. – №. 5. – С. 583-611.
\item [3.] Bjorken J. D. et al. New fixed-target experiments to search for dark gauge forces //Physical Review D. – 2009. – Т. 80. – №. 7. – С. 075018.
\item [4.] Wigmans   R.,   Wigmans   R.   Calorimetry:   Energy   measurement   in particle physics. –Oxford University Press, 2000. – Т. 107.
\item [5.] Adinolfi  M.  et  al.  The  KLOE  electromagnetic  calorimeter  //Nuclear Instruments   and   Methods   in   Physics   Research   Section   A:   Accelerators, Spectrometers, Detectors and Associated Equipment. – 2002. – Т. 482. – No. 1-2. – С. 364-386.
\item [6.] Gingrich D. M. et al. Performance of a large scale prototype of the ATLAS accordion electromagnetic calorimeter //Nuclear Instruments and Methods in Physics Research Section A: Accelerators, Spectrometers, Detectors and Associated Equipment. – 1995. – Т. 364. – №. 2. – С. 290-306.
\item[7.] Akchurin N. et al. Beam test results from a fine-sampling quartz fiber calorimeter for electron, photon and hadron detection //Nuclear Instruments and Methods in Physics Research Section A: Accelerators, Spectrometers, Detectors and Associated Equipment. – 1997. – Т. 399. – №. 2-3. – С. 202-226.
\item[8.] Peigneux J. P. et al. Results from tests on matrices of lead tungstate crystals using high energy beams //Nuclear Instruments and Methods in Physics Research Section A: Accelerators, Spectrometers, Detectors and Associated Equipment. – 1996. – Т. 378. – №. 3. – С. 410-426.
\item[9.] Åkesson T. et al. Performance of the uranium/plastic scintillator calorimeter for the HELIOS experiment at CERN //Nuclear Instruments and Methods in Physics Research Section A: Accelerators, Spectrometers, Detectors and Associated Equipment. – 1987. – Т. 262. – №. 2-3. – С. 243-263.
\item[10.] Cervelli F. et al. A reduced scale em calorimeter prototype for the AMS-02 experiment //Nuclear Instruments and Methods in Physics Research Section A: Accelerators, Spectrometers, Detectors and Associated Equipment. – 2002. – Т. 490. – №. 1-2. – С. 132-139.
\item[11.] Aharrouche M. et al. Energy linearity and resolution of the ATLAS electromagnetic barrel calorimeter in an electron test-beam //Nuclear Instruments and Methods in Physics Research Section A: Accelerators, Spectrometers, Detectors and Associated Equipment. – 2006. – Т. 568. – №. 2. – С. 601-623.
\item[12.] Ganel O., Wigmans R. On the calibration of longitudinally segmented calorimeter systems //Nuclear Instruments and Methods in Physics Research Section A: Accelerators, Spectrometers, Detectors and Associated Equipment. – 1998. – Т. 409. – №. 1-3. – С. 621-628.
\item[13.] Albrow  M.  et  al.  Intercalibration  of  the  longitudinal  segments  of  a calorimeter   system   //Nuclear   Instruments   and   Methods   in   Physics   Research Section  A:  Accelerators,Spectrometers,  Detectors  and  Associated  Equipment. – 2002. – Т. 487. – №. 3. – С. 381-395.
\item[14.] Grindhammer G., Peters S. The parameterized simulation of electromagnetic showers in homogeneous and sampling calorimeters //arXiv preprint hep-ex/0001020. – 2000.
\item[15.] Grindhammer  G.,  Rudowicz  M.,  Peters  S.The  fast  simulation  of electromagnetic  and  hadronic  showers  //Nuclear  Instruments  and  Methods  in Physics   Research   Section   A:   Accelerators,   Spectrometers,   Detectors   and Associated Equipment. – 1990. – Т. 290. – No. 2-3. – С. 469-488.
\item[16.] Del Peso J., Ros E. On the energy resolution of electromagnetic sampling calorimeters //Nuclear Instruments and Methods in Physics Research Section A: Accelerators, Spectrometers, Detectors and Associated Equipment. – 1989. – Т. 276. – №. 3. – С. 456-467.
\item[17.] Galassi M. et al. GNU scientific library //Reference Manual. Edition 1.4, for GSL Version 1.4. – 2003.
\item[18.] ГОСТ  12.0.003-2015  Система  стандартов  безопасности  труда  (ССБТ). Опасные  и  вредные  производственные  факторы.  Классификация Режим доступа: https://docs.cntd.ru/document/1200136071/ (дата обращения: 14.02.21)
\item[19.] ГОСТ  30494-96 Здания  жилые  и общественные.  Параметры микроклимата в помещениях Режим доступа: http://docs.cntd.ru/document/1200003003 (дата обращения: 15.02.21)
\item[20.] ГОСТ 12.1.003-83 Система стандартов безопасности труда (ССБТ). Шум. Общие  требования  безопасности  (с  Изменением  № 1) Режим  доступа: http://docs.cntd.ru/document/5200291 (дата обращения: 15.02.21)
\item[21.] СНиП  23-05-95*  Естественное  и  искусственное  освещение  (с Изменением № 1) Режим доступа:http://docs.cntd.ru/document/871001026 (дата обращения: 15.02.21)
\item[22.] СП  12.13130.2009.  Определение  категорий  помещений,  зданий  и наружных установок по взрывопожарной и пожарной опасности (в ред. изм. №  1, утв. приказом МЧС России от 09.12.2010 № 643). [Электронный ресурс]. Доступ из сборника НСИС ПБ. –2011. – № 2 (45).
\item[23.] ГОСТ 12.1.004-91  Система  стандартов  безопасности  труда.  Пожарная безопасность. Общие требования Режим доступа: https://docs.cntd.ru/document/9051953 (дата обращения: 03.03.2021)
\item[24.] ГОСТ  12.1.009-76  Система  стандартов  безопасности  труда  (ССБТ). Электробезопасность.   Термины   и   определения Режим   доступа: http://docs.cntd.ru/document/5200278 (дата обращения: 18.02.21)
\item[25.] ГОСТ 12.1.019-2017  ССБТ  Электробезопасность Режим  доступа: https://beta.docs.cntd.ru/document/1200161238 (дата обращения: 19.02.21)
\item[26.] ГОСТ Р МЭК 61140-2000 Защита  от  поражения  электрическим  током. Общие положения по безопасности, обеспечиваемой электрооборудованием и   электроустановками   в   их   взаимосвязи.   Режим   доступа: https://docs.cntd.ru/document/1200017996 (дата обращения: 05.03.2021)
\item[27.] СанПиН 2.6.1.2523-09 Нормы радиационной безопасности НРБ-99/2009 Режим доступа: https://base.garant.ru/4188851/53f89421bbdaf741eb2d1ecc4ddb4c33/ (дата обращения: 21.02.21)
\item[28.] СанПиН  2.2.4.548-96  Гигиенические  требования  к  микроклимату производственных помещений Режим доступа: http://docs.cntd.ru/document/901704046 (дата обращения: 15.02.21)
\item[29.] СНиП 41-01-2003 Отопление, вентиляция и кондиционирование Режим доступа: http://docs.cntd.ru/document/1200035579 (дата обращения: 15.02.21)
\item[30.] ГОСТ 12.1.029-80 Средства и методы защиты от шума. Режим доступа: http://docs.cntd.ru/document/5200292 (дата обращения: 15.02.21)
\item[31.] ГОСТ  12.4.026-76*  Система  стандартов  безопасности  труда.  Цвета сигнальные    и    знаки    безопасности Режим    доступа: http://docs.cntd.ru/document/1200003391 (дата обращения: 15.02.21)
\item[32.] СанПиН  2.2.2/2.4.1340-03  О  введении  в  действие  санитарно-эпидемиологических   правил   и   нормативов Режим   доступа: http://docs.cntd.ru/document/901865498 (дата обращения: 16.02.21)
\item[33.] ГОСТ  12.1.006-84  Электромагнитные поля  радиочастот.  Допустимые уровни  на  рабочих  местах  и  требования  к  проведению  контроля Режим доступа: http://docs.cntd.ru/document/5200272 (дата обращения: 16.02.21)
\item[34.] ГОСТ Р 22.0.02-2016  Безопасность  в  чрезвычайных  ситуациях. Термины и определения Режим доступа: https://docs.cntd.ru/document/1200139176 (дата обращения: 11.03.2021)
 \end{itemize}
 
\end{document}   
