\newpage
\section*{Введение}  \label{sec:intro}
\addcontentsline{toc}{section}{Введение}

В 30-х годах XX века наблюдения за движением материи во Вселенной стали указывать на наличие во Вселенной значительной массы, недоступной для прямой регистрации и проявляющей себя только в гравитационном взаимодействии [1]. Эта скрытая масса была названа тёмной материей. Однако, за прошедшее время темную материю не удалось зарегестрировать непосредственно, и до сих пор она остается одной из важнейших загадок современных ксомологии и астрофизики.

Есть основания полагать [2], что поиски темной материи можно проводить не только методами астрофизики, но и на ускорителях частиц..  Предполагается, что при столкновении высокоэнергетичных частиц между собой или с веществом мишени могут образовываться частицы тёмной материи. Такие эксперименты обычно подразумевают использование калориметров для регистрации полной энергии в различных реакциях (герметичная постановка эксперимента со сбросом пучка -- \textit{beam dump}) [3].

Изначально калориметры производились как довольно грубые, но дешевые инструменты для специализированного применения. Например, для детектирования нейтринных взаимодействий. В современных коллайдерных экспериментах они являются одними из ключевых элементов в телескопах детекторов. Они подходят для множества задач, начиная от отбора событий и заканчивая точными измерениями четырехвекторов отдельных частиц и сгустков частиц и получением информации об энерговыделении в результате различных событий.
	
Вклад калориметрической информации в анализ данных заключается, в основном, в идентификации частиц и измерении энергии частиц, порождающих электромагнитные ливни. Ожидается, что значение адронной калориметрии будет расти с дальнейшим ростом энергий сталкивающихся частиц.
            
\textit{NA64} – это эксперимент с фиксированной мишенью, на протонном суперсинхротроне (\textit{SPS}), расположенном в Европейском центре ядерных исследований (\textit{CERN}), и направленный на поиск экзотических легких бозонов. Ключевым детектором в постановке эксперимента является гетерогенный электромагнитный калориметр \textit{ECAL}. Для получения достоверных результатов необходима тщательная калибровка калориметра. 

Одним из наиболее результативных методов калибровки является калибровка, основанная на компьютерном моделировании. Целью настоящей работы является построение модели электромагнитного ливня для реконструкции энерговыделения в калориметре \textit{ECAL}.

Задачи:

 \begin{itemize}[leftmargin=1.6\parindent, wide]
 	\item[---] определение структуры работ в рамках научного исследования;
 		\item[---] обзор литературы с целью определения наиболее подходящей параметризации электромагнитного ливня для известных конструкции калориметра и диапазона энергий;
 			\item[---] получение профиля электромагнитного ливня на основе МК-модели, достижение согласия результатов моделирования с предсказаниями параметризации;
 				\item[---] определение энергетического разрешения электромагнитного калориметра.
 \end{itemize}
