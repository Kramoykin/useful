\newpage
\section{Социальная ответственность}  \label{socchap:1} 

В   рамках выполнения выпускной   квалификационной   работы проводилось калибровка электромагнитного калориметра \textit{ECAL} эксперимента \textit{NA64}. Основная  часть  работы  выполнялась  на  ПК, находящемся в лабораторной аудитории № 248 11-го корпуса ТПУ. 

Работа  заключалась  в  моделировании  развития  электромагнитного ливня,  образованного  в  материале  детектора  инициирующей  частицей  и проведении расчетов, основанных на результатах моделирования.

\subsection{Оценка вредных и опасных факторов}  \label{socsec:1} 

В данном пункте приведен анализ вредных и опасных факторов, которые могут возникать при работе в лаборатории. Все вредные и опасные факторы, характерные для лабораторной среды представлены в таблице \ref{tab:soctab:1}.

%%% Опасные и вредные факторы и нормативные документы
\begin{table}[H]
\footnotesize
\centering
\caption{Возможные вредные и опасные фактры}
\label{tab:soctab1}
\begin{tabular}{|l|l|}
\hline
Факторы(ГОСТ 12.0.003-2015[5:0])                                        & \begin{tabular}[c]{@{}l@{}}Нормативные\\ документы\end{tabular}                                                                                                                                                                                                                                                                                                   \\ \hline
1.Микроклимат                                                           & \begin{tabular}[c]{@{}l@{}}ГОСТ 30494-96.Здания жилые и общественные. \\ Параметры микроклимата в помещениях[5:1]\end{tabular}                                                                                                                                                                                                                                    \\ \hline
2. Шум                                                                  & \begin{tabular}[c]{@{}l@{}}ГОСТ 12.1.003-83.Система стандартов безопасности \\ труда (ССБТ). Шум. Общие требования безопасности (с \\ Изменением N 1) [5:2]\end{tabular}                                                                                                                                                                                          \\ \hline
\begin{tabular}[c]{@{}l@{}}3. Освещенность \\ рабочей зоны\end{tabular} & \begin{tabular}[c]{@{}l@{}}СНиП 23-05-95*.Естественное и искусственное \\ освещение (с Изменением N 1) [5:3]\end{tabular}                                                                                                                                                                                                                                         \\ \hline
4. Пожароопасность                                                      & \begin{tabular}[c]{@{}l@{}}СП 12.13130.2009. Определение категорий помещений, \\ зданий и наружных установок по взрывопожарной и \\ пожарной опасности (в ред. изм. No 1, утв. приказом \\ МЧС России от 09.12.2010 No 643) [5:4]\\ \\ ГОСТ12.1.004-91 Система стандартов безопасности \\ труда. Пожарная безопасность. Общие требования[5:5]\end{tabular}        \\ \hline
5. Электробезопасность                                                  & \begin{tabular}[c]{@{}l@{}}ГОСТ 12.1.009-76 Система стандартов безопасности \\ труда (ССБТ) [5:6]\\ \\ ГОСТ Р12.1.019-2017 ССБТ Электробезопасность [5:7]\\ \\ ГОСТРМЭК61140-2000 Защита от поражения \\ электрическим током. Общие положения по \\ безопасности, обеспечиваемой электрооборудованием и \\ электроустановками в их взаимосвязи [5:8]\end{tabular} \\ \hline
\begin{tabular}[c]{@{}l@{}}6. Радиационная \\ безопасность\end{tabular} & \begin{tabular}[c]{@{}l@{}}СанПиН 2.6.1.2523-09 Нормы радиационной \\ безопасности НРБ-99/2009 [5:9]\end{tabular}                                                                                                                                                                                                                                                 \\ \hline
\end{tabular}
\end{table}

\subsection{Микроклимат}  \label{socsec:2} 

Основными     факторами,     характеризующими     микроклимат производственной среды, являются: температура, подвижность и влажность воздуха.  Отклонение  этих  параметров  от  нормы приводит  к  ухудшению самочувствия  работника,  снижению  производительности  его  труда  и  к возникновению различных заболеваний.

Работа в условиях высокой температуры сопровождается интенсивным потоотделением,  что  приводит  к обезвоживанию  организма,  потере минеральных солей и водорастворимых витаминов, серьезнымизменениямв деятельности сердечно-сосудистой системы, увеличению частоты дыхания, а также  оказываетвлияние  на  функционирование  других  органов  и  систем (ослабление  внимания,  ухудшение  координации  движений,  замедление реакциитела и т.д.). 

Высокая  относительная  влажность при  высокой  температуре  воздуха способствует   перегревуорганизма,   при   низкой   же   температуре увеличивается теплоотдачас поверхности кожи, что ведет к переохлаждению организма. Низкая  влажность вызывает  неприятные  ощущения  в  виде сухости слизистых оболочек дыхательных путей работающего.

При  нормировании метеорологических  условий  в  производственных помещенияхучитывают  время  года,  физическую  тяжесть  выполняемых работ, а также количество избыточного тепла в помещении. Оптимальные и допустимые   метеорологические   условия   температуры   и   влажности устанавливаются согласно [5:10] и приведены в таблице \ref{tab:soctab2}.

Для  удобства  работы  в  помещении  необходимо  нормирование параметров микроклимата, то есть необходимо проведение мероприятий по контролю  способов  и  средств  защиты  от  высоких  и  низких  температур, системы   отопления, вентиляции   и   кондиционировании   воздуха, искусственное освещение и т.п.

%%% Показатели микроклимата
\begin{table}[H]
\footnotesize
\centering
\caption{Оптимальные  показатели  микроклимата  на  рабочих  местах производственных помещений}
\label{tab:soctab2}
\begin{tabular}{|c|c|c|c|c|c|}
\hline
\textbf{Период года} & \textbf{\begin{tabular}[c]{@{}c@{}}Категория работ \\ по уровню \\ энергозатра, Вт\end{tabular}} & \textbf{\begin{tabular}[c]{@{}c@{}}Температура \\ воздуха, {\degree}C\end{tabular}} & \textbf{\begin{tabular}[c]{@{}c@{}}Температура \\ поверхностей\\ , {\degree}C\end{tabular}} & \textbf{\begin{tabular}[c]{@{}c@{}}Относительная\\ влажность \\ воздуха, \%\end{tabular}} & \textbf{\begin{tabular}[c]{@{}c@{}}Скорость \\ движения\\ воздуха,\\ м/с\end{tabular}} \\ \hline
Холодный             & Iа (до 139)                                                                                      & 22-24                                                                      & 21-25                                                                              & 60-40                                                                                    & \begin{tabular}[c]{@{}c@{}}Не более\\ 0,1\end{tabular}                                 \\ \hline
Теплый               & Iа (до 139)                                                                                      & 23-25                                                                      & 22-26                                                                              & 60-40                                                                                    & \begin{tabular}[c]{@{}c@{}}Не более\\ 0,1\end{tabular}                                 \\ \hline
\end{tabular}
\end{table}