\newpage
\section{Функция отклика калориметра}  \label{chap2} 
\subsection{Абсолютный отклик и его отклонения} \label{chap2.1}

Определим  отклик  калориметра  как  средний  сигнал  калориметра  на единицу выделенной энергии. Таким образом, отклик может быть выражен в терминах числа фотоэлектронов на один ГэВ, в пикокулонах на МэВ и т.д. 

%%% average signal for wired chambers (from Wigmans)
\begin{figure}[!h]
    \centering
    \includegraphics[width=0.75\textwidth]{5.jpg}
    \caption{Средний сигнал гетерогенного калориметра с проволочными камерами, работающими в режиме «насыщенной лавины», на электромагнитный ливень как функция выделившейся энергии (а),сигнал для каждой из пяти секций (б) [4]}
    \label{fig:meanSign}
\end{figure}

В общем случае электромагнитные калориметры линейны только тогда, когда вся энергия пучка выделяется в результате процессов, которые могут производить  сигналы  (возбуждение  или  ионизация  в  слое  поглотителя). Нелинейность обычно свидетельствует об инструментальных проблемах, таких как  насыщение  сигнала и  ливневые  утечки.  На  рис. \ref{fig:meanSign} показан  отклик нелинейного  калориметра.  В  этом  детекторе  проволочные  камеры, использующиеся  для детектирования прохождении  частицы  из  ливня, работают  в  режиме  «насыщенной  лавины»,  это  значит,  что  камеры  не различают прохождение одной частицы и \textit{n} частиц. С повышением энергии ливня или плотности  частиц в ливне  эффект насыщения понижает отклик. Рис. \ref{fig:meanSign}(б), на котором представлена зависимость сигнала сегментированного на пять  секций  калориметра  от  числа  образовавшихся  в  каждой  секции позитронов, показывает,  что  не  столько  выделившаяся  энергия,  сколько плотность  образовавшихся  частиц  ответственна  за  подобные  эффекты,  так как  влияние  эффектов  наиболее  заметно  на  ранней  стадии  развития  ливня (секция  1),  при  которой ливень  еще  сильно  сколлимирован. Описанные эффекты можно избежать, используя камеры в пропорциональном режиме.

По своему устройству калориметры подразделяются на два класса:

\begin{enumerate}[wide]
\item Гомогенные калориметры, в которых поглотитель является также активным материалом, производящим сигнал.
\item Гетерогенные калориметры,  в  которых  каждый  материал выполняет свою функцию.
\end{enumerate}

В инструментах, относящихся ко второму классу, только доля энергии ливня выделяется в активном материале. Эта «сэмплирующая» доля обычно определяется на основании сигнала от минимально ионизирующей частицы (\textit{mip}). \textit{Mip} соответствует частице, удельные ионизирующие потери которой в веществах  поглотителя  и  активного  материала  будут  наименьшими. Например,  в  калориметре \textit{D0},  который  состоит  из  пластин  из $\mathrm{U^{238}}$, разделенных  промежутками  в  \mbox{4,6  мм},  заполненными жидким  аргоном, сэмплирующая   доля для \textit{mip} составляет \mbox{13,7   \%}. Однако для электромагнитных   ливней   сэмплирующая   доля   энергии   составляет приблизительно \mbox{8,2 \%}.

%%% e divided by mip as a function of plates thickness
\begin{figure}[!h]
    \centering
    \includegraphics[width=0.75\textwidth]{6.jpg}
    \caption{Отношение \textit{e/mip} как функция толщины слоёв поглотителей для калориметров уран/оргстекло и уран/жидкий аргон. Результат симуляции в программе \textit{EGS4} [4]}
    \label{fig:eDevMip}
\end{figure}

Причиной такого различия снова является тот факт, что основной вклад в сигнал от электромагнитного ливня вносят очень мягкие частицы. Гамма-кванты с энергиями ниже \mbox{1 МэВ} чрезвычайно неэффективно регистрируются в  детекторах этого класса, потому  что  наиболее  вероятное  взаимодействие при таких энергиях это фотоэффект. Сечение фотоэффекта пропорционально $Z^5$,  поэтому  практически  все  взаимодействия  мягких  гамма-квантов происходят  в  слоях  поглотителя,  и  вклад  в  сигнал  можно  ожидать  только если  взаимодействие  произошло  очень  близко  к  границе  поглотителя  с активным  слоем  (тогда  фотоэлектрон,  чей  пробег  меньше  \mbox{1  мм}, может, выбравшись  из  поглотителя, произвести  сигнал  в  жидком  аргоне). По причине  ключевой  роли  фотоэффекта,  его  влияние на  отношение \textit{e/mip} зависит от значений \textit{Z} пассивного и активного материалов (\textit{e/mip} наименьшее для  поглотителей  с  высоким \textit{Z} и  активных  материалов  с  малыми \textit{Z})  и  от толщины слоёв поглотителя (рис. \ref{fig:eDevMip}). Если поглотитель сделать достаточно тонким, то \textit{e/mip} становится практически равным 1.

\subsection{Флуктуации} \label{chap2.2}

Так  как  принцип  работы  калориметров  основан  на статистических процессах, точность калориметрических измерений определена и ограничена флуктуациями.   На   энергетическое   разрешение   электромагнитного калориметра влияет несколько флуктуационных процессов:

 \begin{itemize}[leftmargin=1.6\parindent, wide]
 	\item[---] квантовые   флуктуации   сигнала,   например   фотоэлектронная статистика;
 		\item[---] флуктуации ливневых утечек;
 			\item[---] флуктуации, обусловленные инструментальными эффектами, такими как  электронные  шумы,  ослабление  светового  потока  и  структурные неоднородности;
 				\item[---] флуктуации сэмплирования.
 \end{itemize}
 
 Последний процесс характерен только для гетерогенных калориметров. В хорошо спроектированных калориметрах этот вид флуктуаций преобладает над  остальными.  В  отличие  от  других  флуктуационных  процессов, флуктуации  сэмплирования  обусловлены  правилами  статистики  Пуассона. Таким  образом,  их  вклад  в  энергетическое  и  позиционное  разрешение описывается слагаемым, который пропорционален $1/\sqrt{E}:\sigma /E \sim E^{- 1/2}$.
 
 Флуктуации  сэмплирования  обусловлены  сэмплирующей  долей (или же   относительным   количеством   активного   материала)   и   частотой сэмплирования  (толщиной  слоёв).  В  электромагнитных  калориметрах  с  не газовым  активным  слоем  они  хорошо  описываются  следующей  формулой[4]:
 
 %%% Samplong fluctuations (from Wigmans)
\begin{equation}\label{eq:sampFluc}
\frac{\sigma}{E}=2,7\ \%\sqrt{ d/f_{sampl} }E^{-1/2},
\end{equation}
в которой \textit{d} – это толщина активного слоя (в мм), а $f_{sampl}$ – это сэмплирующая доля для \textit{mip}. Например, рассчитанная по формуле (\ref{eq:sampFluc}) составляющая энергетического разрешения для свинцово-сцинтилляционного калориметра \textit{KLOE} [5], соответствующая флуктуациям  сэмплирования, равна $6,9\ \%/\sqrt{E}$, что хорошо соответствует  экспериментально  найденному  значению  разрешения $5,7\ \%/\sqrt{E}$.

Среди  калориметров,  разрешение  которых  определяется  в  основном квантовыми  флуктуациями  сигналов,  можно  упомянуть  германиевые детекторы,  используемые  для  ядерной $\gamma$-спектрометрии,  и  калориметры  с кварцевым  волокном,  такие  как \textit{CMS} или \textit{HF}. Количество  энергии, необходимой для образования сигнала, в этих двух примерах различается на девять  порядков  по  величине.  В  то  время  как  достаточно  \mbox{1  эВ}  для образования электронно-дырочной пары в германиевом детекторе, светосбор в  кварцевом  калориметре  обычно  около  \mbox{1  ГэВ}  на  фотоэлектрон.  Таким образом   квантовые   флуктуации   сигнала   ограничивают   разрешение германиевых калориметров до величины 0,1 \% на МэВ, а кварцевых до \mbox{10 \% на ГэВ}.

Влияние    флуктуаций    ливневых    утечек    на    разрешение электромагнитного калориметра демонстрируется на рис. \ref{fig:leak}. Эти флуктуации имеют  не  Пуассоновский  характер  и  поэтому,  их  вклад  в  разрешение  не пропорционален $E^{-1/2}$. Оказывается  также,  что  для  заданного  положения ливня,  влияние  продольных  утечек  значительнее,  чем  поперечных.  Эти отличия объясняются различиями в числах различных частиц, ответственных за  утечки.  Например,  флуктуации  в  точке  начала  порожденного  фотоном ливня,  являются  флуктуациями  утечек,  за  которые  ответственна  только единственная  частица – инициирующий  фотон.  Боковые  утечки – это коллективный феномен, в который вносят вклад множество частиц.

В  отличие  от  продольных  и  поперечных  утечек,  третий  тип  утечек, альбедо,  или  же  обратные  утечки  через  тот  торец  детектора,  в  который попадает  инициирующая  частица,  не  может  быть  уменьшен  посредством изменения конструкции детектора. К счастью, эти утечки крайне невелики во всем  диапазоне  энергий,  кроме  очень  малых  значений. Результаты, показанные на рис. \ref{fig:leak}, получены путем Монте-Карло моделирования, но они подтверждаются многими экспериментами.

%%% lateral, lodgitudal and albedo fluctuations
\begin{figure}[H]
    \centering
    \includegraphics[width=0.75\textwidth]{7.jpg}
    \caption{Сравнение эффектов, вызванных ливневыми утечками различного рода. Результат симуляции в программе \textit{EGS4} [4]}
    \label{fig:leak}
\end{figure}

На практике разрешение конкретного калориметра определяется различными видами флуктуаций, каждый из которых имеет характерную энергетическую зависимость.  Обычно  эти  эффекты  некоррелированны,  значит  могут учитываться  в  виде  отдельных  слагаемых.  По  причине  различных зависимостей  от  энергии  общее  разрешение  калориметра  при  различных энергиях   может   быть   обусловлено   различными   флуктуационными эффектами.

%%% Fluctuations impact in the energy resolution
\begin{figure}[H]
    \centering
    \includegraphics[width=0.5\textwidth]{8.jpg}
    \caption{Энергетическое разрешение электромагнитного калориметра эксперимента \textit{ATLAS} [6]}
    \label{fig:resOnEn}
\end{figure}

Это  иллюстрируется  рис. \ref{fig:resOnEn},  на  котором  изображена  зависимость различных слагаемых, входящих в выражение энергетического разрешения, от  энергии для  электромагнитного  калориметра  эксперимента \textit{ATLAS}. Для энергий ниже \mbox{10 ГэВ} преобладающим фактором является электронный шум, между 10 и \mbox{100 ГэВ} – флуктуации сэмплирования и другие стохастические процессы,  при  энергиях  выше  \mbox{100  ГэВ}  независящие  от  энергии  эффекты влияют на энергетическое разрешение.

\subsection{Форма функции отклика} \label{chap2.3}

Не  все  виды  флуктуаций,  увеличивающие  отклонения  отклика, являются   симметричными   относительно   среднего   значения.   Ниже перечислены примеры эффектов, приводящих к несимметричным функциям отклика:

 \begin{itemize}[leftmargin=1.6\parindent, wide]
 	\item[---] Если сигнал сформирован очень маленьким числом частиц (например фотоэлектронами),  то  распределение  Пуассона  становится  ассиметричным. Эффекты  такого  рода  наблюдаются  в  сигналах  калориметров  с кварцевым оптическим волокном (рис. \ref{fig:poissons}).
 		\item[---] Эффекты ливневых утечек приводят к «хвостам» в распределениях сигналов.  Обычно,  такие  хвосты  возникают  в  распределении  сигналов  с низкоэнергетической стороны, потому что энергия покидает активный объем детектора.  Однако  существуют  примеры  детекторов,  в  которых  утечки приводят   к   усилению   сигнала.   Такое   явление   наблюдается   в сцинтилляционных  калориметрах,  в  которых  сцинтилляции  улавливаются кремниевыми диодами. Попадание в такой диод выбравшегося из детектора электрона может привести к тому, что величина сигнала окажется большей, чем от сцинтилляционного фотона (рис. \ref{fig:photPMT},а).
 \end{itemize}
 
 %%% Simmetric and asimmetric signals
\begin{figure}[H]
    \centering
    \includegraphics[width=1.0\textwidth]{9.jpg}
    \caption{Распределение сигналов для электромагнитных ливней, образованных электронами с энергией \mbox{10 ГэВ (а)} и \mbox{200 ГэВ (б)}, в калориметре эксперимента \textit{CMS} с кварцевыми оптическими волокнами [7]}
    \label{fig:poissons}
\end{figure}

 %%% Photodiode and PMT signals
\begin{figure}[H]
    \centering
    \includegraphics[width=1.0\textwidth]{10.jpg}
    \caption{Распределения сигналов от высокоэнергетичного электрона, образующего электромагнитный ливень в $\mathrm{PbWO}_{\mathrm{4}}$ кристальном калориметре. Для снятия сигнала используется кремниевый фотодиод (а) и ФЭУ (б) [8]}
    \label{fig:photPMT}
\end{figure}


