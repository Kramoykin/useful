\newpage  
\section{Моделирование энергетических утечек}  \label{chap6}  

Как было упомянуто в предыдущем разделе, одной из причин, обуславливающих необходимость добавления константного слагаемого к формуле (\ref{eq:delPeso}), могут являться неустранимые энергетические утечки. 

Увеличение геометрических размеров калориметра должно приводить к росту энергии, поглощенной в материале детектора. И, при достаточно больших размерах калориметра, ожидается достижение равенства между поглощенной энергией и энергией пучка, порождающего электромагнитный ливень. 

Было установлено, что при увеличении числа продольных сегментов от 150 до 1000 и увеличении радиуса детектора от \mbox{40 мм} до \mbox{200 мм} энергетические утечки в среднем уменьшились от \mbox{10 ГэВ} до \mbox{200 МэВ}. Однако, дальнейшее увеличение размеров детектора не приводило к снижению величины утечки. 

Геометрическая особенность калориметров типа "шашлык"{}, а именно -- чередование плоских пластин из материалов, физические свойства которых отличаются очень значительно, приводит к предположению о том, что гамма-кванты, попадая в сцинтиллятор под достаточно большими углами, относительно продольной оси калориметра могут, в результате многократного отражения от границы раздела сред сцинтиллятора и свинца, покидать детектор без поглощения (в особенности это касается мягких фотонов падающих на границу раздела под углом большим угла полного внутреннего отражения). 

Для проверки этой гипотезы были получены угловые распределения гамма-квантов, покидающих объем детектора для различных размеров детектора. Угловые распределения при энергии пучка \mbox{100 ГэВ} и \mbox{1000 продольных слоев} для малых и больших размеров детектора показаны на рис. \ref{fig:leakSmall}  и рис. \ref{fig:leakBig} соответственно.

 %%% Gamma leakege in small ECAl
\begin{figure}[H]
    \centering
    \includegraphics[width=0.5\textwidth]{leakage-small.png}
    \caption{Угловое распределение гамма-квантов, покидающих калориметр с радиусом \mbox{40 мм} (на графике указана ошибка, соответствующая одному $\sigma$)}
    \label{fig:leakSmall}
\end{figure}

 %%% Gamma leakege in big ECAl
\begin{figure}[H]
    \centering
    \includegraphics[width=0.5\textwidth]{leakage-big.png}
    \caption{Угловое распределение гамма-квантов, покидающих калориметр с радиусом \mbox{200 мм} (на графике указана ошибка, соответствующая одному $\sigma$)}
    \label{fig:leakBig}
\end{figure}

Можем заметить, что частицы действительно покидают калориметр под большими углами к продольной оси детектора, что подверждает догадку о многократном отражении внутри сцинтилляторных слоёв. Однако, среняя энергия, покидающая детектор, уменьшается более чем на два порядка при увеличении радиуса детектора от 40 до \mbox{200 мм}. При дальнейшем увеличении радиуса калориметра достигается полное отсутствие утечки гамма-квантов.

Таким образом, утечка гамма-квантов из калориметра не может приводить к независящим от энергии пучка флуктуациям. Единственной оставшейся причиной неустранимых утечек могут быть слабо взаимодействующие частицы, рождающиеся в фотоядерных реакциях, например -- нейтроны. 