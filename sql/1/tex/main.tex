% !TEX program = xelatex
%----------------------- Преамбула -----------------------
\documentclass[utf8x, 14pt, oneside, a4paper]{article}

\usepackage{extsizes} % Для добавления в параметры класса документа 14pt

% Для работы с несколькими языками и шрифтом Times New Roman по-умолчанию
\usepackage[english,russian]{babel}
\usepackage{fontspec}
\setmainfont{Times New Roman}

% ГОСТовские настройки для полей и абзацев
\usepackage[left=30mm,right=15mm,top=20mm,bottom=20mm]{geometry}
\usepackage{misccorr}
\usepackage{indentfirst}
\usepackage{enumitem}
\setlength{\parindent}{1.25cm}
\linespread{1.3}
\setlist{nolistsep} % Отсутствие отступов между элементами \enumerate и \itemize

% Дополнительное окружения для подписей
\usepackage{array}
\newenvironment{signstabular}[1][1]{
	\renewcommand*{\arraystretch}{#1}
	\tabular
}{
	\endtabular
}

% Переопределение стандартных \section, \subsection, \subsubsection по ГОСТу;
% Переопределение их отступов до и после для 1.5 интервала во всем документе
\usepackage{titlesec}

\titleformat{\section}[block]
    {\bfseries\normalsize}
    {\hspace{1.25cm}\thesection}
    {1ex}{}   
    
\titleformat{name=\section,numberless}[block]
  {\bfseries\normalsize}
  {}
  {0pt}
  {\hspace*{1.25cm}}
  
  \titleformat{name=\subsection,numberless}[block]
  {\bfseries\normalsize}
  {}
  {0pt}
  {\hspace*{1.25cm}}

\titleformat{\subsection}[block]
    {\bfseries\normalsize}
    {\hspace{1.25cm}\thesubsection}
    {1ex}{}

\titleformat{\subsubsection}[block]
    {\bfseries\normalsize}
    {\hspace{1.25cm}\thesubsubsection}
    {1ex}{}

% Работа с изображениями и таблицами; переопределение названий по ГОСТу
\usepackage[justification=centering]{caption}
\captionsetup[figure]{name={Pic},labelsep=endash}
\captionsetup[table]{singlelinecheck=false, labelsep=endash}

\usepackage{graphicx}
\usepackage{slashbox} % Диагональное разделение первой ячейки в таблицах
\usepackage{multirow}

% Цвета для гиперссылок и листингов
\usepackage{color}

% Гиперссылки \toc с кликабельностью
\usepackage{hyperref}

\hypersetup{
	linktoc=all,
	linkcolor=black,
	colorlinks=true,
}

% Листинги
\setsansfont{Arial}
\setmonofont{Courier New}

\usepackage{color} % Цвета для гиперссылок и листингов
\definecolor{comment}{rgb}{0,0.5,0}
\definecolor{plain}{rgb}{0.2,0.2,0.2}
\definecolor{string}{rgb}{0.91,0.45,0.32}

\usepackage{listings}
\lstset{
	basicstyle=\footnotesize\ttfamily,
	language=[Sharp]C, % Или другой ваш язык -- см. документацию пакета
	commentstyle=\color{comment},
	numbers=left,
	numberstyle=\tiny\color{plain},
	numbersep=5pt,
	tabsize=4,
	extendedchars=\true,
	breaklines=true,
	keywordstyle=\color{blue},
	frame=b,
	stringstyle=\ttfamily\color{string}\ttfamily,
	showspaces=false,
	showtabs=false,
	xleftmargin=17pt,
	framexleftmargin=17pt,
	framexrightmargin=5pt,
	framexbottommargin=4pt,
	showstringspaces=false,
	inputencoding=utf8x,
	keepspaces=true
}

\DeclareCaptionLabelSeparator{line}{\ --\ }
\DeclareCaptionFont{white}{\color{white}}
\DeclareCaptionFormat{listing}{\colorbox[cmyk]{0.43,0.35,0.35,0.01}{\parbox{\textwidth}{\hspace{15pt}#1#2#3}}}
\captionsetup[lstlisting]{
	format=listing,
	labelfont=white,
	textfont=white,
	singlelinecheck=false,
	margin=0pt,
	font={bf,footnotesize},
	labelsep=line
}
%%% Правильные отступы в подписях таблиц
\newlength\myindention
\DeclareCaptionFormat{myformat}%
{\hspace*{\myindention}#1#2#3}
\setlength\myindention{1.25cm}
\captionsetup{format=myformat}
\captionsetup[table]{justification = justified,format=myformat}

% Годные пакеты для обычных действий
\usepackage{ulem} % Нормальное нижнее подчеркивание
\usepackage{hhline} % Двойная горизонтальная линия в таблицах
\usepackage[figure,table]{totalcount} % Подсчет изображений, таблиц
\usepackage{rotating} % Поворот изображения вместе с названием
\usepackage{lastpage} % Для подсчета числа страниц

% Быстрый и грязный способ избежать ошибки с титульником
\renewcommand\maketitle{}

% Директория с рисуночками
\graphicspath{ {../pic} }

%%% Запрет деления слов (перенос слова на новую строку)
\tolerance=1
\emergencystretch=\maxdimen
\hyphenpenalty=10000
\hbadness=10000

%%% Делаем рисунки плавающими
\usepackage{float}

%%% Доп математика
\usepackage{mathtools}

%%% Косая черта в таблицах
\usepackage{diagbox}

% Значок градусов
\usepackage{gensymb}

% Вставяем титульники
\usepackage{pdfpages}

% ---------------------- Документ ---------------------- 
\begin{document}

\newpage
\section{Физические процессы в калориметрах}  \label{chap1}
 
 Существует несколько процессов, играющих роль в развитии электромагнитного ливня. Электроны и позитроны теряют энергию на ионизацию и излучение. Первый процесс преобладает при низких энергиях, второй – при высоких. Критическая энергия, при которой роль обоих процессов одинакова в первом приближении обратно пропорциональна зарядовому числу вещества поглотителя [4]:
 
%%% Kritical energy formula (from Wigmans)
\begin{equation}\label{eq:kriten}
\epsilon_c = \frac{610\ \mathrm{MeV}}{Z+1,24}
\end{equation}
 
Фотоны взаимодействуют с веществом преимущественно посредством фотоэффекта, Комптоновского рассеяния и эффекта образования электро-позитронных пар. Фотоэффект преобладает при низких энергиях, а образование пар при высоких. Соответствующие сечения зависят от {\textit Z}. Например, сечение фотоэффекта пропорционально $Z^{5}$ и $E^{-3}$, в то время как сечение образования пар постепенно возрастает с ростом {\textit Z} и {\textit E}, асимптотически приближаясь к определенному значению при значениях энергии порядка \mbox{1 ГэВ}. Угловое распределение более или менее изотропно для фото- и комптоновских электронов, но имеет строго преобладающее направление для электронов и позитронов, рожденных в результате образования пар.

Начиная с энергий от \mbox{1 ГэВ} и выше, электроны и фотоны образуют электромагнитный ливень в веществе, в которое они проникают. Электроны теряют свою энергию преимущественно на излучение, фотоны с наибольшей энергией, рожденные в этом процессе, конвертируются в электрон-позитронные пары, которые также излучают фотоны и т. д.  Число частиц, образующихся в развивающемся ливне достигает максимума на определенной глубине внутри поглотителя, после чего постепенно убывает. Глубина, на которой ливень достигает своего максимума, логарифмически возрастает с увеличением энергии инициирующей ливень частицы.

%%% Deposited energy graphic (from Wigmans)
\begin{figure}[H]
    \centering
    \includegraphics[width=0.75\textwidth]{1.jpg}
    \caption{Энерговыделение как функция глубины для электромагнитных ливней, образованных 1, 10, \mbox{100 ГэВ} и \mbox{1 ТэВ} электроном в медном поглотителе (а). Радиальное распределение энергии, оставленной \mbox{10 ГэВ} электроном в меди на различных глубинах (б). Результат симуляции в программе \textit{EGS4} [4]}
    \label{fig:enDep}
\end{figure}

Поперечное развитие ливня обусловлено множественным рассеянием электронов и позитронов, а так же изотропным и несоосным с ливнем характером рождения фотонов и позитронов.

Первый процесс преобладает на ранних стадиях развития ливня, а второй после того, как ливень преодолевает максимум. Влияние обоих процессов хорошо демонстрируется рис.  \ref{fig:enDep}(б), на котором изображена радиальная плотность энергии ливня, порожденного электроном в меди, на трех различных глубинах внутри калориметра.

Развитие ливня может быть описано более или менее независимо от материала поглотителя в терминах радиационной длины $X_0$ (в продольном направлении) и Мольеровского радиуса $\mathrm{\rho}_M$ (в поперечном направлении). Радиационная длина различается для фотонов и электронов достаточно существенно. Ливни, инициированные высокоэнергетичным  электроном или же фотоном, имеют существенные отличия. Попадая  в материал, высокоэнергетичные электроны начинают излучать немедленно. На их пути, через первые несколько миллиметров материала, они могут испускать тысячи фотонов тормозного излучения. С другой стороны, высокоэнергетичный фотон может как провзаимодействовать с веществом на такой же длине, так и не провзаимодействовать.  В последнем случае он не потеряет энергии совсем, а в случае взаимодействия, может потерять даже больше, чем электрон в таком же количестве материала. Такая разница иллюстрируется на рис. \ref{fig:gamAndEnFraq}. В одинаковом количестве материала электроны теряют большую долю своей энергии, чем фотоны, но разброс энергетических потерь у фотонов больше.

%%% gamma and electron en fraction (from Wigmans)
\begin{figure}[H]
    \centering
    \includegraphics[width=0.75\textwidth]{2.jpg}
    \caption{Распределение доли энергии, оставленной при прохождении пяти радиационных длин электроном и фотоном с энергиями \mbox{10 ГэВ} [4]}
    \label{fig:gamAndEnFraq}
\end{figure}

В первом приближении профиль электромагнитного ливня в основном определяется значениями $X_0$ и $\mathrm{\rho}_M$,  но  такое  приближение неидеально,  что демонстрируется на рис. \ref{fig:elEnDepMat},  где  показано  энерговыделение  на  единице радиационной длины. Такое отличие проще понять посредством того факта, что количество  вторичных  частиц  после  максимума  ливня  начинает уменьшаться,  и  это  снижение  медленнее  в  материалах с высоким {\textit Z}. Например,  высокоэнергетичный  электрон  порождает  в  свинце  в  три  раза больше  позитронов,  чем  в  алюминии.  В  результате  нужно  больше радиационных  длин  свинца,  чем  алюминия,  чтобы  уместить  в  себе  \mbox{99  \%} ливня. К тому же, максимум ливня располагается в материале с высокими {\textit Z} на большей глубине.

%%% energy deposited by electron in different materials (from Wigmans)
\begin{figure}[H]
    \centering
    \includegraphics[width=0.65\textwidth]{3.jpg}
    \caption{Продольные профили электромагнитных ливней, образованных электроном с энергией \mbox{10 ГэВ} в алюминии, железе и \mbox{свинце [4]}}
    \label{fig:elEnDepMat}
\end{figure}

%%% average shower fraction for different energies and materials (from Wigmans)
\begin{figure}[H]
    \centering
    \includegraphics[width=0.65\textwidth]{4.jpg}
    \caption{Средняя доля энергии, выделившейся в блоке материала в результате прохождения его частицей. Показаны результаты для ливней, образованных электронами различных энергий в медном поглотителе (а) и результаты для ливней, образованных электроном с энергией \mbox{100 ГэВ} в различных поглотителях (б). Результат симуляции в программе \textit{EGS4} [4]}
    \label{fig:fracEnAndMat}
\end{figure}

Зависимость  толщины  калориметра,  необходимой  для  того,  чтобы вместить   электромагнитный   ливень,   образованный   электроном,   для различных материалов представлена на рис. \ref{fig:fracEnAndMat}(б). При поглощении \mbox{99 \%} электромагнитного  ливня разница  между  материалом  с  высоким {\textit Z} и материалом  с  низким {\textit Z} может  достигать  десяти $X_0$. И  по  причинам, описанным  выше,  нужно  еще  больше  материала,  чтобы  поглотить  ливень, образованный фотоном. Зависимость доли поглощенной энергии от толщины калориметра изображена на рис. \ref{fig:fracEnAndMat}(а). Для поперечной локализации ливня энергетическая  зависимость  отсутствует,  а различия  в  материале  не  такие значительные,  как  для  продольного. Достаточно  длинный  цилиндр таким образом  поглотит  одинаковую  долю  энергии  от \mbox{1  ГэВ}  ливня  и  от \mbox{1  ТэВ} ливня.

Нарушения пропорциональности, рассмотренные на рис. \ref{fig:elEnDepMat} и \ref{fig:fracEnAndMat}, вызваны явлениями, которые происходят при энергиях ниже критической. Например, в  свинце  более  \mbox{40 \%}  энергии  ливня оставляется  в  веществе частицами  с энергиями ниже \mbox{1 МэВ}, в то время как критическая энергия имеет значение примерно \mbox{7 МэВ}. Только четверть энергии оказывается внесена позитронами, остальное  приносят  электроны.  Эти  факты получены  в  результате  Монте-Карло  симуляции  развития  ливня, и показывают,  что  комптоновское рассеяние  и  явление  образования  пар в  основном  ответственны  за формирование профиля ливня. Оба процесса преобладают на энергиях ниже критической,  а  значит,  недостаточно  точно  описываются  терминами радиационной длины $X_0$ и Мольеровского радиуса $\mathrm{\rho}_M$.

\end{document}   
